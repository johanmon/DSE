\documentclass[a4paper, 11pt]{article}

\usepackage{ifthen}
\usepackage[latin1]{inputenc}


\newcommand{\nnsection}[1]{
\section*{#1}
\addcontentsline{toc}{section}{#1}
}

\begin{document}

\begin{center}
\vspace{20pt}
\textbf{\large Toky: a token based distributed mutual-exclusion lock}\\
\vspace{10pt}
\textbf{Johan Montelius}\\
\vspace{10pt}
\today{}
\end{center}

\nnsection{Introduction}

Your task is to implement a distributed mutual-exclusion lock. The
lock will use a token that is the key to enter the critical
section. In this tutorial you could take advantage of the environment
implemented in the Muty seminar. The worker and gui modules could be
adapted and be used to experiment and visualize the final solution. In the
description below we will only describe the implementation of the lock
algorithm.

\section{The architecture}

The architecture consists of a set of workers that needs to coordinate
their entry of a critical section. Each worker is communicating with
its own lock process and the lock processes determine in what order
the workers should be allowed to enter the critical section.


\section{the lock}

The algorithm that we are to implement is by Neilsen-Mizuno and is a
token based algorithm. There is only one token in the system and this
is passed between processes that request to move into the critical
section.

The first token based algorithm that one learns is probably some
variant of a token-ring algorithm. A set of lock processes are then
linked in a virtual ring and pass a token around in the ring. A
process that needs to move into the critical section will then wait
for the token and hold it until it leaves the critical section. In
the algorithm by Neilsen-Mizuno is more efficient and a lot more fun
to implement.

\subsection{the node}

A lock process, or a node, in the system can hold a token. We will
make sure that there is only one token in the system and this is of
course the most important invariant to uphold. A worker can order a
lock process to take the token, once it is taken the token is handed
over to the worker. When the worker exits from the critical section
the token is returned to the lock process. 

The difference from a token-ring algorithm is that a lock process will
keep the token in order for the worker to request it again. This can
continue until another process requests the token. How the token is
passed around between lock processes is the interesting part of the
algorithm.

\subsection{the row}

The lock processes are linked in a row where nodes have links only in
one direction. One node is the root node and the nodes in the row have
their pointers set towards this root. When a node requests the lock it
will take over the root role and make the nodes between itself and
the previous root point towards it. 

An invariant of the algorithm is that if one follows the links one will 
find the root node that holds or will eventually hold the
token. A simple initial configuration is to link all nodes in one row
and let the node in the end of the row be the root node and hold the token.

The nodes that have requested the lock are also linked together in a
secondary chain. They form a chain of nodes where the first node holds
the token. The last node in this chain is also the root node of the
tree. The token will move from node to node following the chain and
thus eventually end up at the root node.

If a process is asked by its master process to take the lock, the process
will send a request message towards the root. The message has the
following structure:

\begin{itemize}
\item {\tt \{request, From, Requester\}}: where {\tt From} is the send
  of the message and {\tt Requester} is the requester of the lock.
\end{itemize}

After having sent the request message the node becomes the new root 

A node that receives a request message will forward the message
towards the root but replace the {\tt From} field to the process
identifier of the node itself. It will also change its root pointer to
point to the send of the message. The root pointer is changed since
the requester will eventually receive the token and we need to
maintain the invariant. Keeping the root pointer as is is dangerous
since the node at the end of the path will when it moves out of the
critical section forward the token directly to the requester. 

A root node that receives a request message will as any node set its
root pointer to the sender of the message but must also make sure that
the requester eventually gets the token. The procedure now depends on
the state of the node: if the node holds the token it is sent directly
to the requester otherwise the process identifier of the requester is
kept as the next process to send the token to once the master of the
node returns the token.

\subsection{the queue}

A worker that has ordered its lock process to take the lock will be
suspended waiting for the process to hand it the token. The lock
process will eventually receive the token and then hand it over to the
worker even if it has another lock process waiting for the token. 

A node that has requested the lock knows that it will either receive
the token directly from a root node holding the token or be linked to
by a root node that will eventually receive the token. The path of
linked nodes that have requested the lock thus forms a queue of
suspended locks. 


\section{the implementation}


A lock process, or node, has only four properties: 

\begin{itemize}

\item {\bf Root}: is true if the node is the root node, otherwise it is a
 process identifier of a node closer to the root.

\item {\bf Last}: is true if the node is the last in a chain of
  waiting nodes, otherwise it is a process identifier of the next node
  in the chain.

\item {\bf Suspended}: is false if the master process is not suspended
  waiting for the token otherwise it is the process identifier of the
  waiting master process.

\item {\bf Token}: true is the node is holding the toke, false otherwise.

\end{itemize}

The names of the properties are carefully chosen so that we can write
tests in a logical way. If we want to test if a node is a root node we
will write:

\begin{verbatim}
    if
      Root ->
           do this;
      true ->
           do this where Root is the pointer towards the root
    end
\end{verbatim}

You will see that the code becomes quite compact.

Let's look at the messages. We only need two messages from the worker
to the lock process. One to order the process to take the lock and one
to release the lock. The lock process need only one message to reply
to the worker.

\begin{itemize}
\item {\tt \{take, Master\}}: the lock is orderd by the master to take the lock
\item {\tt taken}: the lock notifies the master that the lock is taken
\item {\tt release}: the master releases the lock 
\end{itemiize}

The messages between lock processes are also very simple. Either a
process forwards a request or it hands over the token.

\begin{itemize}
\item {\tt \{request, From, Requester\}}: a request from the
  Requester, delivered by From.
\item {\tt token}: the token
\end{itemiize}

The let's look at the algorithm; if a node receives a request three
things could happen. If the node is a root and holds the token the
token is simply passed to the requester. If the root is not holding
the token we know that the token will eventually be passed to the node
so we add the requester as the next node in the chain of waiting
nodes. It is important to realize that the root node is the last node
in the chain since the request arrived. 

\begin{verbatim}
	{request, From, Requester} ->
	    if Root -> 
		if Token ->
		    Requester ! token,
		    node(From, Last, Suspended, false);
		true ->
		    node(From, Requester, Suspended, false)
		end;
	     true ->
		Root ! {request, self(), Requester},
		node(From, Last, Suspended, Token)
	    end;
\end{verbatim}

If the node is not a root node we simply pass the request forward but
replace the sender field to the process identifier of the node. In
all cases we change our root pointer to the process identifier of the
sender of the request.

If the node receives the token it is sent from another lock process
that has forwarded the token along the chain of waiting nodes. The
only reason that a node is given the token is that its master is
suspended waiting for the lock. The node can then simply inform the
master that the lock is taken


if the master process is
suspended it should be given the token otherwise we must check if we
are the last node in the chain if waiting nodes. If we are the last
node we simply keep the token but if we have a node waiting we pass
the token along.

\begin{verbatim}
	token ->
	    Suspended ! taken,
	    node(Root, Last, false, false)
	end;
\end{verbatim}

The request from a master process to take the lock is divided into two
cases. If the node holds the token then it can be handed over directly
to the master process. Otherwise a request is sent towards the
root. The node then becomes a root node in a new tree. The process
identifier of the master is added as suspended.

\begin{verbatim}
	{take, Master} ->
	    if
		Token ->
		    Master ! taken,
		    node(Root, Last, false, false);
		true ->
		    Root ! {request, self(), self()},
		    node(true, Last, Master, Token)
	    end;
\end{verbatim}

When the master releases the lock 


\section{performance}

The main advantage of the algorithm is that a lock process can hold
the token as long as no other process is requesting the lock. The
delay obtaining the lock is then constant and so is the release
operation.

\begin{itemize}
\item What is the average number of messages needed to request the lock? 
\item How does this wary with load? 
\item What is the synchronization delay?
\item In what scenarios is it a good algorithm to use?
\end{itemize}

\end{document}
