\documentclass[a4paper,11pt]{article}

\usepackage{ifthen}
\usepackage[latin1]{inputenc}


\newcommand{\nnsection}[1]{
\section*{#1}
\addcontentsline{toc}{section}{#1}
}

\begin{document}

\begin{center}
\vspace{20pt}
\textbf{\large Paxy: the paxos protocol}\\
\vspace{10pt}
\textbf{Johan Montelius}\\
\vspace{10pt}
\today{}
\end{center}

\nnsection{Introduction}

This exercise will give you the opportunity to learn the Paxos
algorithm for gaining consensus in a distributed system. You should
know the basic operations of the algorithm but you do not have to know
all the details, that is the purpose of this exercise.

The code given is not complete, we use {\tt ...} etc to indicate that
you have to fill in the missing pieces. 

\section{Paxos}

The Paxos algorithm has three different processes: proposers,
acceptors and learners. The functionality of all three is often
included in one process but it will be easier to implement the
proposer and acceptor as two separate processes. The learner process
will not be implemented since it is not needed to reach a
consensus. In a real system it is of course important to also know the
outcome of the algorithm but we will do without learners.

\subsection{sequence numbers}

We will need some basic support to handle sequence numbers. Since
proposers need unique sequence numbers we need a way to generate and
compare sequence numbers. One way of guarantee uniqueness is to use a
tuple and let the first element be an, per proposer, increasing
integer and the second an identifier unique for the proposer. We build
a small {\tt order} module that can be found in the appendix to this
description. It will be quite easy to see what we mean when we use the
exported functions.

\subsection{the acceptor}

Let's start with the acceptor. The acceptor has a state consisting of:

\begin{itemize}
\item {\bf Key}: a unique Key (atom) of the acceptor,
\item {\bf Promise}: promised not to accept any ballot below this number
\item {\bf Voted}: the highest ballot number accepted
\item {\bf Accepted}: the value that has been accepted
\end{itemize}

Note that an acceptor can accept many values during the execution but
we must remember the value with the highest ballot number. 

When we start an acceptor we have not promised anything nor accepted a
value so the {\tt Promise} and {\tt Voted} parameters are instantiated
to null sequence numbers that are lower than any other sequence
number. The {\tt Accepted} parameter is initialized to {\tt na} to
indicate that it is {\it not applicable}.

The initialization of the acceptor will look as follows (we will add
things later on but this is ok for now). 

\begin{verbatim}
-module(acceptor).

-export([start/1]).

start(Key) ->
    spawn(fun() -> init(Key) end).

init(Key) ->
    Promise = order:null(), 
    Voted = order:null(),
    Accepted = na,
    acceptor(Key, Promise, Voted, Accepted).
\end{verbatim}

The acceptor is a process that is waiting for two types of messages: a
prepare requests and accept requests. A prepare request, {\tt
  \{prepare, Proposer, Round\}} will result in a promise, if we have
not made any promises that prevents us to make such a promise. The
round number of the prepare request must be compared with the {\tt
  Promise} already given. If the round number is higher we return a
promise, {\tt \{promise, Round, Voted, Accepted\}}. It is of course
very important that we in this message return the current accepted
value and in which round we voted for this value.

If we can not give a promise we do not have to do anything but it
could be polite to send an {\tt sorry} message. If we really want to
make life hard for the proposer we could even send back a promise. If
we have promised not to vote in round lower than round $17$, we could
of course promise not to vote in a round lower than $12$. The proposer
will of course take our promise as an indication that it is possible
for us to vote for a value in round $12$ but that will of course not
happen. To help the proposer we should inform it that we have have
promised not to vote in the round requested by the proposer (we could
even inform the proposer what we have promised but let's keep thing
simple).

\begin{verbatim}
   {prepare, Proposer, Round} ->
       case order:gr(..., ...) of
           true ->
               ... ! {promise, ..., ..., ...},
               acceptor(Name, ..., Voted, Accepted);
           false ->
               ... ! {sorry, ...},
               acceptor(Key, ..., Voted, Accepted)
       end;
\end{verbatim}


The accept request, sent by a proposer when it has received accept
messages from a majority, also have two outcomes; either we can accept
the request and then cast our vote in the ballot or we have a promise
that prevents us from accepting the request. Note that we do not
change our promise just because we vote for a new value.

Again, if we cannot accept the request we could simply ignore the
message but it is polite to inform the proposer.

\begin{verbatim} 
    {accept, Proposer, Round, Proposal} ->
        case order:goe(..., ...) of
            true ->
                ... ! {vote, ...},
                case order:goe(..., ...) of
                    true ->
                        acceptor(Name, Promise, ..., ...);
                    false ->
                        acceptor(Name, Promise, ..., ...)
                end;                            
            false ->
                ... ! {sorry, ...},
                acceptor(Name, Promise, ..., ...)
        end;
\end{verbatim}

Nothing prevents an acceptor to accept a value in round $17$ and then
accept another value if ask to do so in round $12$ (provided of course
that it has not promised not to do so). This is a very strange
situation but it is allowed. If we accept a value in a lower round we
should of course still remember the value of the highest ballot number.

We also include a message to terminate the acceptor. You can also add
messages for status information, a catch all clause etc. Also add
print out statements so that you can track what the acceptor has done.

\begin{verbatim}
    stop ->
       ok;
\end{verbatim}


\subsection{the proposer}

The proposer work in rounds, in each round it will try to get
acceptance of a proposed value or at least make the acceptors agree on
any value. If this does not work it will try again and again but each
time with a higher round number. 

\begin{verbatim}
-module(proposer).

-export([start/4]).

-define(timeout, 200).
-define(backoff, 10).
-define(delay, 20).

start(Key, Proposal, Acceptors, Seed) ->
    spawn(fun() -> init(Key, Proposal, Acceptors, Seed) end).

init(Key, Proposal, Acceptors, Seed) ->
    random:seed(Seed, Seed, Seed),
    Round = order:one(Name),
    round(Key, ?backoff, Round, Proposal, Acceptors).
\end{verbatim}

In a round the proposer will wait for accept and vote messages for up
to {\em timeout} milliseconds. If it has not received the necessary
number of replies it will abort the round. It will then back-off an
increasing number of milliseconds before starting the next round. It
will try its best to get the acceptors to vote for a proposal but as
you will see it will be happy if they can agree on anything. The {\tt
  delay} will be used to introduce a slight delay in the system to
make simulations more interesting.

Each round consist of one ballot attempt. The ballot either succeeds or
aborts, in which case a new round is initiated.

\begin{verbatim}
round(Key, Backoff, Round, Proposal, Acceptors) ->
    case ballot(..., ..., ...) of
        {ok, Decision} ->
            io:format("~w decided ~w in round ~w~n", [..., ..., ...]),
            {ok, ...};
        abort ->
            timer:sleep(random:uniform(...)),
            Next = order:inc(...),
            round(..., (2*...), ..., ..., ...)
    end.
\end{verbatim}

A ballot is initialized by multi-casting a prepare message to all
acceptors. The process then collects all promises and also the
accepted value with the highest sequence number so far. If we receive
promises from a quorum (a majority) we start the voting process by
multi-casting an accept message to all acceptors in the quorum. In the
accept message we include the value with the highest sequence number
accepted by a member on the quorum.

\begin{verbatim}
ballot(Round, Proposal, Acceptors) ->
    prepare(..., ...),
    Quorum = (length(...) div 2) +1,
    Max = order:null(),
    case collect(..., ..., ..., ...) of
        {accepted, Value} ->
            accept(..., ..., ...),
            case vote(..., ...) of
                ok ->
                    {ok, ...};
                abort ->
                    abort
            end;
        abort ->
            abort
    end.
\end{verbatim}

The collect procedure will simply receive promises and, if no acceptor
has any objections, learn the so far accepted value with the highest
ballot number. Note that we need a time out since acceptors could take
forever or simply refuse to reply. Also note that we have tagged the
sent request with the sequence number an only accept replies with the
same sequence number, also that we need a catch all alternative since
there might be delayed messages out there that otherwise would just
stack up.

\begin{verbatim}
collect(0, _, _, Proposal) ->
    ...;
collect(N, Round, Max, Proposal) ->
    receive 
        {promise, Round, _, na} ->
            collect(..., ..., ..., ...);
        {promise, Round, Voted, Value} ->
            case order:gr(..., ...) of
                true ->
                    collect(..., ..., ..., ...);
                false ->
                    collect(..., ..., ..., ...)
            end;
        {promise, _, _,  _} ->
            collect(..., ..., ..., ...);
        {sorry, Round} ->
            collect(..., ..., ..., ...),
        {sorry, _} ->
            collect(..., ..., ..., ...)
    after ?timeout ->
            abort
    end.
\end{verbatim}

Collecting votes is almost the same procedure. We are only waiting for
votes and need only count them until we have received them all. If
we're unsuccessful we abort and hope for better luck next round.

\begin{verbatim}
vote(0, _) ->
    ...;
vote(N, Round) ->
    receive
        {vote, Round} ->
            vote(..., ...);
        {vote, _} ->
            vote(..., ...);
        {sorry, Round} ->
            vote(.., ...),
        {sorry, _} ->
            vote(..., ...)
    after ?timeout ->
            abort
    end.
\end{verbatim}

The only things that is left is to implement the sending of
requests. The prepare request will send the name of the acceptor as
part of the message. This name is returned by the acceptor and can then
be collected to identify the acceptors in the quorum. 

\begin{verbatim}
prepare(Round, Acceptors) ->
    Fun = fun(Acceptor) -> send(Acceptor, {prepare, self(), Round}) end,
    lists:map(Fun, Acceptors).

accept(Round, Proposal, Acceptors) ->
    Fun = fun(Acceptor) -> send(Acceptor, {accept, self(), Round, Proposal}) end,
    lists:map(Fun, Acceptors).
\end{verbatim}

Sending a message is of course trivial but we will, for reasons
described later, implement it in a separate procedure.

\begin{verbatim}
send(Name, Message) ->
    Name ! Message.
\end{verbatim}


\section{Experiment}

Let's set up a test and see if a set of acceptors can agree on
something. We start five acceptors and have three proposers. The
proposers try to make the acceptors vote for their suggestion. The
proposers will hopefully find a quorum and then learn the agreed
value. A test module will help us set up the experiments.

\begin{verbatim}
start(Seed) ->
    register(a, acceptor:start(a)),
    register(b, acceptor:start(b)),
    register(c, acceptor:start(c)),
    register(d, acceptor:start(d)),
    register(e, acceptor:start(e)),
    Acceptors = [a,b,c,d,e],
    proposer:start(kurtz, green, Acceptors, Seed+1),
    proposer:start(willard, red, Acceptors, Seed+2),
    proposer:start(kilgore, blue, Acceptors, Seed+3),
    true.
\end{verbatim}

Since the acceptors stay alive even if a decision has been made we
need to terminate them explicitly. The code below becomes useful
during debugging since a crashed acceptor will be de-registered (and
sending a message to an unregistered name will cause an exception).

\begin{verbatim}
stop() ->
    stop(a),
    stop(b),
    stop(c),
    stop(d),
    stop(e).

stop(Name) ->
    case whereis(Name) of
        undefined ->
            ok;
        Pid ->
            Pid ! stop
    end.
\end{verbatim}

Add code to trace each state transition in the acceptor and
proposer. Try to follow the execution and the progress of the
algorithm. 

Do some experiments and try to introduce delays in the
acceptor. Insert larger delays and see if the algorithm still
terminates.

Could you even come to an agreement when you ignore messages? Try
ignoring to send {\tt sorry} messages or simply randomly drop a
vote. If you drop too many messages a quorum will of course never be
found but we could probably loose quite many. Does the algorithm ever
report conflicting answers?

What happens of we increase the number of acceptors to say $9$ or
$17$? Will we reach a decision? What if we have also have $10$
proposers?


\section{Fault tolerant}

In order to make the implementation fault tolerant we need to
remember what we promise and what we vote for. If we use the module
{\tt pers} given in the appendix we can initialize our state to the
state we had when we crashed and store state changes as we make promises.
Where in the acceptor should we add this? 

We also have to be careful when we send a message to an acceptor. We
should first check that the acceptor is actually registered, if not it
means that the acceptor is down. If we knew that the acceptor was
registered on a remote node we could ignore this procedure since
sending a message to a remote process always succeeds. If the acceptor
is a locally registered process the send operation could throw an
exception, something that we want to avoid.

\begin{verbatim}
send(Name, Message) ->
    case whereis(Name) of 
        undefined ->
            down; 
        Pid -> 
            Pid ! Message,
            timer:sleep(random:uniform(?delay))
    end.
\end{verbatim}

Simulate a crash and restart using the procedure below (in the {\tt
  test} module and see if the protocol still comes to a
  consensus. You might have to increase the sleep period in the
  acceptor to make the execution run slower.

\begin{verbatim}
crash(Name) ->    
    case whereis(Name) of 
        undefined ->
            ok;
        Pid ->
            unregister(Name),
            exit(Pid, "crash"),
            register(Name, acceptor:start(Name))
    end.
\end{verbatim}


\section{Carrying on}

There are some improvements that could be made in the implementation
of the proposer. If we need three promises for a quorum and we have
received three {\tt sorry} messages from the in total five acceptors
then we can abort the ballot. Change the code of of the {\tt
  collect/4} and {\tt vote/2} procedures to also keep track of how
many messages in total there are still out there. 

As you have probably noticed the decision process is fast if we only
have one active proposer. Can we have an election in the beginning and
decide who is to be the active proposer?

In the above implementation there is only one decision being made. A
more practical system would of course have a series of decisions to
make. Could we implement an acceptor that is willing to accept
sequences of values. 

A proposer would then be sent a value that it should try to add to the
sequence. It will of course still play by the rules and accept that
other values could be added before its own value. 


\nnsection{Appendix: order}

\begin{verbatim}
-module(order).

-export([null/0, one/1, gr/2, goe/2, inc/1]).

null() ->
    {0,0}.

one(Id) ->
    {0, Id}.

gr({N1,I1}, {N2,I2}) ->
    if 
        N1 > N2 ->
            true;
        ((N1 == N2) and (I1 > I2)) ->
            true;
        true ->
            false
    end.

goe({N1,I1}, {N2,I2}) ->
    if 
        N1 > N2 ->
            true;
        ((N1 == N2) and  (I1 >= I2)) ->
            true;
        true ->
            false
    end.

inc({N, Id}) ->
    {N+1, Id}.
\end{verbatim}

\nnsection{Appendix: pers}

\begin{verbatim}
-module(pers).

-export([read/1, store/4, delete/1]).

read(Id) ->
    {ok, Id} = dets:open_file(Id, []),
    case dets:lookup(Id, perm) of
        [{perm, Bn, An, Av}] ->
            {Bn, An, Av};
        [] ->
            {order:null(), order:null(), na}
    end.

store(Id, Bn, An, Av)->
    dets:insert(Id, {perm, Bn, An, Av}).
        
delete(Id) ->                  
    dets:delete(Id, perm),
    dets:close(Id).
\end{verbatim}

\end{document}
