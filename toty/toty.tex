\documentclass[a4paper,11pt]{article}

\usepackage{ifthen}
\usepackage[latin1]{inputenc}

\usepackage{alltt}


\newcommand{\nnsection}[1]{
\section*{#1}
\addcontentsline{toc}{section}{#1}
}

\begin{document}

\begin{center}
\vspace{20pt}
\textbf{\large Toty: a total order multicast}\\
\vspace{10pt}
\textbf{Johan Montelius}\\
\vspace{10pt}
\today{}
\end{center}

\nnsection{Introduction}

The task is to implement a total order multicast service using a
distributed algorithm. The algorithm is the one used in the ISIS
system and is based on requesting proposals from all nodes in a group.

\section{The architecture}

We will have a set of workers that communicate with each other using
multicast messages. Each worker will have access to a multicast
process that will hide the complexity of the system. The multicast
processes are connected to each other and will have to agree on an
order on how to {\em deliver} the messages. There is a clear
distinction between receiving a message, which is done by the
multicast process, and delivering a message, which is when the worker
sees the message. A multicast process will receive messages in
unspecified order but only deliver the messages to its worker in a
total order i.e. all workers in the system will deliver the messages
in the same order.

\subsection{the worker}

A worker is in our example a process that at random intervals wish to
send a multicast message to the group. The worker will wait until it
sees its own message before it sends another one to prevent an
overflow of messages in the system.

The worker will be connected to a gui process that is simply a colored
window. The window is initially black or in RGB talk {\tt \{0, 0,
  0\}}. This is also the initial state of the worker. Each message
that is delivered to the worker is a integer N in some interval, say 1
to 20. A worker will change its state by adding N to the R value and
rotate the values. If the state of the worker is {\tt \{R, G, B\}}
and the worker is delivered message N, the new state is {\tt \{G, B,
  (R+N) rem 256\}}.

The color of a worker will thus change over time and the order of
messages is of course important. The sequence 5, 12, 2 will of course
not create the same color as the sequence 12, 5, 2. If all workers
start with a black color and we fail to deliver the messages in the
same order the colors of the workers will start to diverge.

We will experiment with different implementations and to prepare for
the future we make the initialization a bit cumbersome. To start with
we provide some parameters to the worker to make experiments easier to
manage. We can change the module of the multicast process and we can
experiment with different values of sleep and jitter time. The sleep
time is for up to how many milliseconds the workers should wait
until the next message is sent and the jitter time is a parameter to
the multicast process.

When started, the worker should register with a group manager and will
be given the initial state and the process identifier of the other
members in the group. This might look like over-kill but we will use
it in the coming seminars.


\subsection{basic multicast}

As a first experiment we should use a process that implements basic
multicast. The multicaster is simply give a set of peer processes and
when it is told to multicast a message the message is simply sent to
all the peers, one by one. 

To make the experiments more interesting we include a jitter parameter
when we start the process. The process will wait up to this many
milliseconds before sending the next message. This will allow messages
to interleave and possibly cause problems for the workers.

\subsection{experiment}

Experiment with different values for sleep and jitter and see
when messages are not delivered in total order.

Ponder what would happen if we had a system that relied on total order
delivery but that this was not clearly stated. If congestion was low
and we did not have any delays in the network, how long time does it
take before messages are delivered out of order? How hard would it be
to debug this system and figure out what went wrong?

\section{total order multicast}

The multicast process is of course the tricky part. We will here go
through the code but you will have to do some programming yourself.

The main procedure has the following state:

\begin{itemize}
\item Master: the process to which messages are delivered
\item Next: the next value to propose
\item Nodes: all peers in the network
\item Cast: a set of references to messages that have been sent out
  but no final sequence number have yet been assigned
\item Queue: messages that have been received but not yet delivered
\item Jitter: the jitter parameter that induces some network delay
\end{itemize}

The sequence numbers are represented by a tuple {\tt\{N, Id\}}, where
\{N\} is an integer that is incremented every time we make a proposal
and {\tt Id} is our process identifier. 

The {\tt Cast} set is represented a s list of tuples {\tt \{Ref, L,
  Sofar\}}, where {\tt L} is the number of proposals that we a re
still waiting for and {\tt Sofar}, the highest proposal received so
far. 

The {\tt Queue} is an ordered list of entries represented messages
that we have been received but for which no agreed value exist. The
list is ordered based in the proposed or agreed sequence number. The
proposed entries are entries where we have proposed a sequence
number. If we have entries with agreed sequence numbers at the front
of the queue these can be removed and delivered to the worker.

\subsection{sending of a message}

A send message is a directive to multicast a message. We first have to
agree in which order to deliver the message and therefore send a
request for proposals to all peers. 

The request should be sent to all nodes with a unique reference.  This
reference is also added to the casted set with information on how many
nodes that have to report back. We're leaving {\tt ?} at
places in the code where you have to fill in the right values.

\begin{verbatim}
{send, Msg} ->
    Ref = make_ref(),
    request(?, ?, ?, ?),
    Cast2 = cast(?, ?, ?),
    server(Master, Next, Nodes, Cast2, Queue, Jitter);
\end{verbatim}

Note that we're also sending a request to ourselves. We will handle our
own proposal the same way as proposals from everyone else. This might
look strange but it makes the code much easier.

\subsection{receiving a request}

When the process receives a request it should reply with a new
sequence number. It should also queue the message using the proposed
sequence number as key.

\begin{verbatim}
{request, From, Ref, Msg} ->
    From ! {proposal, ?, ?},
    Queue2 = insert(?, ?, ?, ?),
    Next2 = increment(?),
    server(Master, Next2, Nodes, Cast, Queue2, Jitter);
\end{verbatim}

We increment the value for the next sequece number, what ever happens
we must not propose the same or lower sequence number than the ones we
have proposed already.

\subsection{receiving a proposal}

A proposal is sent as a reply to a request that we have sent
earlier. The proposal contains the message reference and the proposed
sequence number. 

If the proposal is the last proposal that we are waiting for we have
also found an agreed sequence number. We implement this by calling the
function {\tt proposal/3} that will update the set and either return
{\tt\{agreed, Seq, Cast2\}} if an agreement was found or simply the
updated list.

\begin{verbatim}
{proposal, Ref, Proposal} ->
    case proposal(?, ?, ?) of
        {agreed, Seq, Cast2} ->
            agree(?, ?, ?),
            server(Master, Next, Nodes, Cast2, Queue, Jitter); 
        Cast2 ->
            server(Master, Next, Nodes, Cast2, Queue, Jitter)
    end;
\end{verbatim}

If we have an agreement this should be sent to all nodes in the
network. This is handled by the {\tt agree/3} procedure.

\subsection{agree at last}

An agree message contains the agreed sequence number of a particular
message. The message that is in the queue must be updated and possibly
moved back in the queue (the agreed number could be higher than the
proposed number).  This is handled by the function {\tt update/3}.


We also need to increment our next 

\begin{verbatim}
{agreed, Ref, Seq} ->
    Updated = update(?, ?, ?),
    {Agreed, Queue2} = agreed(?),
    deliver(?, ?),
    Next2 = increment(?, ?),
    server(Master, Next, Nodes, Cast, Queue2, Jitter);
\end{verbatim}

If the agreed sequence number is greater than the one that we
currently have we need to increment our value; we must make sure that
we never propose a value that could be lower than an agreed value.

If the first message in the queue now has an agreed sequence number it
could be delivered. The function {\tt agreed/2} will remove the
messages that can be delivered and return them in a list. These
messages can then be delivered using the {\tt deliver/2} procedure.


\section{experiments}

Try running the tests using the total order multicaster. Does it keep
workers synchronized? We have a lot of messages in the system, how
many messages can we multicast per second and how does this depend on
the number of workers? 

Build a larger distributed network, how large can it be before we are
down on our knees?

\end{document}
