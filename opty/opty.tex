\documentclass[a4paper, 11pt]{article}

\usepackage{ifthen}
\usepackage[latin1]{inputenc}


\newcommand{\nnsection}[1]{
\section*{#1}
\addcontentsline{toc}{section}{#1}
}

\begin{document}

\begin{center}
\vspace{20pt}
\textbf{\large Opty: optimistic concurrency control}\\
\vspace{10pt}
\textbf{Johan Montelius}\\
\vspace{10pt}
\today{}
\end{center}

\nnsection{Introduction}

In this session you will implement a transaction server using
optimistic concurrency control. You will also learn how to implement a
updatable data structure in Erlang that can be accessed by concurrent,
possibly distributed, processes. Before you start you should know how
optimistic concurrency control with backwards validation works.

\section{The architecture}

The architecture consist of a server having access to a store and a
validator. The store consist of a set of entries, each holding a value
and a unique reference. The reference is updated in each write
operation so we can validate that nothing has happened with the entry
since we last read its value.

A client starts a transaction by creating a transaction handler that
is given access to the store and the validator process. The
transaction server is thus not involved in the transaction; it
is simply a process from which we can get access to the store and the
validator process.

\subsection{the handler}

The transaction handler will handle read and write requests from the
client and will also to close the transaction. In each read operation
the handler keeps track of the unique reference of the entry that is
read. It also keeps the write operation in a local store so that the
real store is not modified until we want to close the transaction. 

When the transaction is to be closed the handler send the read and
write sets to the validator.

\subsection{the validator}

The validation process will be able to tell if a value of an entry
has been changes since the transaction read the entry. If non of the
entries have changed the transacting can commit and the associated
write operations are performed. 

Since there is only one validator in the system the validator is the
only process that actually writes anything to the store.

\section{The Implementation}

Since data structures in Erlang are not mutable (we can not change
their values), we need to do a trick. The store is represented by a
tuple of process identifiers. Each process is an entry and
we can change its value by sending it write messages. 

A transaction handler is given a copy of the tuple when created but
the processes that represent the entries are of course not copied. The
store can thus be shared by several transaction handlers, all who can
send read messages to the entries.

The validator will not need access to the whole store since it will be
given the read and write sets from the transaction handlers. These sets
will contain the process identifiers of the relevant entries.

\subsection{an entry}

A process that implements an entry should only have a value and a
unique reference as its state. We will below call this reference
``time stamp'' but it has nothing to do with time, it's simply a unique
reference. The reference will be given to a reader of the value so
that the validator later can determine if the value has been
changed. We could do without the reference but it has its advantages.

\begin{verbatim}
-module(entry).

-export([new/1]).

new(Value) ->
    spawn_link(fun() -> init(Value) end).

init(Value) ->
    entry(Value, make_ref()).

\end{verbatim}

\noindent Note that we are using the primitive {\tt
  spawn\_link/1}. This is to ensure that if the creator of the entry
dies, then the entry should also die.

Three messages should be handled by an entry:

\begin{itemize}

\item {\tt \{read, Ref, Handler\}}: a read request from a handler tagged
  with a reference. We will return a message tagged with the reference
  so that the handler can identify the correct message. The reply will
  contain the process identifier of the entry, the value and the
  current time stamp.

\item {\tt \{write, Value\}}: changes the current value of the entry,
  no reply needed. The time stamp of the entry will be updated.

\item {\tt \{check, Ref, Read, Handler\}}: check if the time stamp of
  the entry has changed since we read the value (at time {\tt
    Read}). A reply, tagged with the reference, will tell the handler
  if the time stamp is still the same or if it has to abort.

\item {\tt stop}: terminate.

\end{itemize}

We here talk about {\em the handler} and not {\em the client}, this
will be clear later. Why do we want the read message to include the
process identifier? Not quite clear at the moment but you will see
that it makes it easy to write an asynchronous transaction handler.

\begin{verbatim}
entry(Value, Time) ->
    receive
        {read, Ref, Handler} ->
            Handler ! {Ref, self(), Value, Time},
            entry(Value, Time);
        {check, Ref, Read, Handler} ->
            if 
                Read == Time ->
                    Handler ! {Ref, ok};
                true ->
                    Handler ! {Ref, abort}
            end,
            entry(Value, Time);
        {write, New} ->
            entry(New, make_ref());
        stop ->
            ok
    end.
\end{verbatim}

\subsection{the store}

We will hide the representation of the store and provide only an API
to create a new store and to look-up an entry in the store. Creating a
tuple is done simply by first creating a list of all process
identifiers and then turning it into a tuple.
\begin{verbatim}
-module(store).

-export([new/1, stop/1, lookup/2]).

new(N) ->
    list_to_tuple(entries(N, [])).

stop(Store) ->
    lists:map(fun(E) -> E ! stop end, tuple_to_list(Store)).

lookup(I, Store) ->
    element(I, Store). % this is a builtin function

entries(N, Sofar) ->
    if 
        N == 0 ->
            Sofar;
        true ->
            Entry = entry:new(0),
            entries(N-1,[Entry|Sofar])
    end.
\end{verbatim}

\subsection{transaction handler}

A client should never access the store directly. It will perform all
operations through a transaction handler. A transaction handler is
created for a specific client and holds the store and the
process identifier of the validator process. 

We will implement the handler so that a client can make asynchronous
reads to the store. If latencies are high there is no point in waiting
for one read operation to complete before initiating a second
operation.

The task of the transaction handler is to record all read operations
(and at what time these took place) and make write operations only
visible in a local store. In order to achieve this the handler will
keep two sets: the read set ({\tt Reads}) and the write set ({\tt
  Writes}). The read set is a list of tuples {\tt \{Entry, Time\}} and
the write set is a list of tuples {\tt \{N, Entry, Value\}}. When it
is time to commit, the handler sends the read and write sets to the
validator.

When the handler is created it is also linked to its creator. This
means that if one dies both die. This sounds hard but it will be 
explained once we implement the server.

\begin{verbatim}
-module(handler).

-export([start/3]).

start(Client, Validator, Store) ->
    spawn_link(fun() -> init(Client, Validator, Store) end).

init(Client, Validator, Store) ->
    handler(Client, Validator, Store, [], []).
\end{verbatim}

The message interface to the handler is as follows:

\begin{itemize}

\item {\tt \{read, Ref, N\}}: a read request from the client
  containing a reference that we should use in the reply message. The
  integer {\tt N} is the index of the entry in the store. The handler
  should first look through the write set to see if entry
  {\tt N} has been written. If no matching operation is found a
  message is sent to the nth entry process in the store. This entry
  will reply to the handler since we need to record the read time. 

\item {\tt \{Ref, Entry, Value, Time\}}: a reply from an entry that
  should be forwarded to the Client. The entry and time is saved in
  the read set of the handler. The reply to the client is {\tt \{Ref, Value\}}.

\item {\tt \{write, N, Value\}}: a write message from the client.  The
  integer {\tt N} is the index of the entry in the store and {\tt
    Value}, the new value. The entry with index N and the value is
  saved in the write set of the handler.

\item {\tt \{commit, Ref\}}: a commit message from the client. This is
  the time to contact the validator and see if there are any conflicts
  in our read set. If not, the validator will perform the
  write operations in the write set and reply directly to the client.
\end{itemize}

\noindent Here is some skeleton code for the handler.

\begin{verbatim}
handler(Client, Validator, Store, Reads, Writes) ->         
    receive
        {read, Ref, N} ->
            case lists:keysearch(N, 1, Writes) of
                {value, {N, _, Value}} ->
                       :
                    handler(Client, Validator, Store, Reads, Writes);
                false ->
                       :
                       :
                    handler(Client, Validator, Store, Reads, Writes)
            end;

        {Ref, Entry, Value, Time} ->
               :
            handler(Client, Validator, Store, [...|Reads], Writes);

        {write, N, Value} ->
            Added = [{N, ..., ...}|...],
            handler(Client, Validator, Store, Reads, Added);

        {commit, Ref} ->
            Validator ! {validate, Ref, Reads, Writes, Client};

        abort ->
            ok
    end.
\end{verbatim}


\subsection{validation}

The validation handler is responsible of doing the final validation of
transactions. The task is made quite easy since only one transaction
is validated at a time. There are no concurrent operations that could
possibly conflict with the validation process. 

When we start the validator we also link it to the processes that
creates it. This is to ensure that we don't have any zombie processes.

\begin{verbatim}
-module(validator).

-export([start/0]).

start() ->
    spawn_link(fun() -> init() end).

init()->
    validator().
\end{verbatim}

The validator receives a request from a client containing everything
that is needed both to validate that the transaction is allowed and to
perform the write operations that will be a result of the
transaction. The request contains:

\begin{itemize}

\item Ref: a unique reference to tag the reply message.

\item Reads: a list of read operations that have been performed. The
  validator must ensure that the entries of the read operations have
  not been changed.

\item Writes: the pending write operations that, if the transaction is
  valid, should be applied to the store.

\item Client: the process identifier of the client to whom we should
  return the reply.
\end{itemize}

Validation is thus simply checking if read operations are still valid
and if so update the store with the pending write operations.

\begin{verbatim}
validator() ->
    receive
        {validate, Ref, Reads, Writes, Client} ->
            case  validate(Reads) of
                ok ->
                    update(Writes),
                    Client ! {Ref, ok};
                abort ->
                    Client ! {Ref, abort}
            end,
            validator();
         _Old ->
            validator()
    end.
\end{verbatim}

Since a read operation is represented with a tuple {\tt \{Entry,
  Time\}} the validator need only send a {\tt check} message to the
entry and make sure that the current time-stamp of the entry is the
same. 

\begin{verbatim}
validate(Reads) ->
    {N, Tag} = send_checks(Reads),
    check_reads(N, tag).
\end{verbatim}

For better performance the validator can first send check messages to
all entries and then collect the replies. As soon as one entry replies
with an {\tt abort} message, we're done. Note however, that we must be
careful so that we are not seeing replies that pertain to a previous
validation. When we send our check request we therefore tag them with
a unique reference so that we know that we're counting the right replies.

\begin{verbatim}
send_checks(Reads) ->
    Tag = make_ref(),
    Self = self(),
    N = length(Reads),
    lists:map(fun({Entry, Time}) -> 
                      Entry ! {check, Tag, Time, Self}
              end, 
              Reads),
    {N, Tag}.
\end{verbatim}

Collecting the replies is a simple task and we only have to be
careful so that we don't collect an old reply.

\begin{verbatim}
check_reads(N, Tag) ->
    if 
        N == 0 ->
            ok;
        true ->
            receive
                {Tag, ok} ->
                    check_reads(N-1, Tag);
                {Tag, abort} ->
                    abort
            end
    end.
\end{verbatim}

Old messages that are still in the queue must be removed somehow, this
is why, and this is very important, the main loop of the validator
includes a catch all clause. 

\subsection{the server}

We now have all the pieces to build the transaction server. The server
will construct the store and one validator
process. Clients can then open a transaction and each transaction will
be given a new handler that is dedicated to the task.

\begin{verbatim}
-module(server).

-export([start/1, open/1, stop/1]).

start(N) ->
    spawn(fun() -> init(N) end).

init(N) ->
    Store = store:new(N),
    Validator = validator:start(),
    server(Validator, Store).
\end{verbatim}

The server could simply wait for requests from clients and spawn a new
transaction handler. There is however a trap here that we don't want
to fall into. If the server creates the transaction handler we must
let the handler be independent from the server (not linked). If the
handler dies we don't want the server to die. On the other hand, if
the client dies we do want the handler to die. The solution is to let
the process of the client create the handler.

\begin{verbatim}
open(Server) ->
    Server ! {open, self()},
    receive
        {transaction, Validator, Store} ->
            handler:start(self(), Validator, Store)
    end.
\end{verbatim}

\noindent This will also have implications on where the transaction
handler is running. Something that we might discuss further when we
are done with the server.

The server now becomes almost trivial.

\begin{verbatim}
server(Validator, Store) ->
    receive 
        {open, Client} ->
             :
            server(Validator, Store);
        stop ->
             store:stop(Store)
    end.
\end{verbatim}

You're done, you have implemented a transaction server using
optimistic concurrency control.

\section{Performance}



Does it perform? How many transaction can we do per second? What are
the limitations on the number of concurrent transactions and the
success rate? This does of course depend on the size of the store, how
many write operations each transaction does and how much delay we have
in between the read instructions of the transaction and the final commit
instruction.

Some Erlang related questions can be fair to raise. Is it realistic
with the store implementation that we have? Regardless of the
transaction handler, how fast can we operate on the store? What
happens if we run this in a distributed Erlang network, what is
actually copied when a transaction handler is started? Where is the
handler running? Pros and cons of this implementation strategy?


\end{document}
