\documentclass[a4paper,11pt]{article}

\usepackage{ifthen}
\usepackage[latin1]{inputenc}


\newcommand{\nnsection}[1]{
\section*{#1}
\addcontentsline{toc}{section}{#1}
}

\begin{document}

\begin{center}
\vspace{20pt}
\textbf{\large Paxy: the paxos protocol}\\
\vspace{10pt}
\textbf{Johan Montelius}\\
\vspace{10pt}
\today{}
\end{center}

\nnsection{Introduction}


We will implement a sequentially consistent highly available
server. 


\nnsection{The Architecture}


\nnsubsection{the failue model}

We will divide our implementation into nodes. Processes within a node
communicate using reliable channels and have access to perfect failure
detectors. Communication between nodes is unreliable but provide fifo
ordering of messages. Failure detection inbetween nodes are unreliable
and wil not be used.

A node behaves correctly until it crashes. When a node crashes it
enters a non-operational state and sease to communicate with other
nodes. The node will eventually recover and again enter operational
mode, it the behaves correct until it crashes.

The system should be availabel as long as at most one node is
non-operational. If more than one node is non-operational the system
coudl be unavailable but should still have a sequential consistent
behaviour.

The failure model is follows: at most one node can crash at any
given moment. A node that has crashed can be restarted but no other
node will fail untill the node is fully operational.

\nnsubsection{nodes a processes}

The architecture consist of the main node; the primary, the slave and
the arbitrator. The primary and slave nodes have three processes each
proposer, acceptor and server, the arbitrator only has the acceptor
process and is only needed in case the primary or slave node has
crahed.

The proposers and acceptors implement the multi-paxos protocol and the
server processes serve as the learners. The servers will respond
directly to the clint applications.

A client node will use a process that handles the communication with
the replicated server. The client process provides an assynchrounous
message interface; the application can send messages to it that it
will forward to the replicated server. If the 


\nnsubsection{the client process}

The client process has a two coordinators that it will be able to
communicate with; one is the master and one is the slave. It will have
apreference for communicating with the master but if it suspects that
the master is unavailable it will resend messages to the slave.

The client process will resend messages until it receives an
acknowledgment from either of the proposers. Since all clients will
favour the primary the primary will maintain a role as the sole
proposer. When it is slow in responding clients will shift to the
slave proposer but will return to the primary proposer as soon as
possible. This mean that the primary and slave proposers might both be
activated if the connection to the primary is faulty. 

\nnsubsection{proposers and acceptors}

The proposers and acceptors implement the multi-paxos protocol.

\subsection{}
  
\nnsection{Discussion}

\nnsubsection{at-least-once message delivery}

Since the client process will resend messages to the slave proposer in
case the primary is slow in responding we will have a at-least-once
semantics regarding message delivery from the client application
towrds the replicated server. It is left for the application layer to
resole this and possibly impement a exactly-oce semantics. This 

\end{document}
