\documentclass[a4paper, 11pt]{article}

\usepackage{ifthen}
\usepackage[latin1]{inputenc}

\newcommand{\nnsection}[1]{
\section*{#1}
\addcontentsline{toc}{section}{#1}
}

\begin{document}

\begin{center}
\vspace{20pt}
\textbf{\large Timey: time based concurrency control}\\
\vspace{10pt}
\textbf{Johan Montelius}\\
\vspace{10pt}
\today{}
\end{center}

\nnsection{Introduction}

In this session you will implement a transaction server using
time based concurrency control. You will also learn how to implement a
updatable data structure in Erlang that can be accessed by concurrent,
possibly distributed, processes. Before you start you should know how
time based concurrency control.

\section{The architecture}

The architecture consist of a server having access to a store. The
store consist of a set of entries, each holding a committed value and a
list of suspended read and write operations. An entry is responsible
for the concurrency control and will, as we will see, have quite complex
state transitions. 

A client starts a transaction by requesting a transaction handler from
the server. The transaction handler is given a unique time stamp when
created and will keep track of all write operations performed during
the transaction. When the client request the transaction to be committed
the handler makes sure that all write operation are also committed.

\subsection{an entry}

An entry is, as you will see, the most complex part. We will describe
it gradually, adding requirements as we realize the important role it
serves. To start, the entry has four values:

\begin{itemize}
\item {\bf committed}: the value that has been committed.
\item {\bf written}: the time this value was committed.
\item {\bf read}: the last time (highest time stamp) this value was read.
\item {\bf susp}: an ordered list of suspended read and write operations.
\end{itemize}

We will have four messages that the entry must be able to handle. All
messages are from a transaction handler. This is a short description
of the protocol.

\begin{itemize}
\item {\bf Read}: a read request containing a time stamp. The handler
  should be given a committed {\bf value} when available. If the request is
  too late the reply is a {\bf abort} message.
\item {\bf Write}: a write request containing a time stamp The handler
  is sent a {\tt ok} or {\tt abort} message.
\item {\bf Commit}: a commit request specifying a earlier write
  request. The write request, if not previously aborted, is promoted to
  a committed value. No reply is sent back to the handler.
\item {\bf Abort}: am abort request specifying a earlier write
  request. The write request, if not previously aborted, is aborted. No
  reply is sent back to the handler.
\end{itemize}

In the implementation we will of course see how to connect requests to
replies and how to keep track of which commit and abort request belong
to which write operations but we will go through the general idea
first.

\subsection{a read request}

We will start by looking at the read request and solve the easy
case. When a read request is received by an entry it will check if the
time of the read request is later than the currently committed
value. If not it means that the read operation came too late and an
abort reply is sent to the handler.

If the request was not to late we must check the suspension list. It
could be that there is a suspended write operation, not yet committed,
that has a time stamp value that is earlier than the read
operation. If this is the case the read operation must be inserted
behind the write operation in the suspension lists. A special case
occur if we find a write operation with the same time stamp. This would
mean that a client first writes a value to an entry and then reads the
same entry. The client should of course read its writes so we must
return the written value even if this is only a tentative value. 

If the suspension list is empty or only contains suspended write
operations that have time stamps later than the read operation we can
safely read the current value. We then send a reply to the handler
and also update the read time of the committed value.

\subsection{a write request}

A write request is almost simpler. We must first examine the read
time of the current value. If the read time is later than the time of
the write operation there is not much to do, we have to
abort. Assuming that this is not the case; examine the write time of
the current value. If the write time is later than the time of the
write operation we could simply forget that this operation was ever
done. If a value has been written (and committed) at time $14$ but no
one has read the value since time $10$, then what difference can a
write operation at time $12$ do? We can simply report to the handler
that the operation has been handled.

If the write operation is later than the current write time we must do
some bookkeeping. We can of course not replace the current value
straight away, instead we must suspend the operation as a tentative
write operation. The write operation is thus inserted into the list of
suspended operations. The handler is of course informed that the
operation has been correctly handled.

\subsection{a commit request}

Now for the tricky part; what should we do when a write operation
later is committed? If a write operation to be committed is the first
operation in the suspension list we should remove it from the list and
promote the tentative value to current value of the entry. 

If the write operation was followed by a sequence of suspended read
operations these can now be scheduled and the committed value is sent
back as a reply. This will of course also update the read time of the
committed value.

Now assume that the first operation in the list is another write
operation and the operation to be committed is somewhere down the
list. We should promote the value to become a committed value but the
operation is still in the list of suspended operation. The list would
then contain tentative write, committed writes and suspended read
operations. 

Having committed values in the suspension list changes how a write
operation is committed. If the write operation to be committed is the
first element in the list and all following read operations have been
removed, it could be that the next element is a committed value. This
operation must now be promoted to the current value of the entry; a
process that might trigger even more read operations. 

\subsection{the handler}

The role of the handler becomes quite simple since most of the
concurrency control is handled by the entries. The handler need only
keep track of a time stamp and attache this to all read and write
operations that it sends to the entries. Read request and read replies
are thus very simple to handle. Write requests have to be stored so
that the handler can schedule commit requests when the client wants to
commit. The handler should also keep track of which write request that
have been successfully handled. If any write request has to abort the
whole transaction must abort.

It's important to understand that the commit request that the
transaction is sending to all entries that have been involved in write
requests does not return any message. The handler knows that the write
operations have succeeded if they all return an ok message when
issued. The ok message means that the request is added as an tentative
write operation and nothing can make this operation fail. The commit
request will promote the value to a committed value but nothing can
prevent this from happening.

\subsection{a client}

The client is happily unaware of what is going on. It will contact
the server and request a new transaction handler. It then
communicates with the transaction handler and sends read and write
request. When it is done it will send a commit request and wait for a
result.

The client does not receive a message for each write request. The
handler will check that all write requests are handled properly and
only report on the transaction as a whole.


\section{The Implementation}

Let's start with the most complicated part, the implementation of the
entry process.

\subsection{the entry}

The process will have a state consisting of: the current
value, its write time, last time it was read and the list of suspended
operations. Let's call the module {\tt entry} and prepare to export
the procedures we need and also some for testing that our implementation works.

\begin{verbatim}
-module(entry).

-export([new/1]).
-export([suspend_read/4, suspend_write/4, suspend_commit/2, abort_write/2]).

new(Value) ->
    spawn_link(fun() -> init(Value) end).

init(Value) ->
    entry(Value, 0, 0, []).

entry(Value, Written, Read, Susp) ->
    receive 
      {read, Ref, Time, Handler}  ->
           :

      {write, Ref, Time, Tentative, Handler} ->
           :

      {commit, Ref} ->
           :

      {abort, Ref} ->
           :

      stop ->
           ok
    end.
\end{verbatim}

The entry procedure will go into a loop and wait for request from any
transaction handler. Request from the transaction handler will be
tagged with unique references so that the handler can properly
identify the reply from the entry. The handler will operate
asynchronously and might have several outstanding requests.

\subsubsection{a read request}

The first message to handle is a read request. The request contains a
unique reference, the time stamp of the request and the process
identifier of the handler so that we can reply. We will make use of a
supporting function called {\tt suspend\_read/4}. This function will,
if necessary, insert a read operation in the list of suspended
operation and then return the updated list. If however, there is no
tentative write operation in the list with lower time stamp then the
current value of the entry is read.

\begin{verbatim}
{read, Ref, Time, Handler}  ->
    if
        (Time >= Written) ->
            case suspend_read(Time, Ref, Handler, Susp) of
                no ->
                    Handler ! {reply, Ref, {ok, Value}},
                    entry(Value, Written, Time, Susp);
                Susp2 ->
                    entry(Value, Written, Read, Susp2)
            end;
        true ->
            Handler ! {reply, Ref, abort},
            entry(Value, Written, Read, Susp)
    end;
\end{verbatim}

\subsubsection{a write request}

The write request is equally simple; the request consist of a unique
reference, a time stamp, the tentative value and a process identifier
of the handler. We first check that the request is not too late
i.e. that a read operation has read the value at a ``later'' time. 

\begin{verbatim}        
{write, Ref, Time, Tentative, Handler} ->
    if 
        Time >= Read ->
            if 
                (Time >= Written)  ->
                    Susp2 = suspend_write(Time, Ref, Tentative, Susp),
                    Handler ! {ok, Ref},
                    entry(Value, Written, Read, Susp2);
                true ->
                    Handler ! {ok, Ref},
                    entry(Value, Written, Read, Susp)
            end;
        true ->
            Handler ! {abort, Ref},
            entry(Value, Time, Read, Susp)
    end;
\end{verbatim}

We make use of a supporting function called {\tt suspend\_write/4}
that will insert the write operation at the right position in the
suspension list. As opposed to the insertion of a read operation, the
write operation will always be inserted the list. The suspended
write request contains the unique reference so that we can identify it
later when the handler wants to commit.

When we report back to the handler we of course tag the reply with the
unique reference so that the handler can match the request to the
reply.

Note that the write request will always result in a reply directly. We
know if the operation was too late or arrived in time. For the read
request we only replied directly if it was possible to read the
committed value. If not, the reply comes when a commit request triggers
the suspended read operation to be scheduled.

\subsubsection{a commit request}

The commit request contains a reference that identifies a previous
issued write request. As we discussed there are two cases that we must
take care of; either it is the first write operation in the list or it
is somewhere down the list (or not in the list at all).

A suspended write operation is represented with a tuple containing the
reference so it is easy to identify it. If it is the first write
operation that should be promoted we should of course also schedule
all suspended read operations (and possibly also suspended commit
operations). We could have made use of a function also in this
case but it turns out to be more practical to call a procedure {\tt
  commit/4} that when it has done its job will call the {\tt entry/4}
procedure with the right state.

\begin{verbatim}
{commit, Ref} ->
    case Susp of
        [{Time, {write, Ref, Tentative}}|Susp2] ->
            commit(Tentative, Time, Read, Susp2);
        _ ->
            Susp2 = suspend_commit(Ref, Susp),
            entry(Value, Written, Read, Susp2)
    end;
\end{verbatim}

If it is not the most recent write operation that should be committed
we promote a suspended write operation somewhere down the list. For
this we use the function {\tt suspend\_commit/2} that will simply
transform a write operation to a commit operation.

\subsubsection{a abort request}

The abort request is trivial. The request contains a reference that
identifies a write request that should be removed. 

\begin{verbatim}
{abort, Ref} ->
    Susp2 = abort_write(Ref, Susp),
    entry(Value, Written, Read, Susp2);     
\end{verbatim}


\subsubsection{committing}

The procedure {\tt commit/4} is called when we have received a commit
request and the first suspended write operation in the list was
identified as the operation to commit. We call the procedure with the
committed value, the time this was written, the last time the entry was
read and the list of suspended operations.

Before we continue we must schedule and remove and suspended read
operations. If there are no operations in the list we can of course
call the {\tt entry/4} procedure directly with the parameters we have
at hand. If the first operation is a read operation we send a reply
with the current value to the handler that issued the request. When we
continue down the list we update our read time stamp.

If the first operation is a committed value it means that this value is
now the current value so we should continue to remove read operations
but now with the committed value as the current value. We also update
the write time stamp.

\begin{verbatim}
commit(Value, Written, Read, Susp) ->
    case Susp of
        [] ->
            entry(Value, Written, Read, []);
        [{R, {read, Rf, Handler}}|Susp2] ->
            Handler ! {reply, Rf, {ok, Value}},
            commit(Value, Written, R, Susp2);
        [{W, {committed, Committed}}|Susp2] ->
            commit(Committed, W, Read, Susp2);      
        _ ->
            entry(Value, Written, Read, Susp)
    end.
\end{verbatim}

If the first operation is not a read nor a commit we have done our job
and continue by calling the {\tt entry/4} procedure.

\subsubsection{supporting functions}

The only thing that remains is writing the supporting functions. The
list of suspended operations is represented using the following tuples:

\begin{itemize} 
\item {\tt \{Time, \{read, Reference, Handler\}\}}: a read operation
  at {\tt Time} with a unique {\tt Reference} issued by a {\tt Handler}.

\item {\tt \{Time, \{write, Reference, Tentative\}\}}: a write operation
  at {\tt Time} with a unique {\tt Reference} and a {\tt Tentative} value.

{\tt \{Time, \{committed, Committed\}\}}: a committed write operation
  at {\tt Time} with a {\tt Committed} value.
\end{itemize} 

The description of the supporting functions is as follows:

\begin{itemize} 

\item {\tt suspend\_read(Time, Ref, Handler, Susp)}: if there is a
  write operation in {\tt Susp} that is earlier than {\tt Time} then
  insert the read operation {\tt \{read, Time, Handler\}} in the
  list. If there is a write operation at {\tt Time} then reply to the
  {\tt Handler} with the value of the write operation and return {\tt
    Susp}. If neither holds true, return {\tt no}.

\item {\tt suspend\_write(Time, Ref, Tentative, Susp)}: insert the 
write operation {\tt \{Time, \{write, Ref, Tentative\}\}} in {\tt Susp}.

\item {\tt suspend\_commit(Ref, Susp)}: if there exist a write
  operation {\tt \{Time, \{write, Ref, \_\}\}} in {\tt Susp} then
  replace it with {\tt \{Time, \{committed Tentative\}\}}.

\item {\tt abort\_write(Ref, Susp)}: if there exist a write
  operation {\tt \{Time, \{write, Ref, \_\}\}} in {\tt Susp} then
  remove it. 
\end{itemize}


The implementation is left as an exercise.

\subsection{the store}

The store is simple a tuple of process identifiers of entries to
which the handler can send read and write requests. We create it by
first constructing a list of all the process identifiers and then
turning this list into a tuple. The handler, that will access the
store, need only use the interface {\tt lookup/2} to get the right
process identifier.

\begin{verbatim}
-module(store).

-export([new/1, stop/1, lookup/2]).

new(N) ->
    list_to_tuple(entries(N, [])).

stop(Store) ->
    lists:foreach(fun(E)-> E ! stop end, tuple_to_list(Store)).

lookup(I, Store) ->
    element(I, Store). % this is built-in

entries(N, Sofar) ->
    if 
        N == 0 ->
            Sofar;
        true ->
            Entry = entry:new(0),
            entries(N-1,[Entry|Sofar])
    end.
\end{verbatim}

\subsection{the handler}

The transaction handler will be started by the client and is given a
unique identifier, a unique time stamp and a store. The identifier is
used to report the result of the transaction back to the client. The
identifier allows a client to have more than one transaction open. 

The job of the handler is mainly to accept request from the client and
to forward them to the right entry. It will also keep track of all
write operations and thus keep two extra parameters: the set of
unconfirmed write operations and the set of confirmed write
operations.


\begin{verbatim}
-module(handler).

-export([start/4]).

start(Client, Id, Time, Store) ->
    spawn(fun() -> init(Client, Id, Time, Store) end).

init(Client,  Id, Time, Store) ->
    handler(Client, Id, Time, Store, [], []).

handler(Client, Id, Time, Store, Unconf, Conf) ->           
    receive
        :
        :

        Error ->
            io:format("strange message ~w~n", [Error])
    end.
\end{verbatim}

Read operations are forwarded directly to the right entry with the
handler as the sender. The replies are returned to the client and one
could ask if it is really necessary to send these first to the handler
and then to the client; why not send them to the client directly? We
will discuss these questions later but for now it's fine to do the
simple things.

Note that the handler does not care if a read operation fails. It
simply passes the result back to the client.

\begin{verbatim}
{read, Ref, N} ->
    Entry = store:lookup(N, Store),
    Entry ! {read, Ref, Time, self()},
    handler(Client, Id, Time, Store,  Unconf, Conf);

{reply, Ref, {ok, Value}} ->
    Client ! {Ref, {ok, Value}},
    handler(Client, Id, Time, Store,  Unconf, Conf);        

{reply, Ref, abort} ->
    Client ! {Ref, abort},
    handler(Client, Id, Time, Store,  Unconf, Conf);        
\end{verbatim}

The write operations must be handler with a bit more care. When the
client wish to commit we must know if all write operations could be
committed. We will thus save the reference of a write operation in the
list of unconfirmed operations. When a {\tt ok} message arrives the
operation is removed from the set of unconfirmed to the set of
confirmed operations. Since we later have to send a message to the same entry we 

If an {\tt abort} message is returned then the whole transaction must
abort and the client is informed by an abort message tagged with the
identifier of the transaction.

\begin{verbatim}
{write, N, Value} ->
    Ref = make_ref(),
    Entry = store:lookup(N, Store),
    Entry ! {write, Ref, Time, Value, self()},
    handler(Client, Id, Time, Store, [{Ref, Entry}|Unconf], Conf);

{ok, Ref} ->
    {value, Op} = lists:keysearch(Ref, 1, Unconf),
    Unconf2 = lists:keydelete(Ref, 1, Unconf),
    handler(Client, Id, Time, Store, Unconf2, [Op|Conf]);           

{abort, _Ref} ->
    Client ! {abort, Id};
\end{verbatim}

When the handler is asked to commit the transaction it will send
commit messages to all write operations issued during the
transaction. It will then make sure that there are no unconfirmed
transactions left. It does not have to check if the commit operations
succeeded since this is guaranteed by protocol. 

\begin{verbatim}
commit ->
    lists:map(fun({Ref,Pid}) -> Pid ! {commit, Ref} end, Unconf),
    lists:map(fun({Ref,Pid}) -> Pid ! {commit, Ref} end, Conf),
    case commit(Unconf) of
        commit ->
            Client ! {committed, Id};
        abort ->
            Client ! {abort, Id}
    end;
\end{verbatim}

An abort message from the client will simply result in abort messages
to all outstanding write operations.

\begin{verbatim}
abort ->
    lists:map(fun({Ref,Pid}) -> Pid ! {abort, Ref} end, Unconf),
    lists:map(fun({Ref,Pid}) -> Pid ! {abort, Ref} end, Conf);
\end{verbatim}

To complete the handler we need one procedure that is waiting for the
last {\tt ok} messages. If all unconfirmed write operations have
confirmed that they will be able to commit the transaction s a whole
can commit .

\begin{verbatim}
commit([]) ->
    commit;
commit(Unconf) ->
    receive
        {abort, _Ref} ->
            abort;
        {ok, Ref} ->
            commit(lists:keydelete(Ref, 1, Unconf))
    end.
\end{verbatim}


\subsection{the server}

The server simply keeps track of the store and the current time
stamp. A client will call {\tt open/1} to get access to newly created
transaction handler with the correct time stamp, access to the store
and and a unique transaction identifier.

\begin{verbatim}
-module(server).

-export([start/1, open/1, stop/1]).

start(N) ->
    spawn(fun() -> init(N) end).

init(N) ->
    Store = store:new(N),
    server(1, Store).

open(Server) ->
    Server ! {open, self()},
    receive
        {transaction, Time, Store} ->
            Id = make_ref(),
            Handler = handler:start(self(), Id, Time, Store),
            {Id, Handler}
    end.

stop(Server) ->
    Server ! stop.
\end{verbatim}

We have a option here; should the transaction handler be started by
the server or by the client. In the code listed, the handler is started
by the client and thus will reside at the node of the client. This
means that the tuple that represents the store has to be copied,
possibly to another machine. If the handler was created by the server
the store would also be copied but then only between processes on the
same Erlang node. The approach chosen have some advantages in that once
we have started the handler it will not place any burden on the node of
the server. 

\begin{verbatim}
server(Time, Store) ->
    receive 
        {open, Client} ->
            Client ! {transaction, Time, Store},
            server(Time+1, Store);
        stop ->
            store:stop(Store)
    end.
\end{verbatim}

\section{Evaluation and discussion}

\subsection{performance}

How well does the transaction handler perform? We can run some simple
test with a couple of read and write operations to see how long time a
singe transaction takes. We can then extend the tests to have several
clients running on individual machines and run transaction concurrently
to see how many transaction we could handle per second.

The number of failed transactions does of course depend on the number
of read and write operations of each transaction and how large the
store is compared to the number of concurrent transactions. The smaller
the store the more likely it is that two transaction collide and one
of them turns up to late. 

One can also do experiment to see how long the lists of suspended
operations are. Is it worth the trouble to optimize the handling of
large lists or it is something else that we can do to increase
performance?

\subsection{read short-cut}

One question that we skipped over was how come the handler ignores if a
read operation had to abort. The reasoning is as follows. Assume that a
client issues two read operation and is informed that one of them
could not be handled but had to abort. The client has not seen an old
value nor a value that is not yet committed so the client can not
violate the consistency constraints by continue the transaction and
later commit. It is thus up to the handler to decide what to do.

This of course also means that the reply from an entry could be sent
directly to the client form the entry. Change the code and see if
there is any advantages in performance.


\subsection{advanced bookkeeping}

Assume that transactions are long lived; a client might open a
transaction and keep it open for minutes. Now if one transaction
commits a value at time $17$ but this value is still not the current
value of the entry (there is a tentative value at time $14$). Now what
will happen if we have a read request to the entry at time $18$? In
our current implementation this request will be suspended but we could
of course also send a reply back since we know what the value will be.

Design a scheme where read request can be replied to even if a
committed value is yet not the current value of entry. You have to be
careful; what would happen if we reply to a read request with time
stamp $22$ and then later receives a write request at time stamp $19$.


\end{document}
