\documentclass[a4paper,11pt]{article}


\usepackage{ifthen}
\usepackage[latin1]{inputenc}

\newcommand{\nnsection}[1]{
\section*{#1}
\addcontentsline{toc}{section}{#1}
}

\author{Johan Montelius}
\title{Donny: available key value store}

\begin{document}

\maketitle

\nnsection{Introduction}

In this tutorial we're going to look at the Dynamo key value store
implemented by Amazon. The Dynamo system is a complete distributed
hash table with replicated data to provide a fault tolerant
service. We're not going to implement a distributed hash table in this
exercises rather look at the thing that makes Dynamo so special.

Dynamo trades consistency for high availability. If things break down
we will not guarantee consistency but at least we will be able to
proceed. When things get back to normal we will detect the
inconsistency and try to resolve the problems. 

This tutorial is inspired by Dynamo but the result is so
far from Dynamo that we will call it Donny. We will capture and handle
the same problems but the setting is quite different.


\section{Time and value}

We will first implement some module that will be able to handle time
stamped values. We will build a key value store but we will associate
a set of values with each key. Each value is time stamped with a
vector clock. If we know which value is the last value then there is
only one value associated with the key but it could be that we have
two values where we don't know which is the last written value.

The vector clocks will only contain three indices so we can represent a
vector by a tuple with three integers {\tt \{A,B,C\}}. A module {\tt time}
will implement the following functions:

\begin{itemize}
\item {\tt create()} : returns a null time stamp
\item {\tt increment(Index, Time)} : increment the index of the time stamp vector by one
\item {\tt gt(T1, T2)} : evaluates to true if T1 is greater or equal to T2
\end{itemize}




\section{Donny}

A process, called {\tt donny}, that will
hold a key value store and be able to {\tt update} and {\tt lookup}
values of the store. To separate how the key values are actually
stored we first implement a module {\tt storage} that exports the functions:

\begin{itemize}
\item {\tt create()} : returns a empty store
\item {\tt update(Key, Value, Store)} : returns the updated store where {\tt Key} is associated to {\tt Value}.
\item {\tt lookup(Key, Value)} : returns the value associated to the {\tt Key}.
\end{itemize}


\begin{verbatim}
        {add, Key, Value, Ref, Client} ->
            Updated_store = storage:add(Key, Value, Store),
            Client ! {ok, Ref},
            loop(Updated_store);
\end{verbatim}

\begin{verbatim}
	{lookup, Key, Ref, Client} ->
	    io:format("key: ~w, ref: ~w, client: ~w~n", [Key, Ref, Client]),
	    case storage:lookup(Key, Store) of
		{value, {Key, Values}} ->
		    io:format("values: ~w  ~w~n", [Client, Values]),
		    Client ! {ok, Ref, Values};
		false ->
		    io:format("false: ~w~n", [Client]),
		    Client ! {false, Ref}
	    end,
	    io:format("lookup done~n", []),
	    loop(Store);
    end.
\end{verbatim}



\end{document}
