\documentclass[a4paper, 11pt]{article}

\usepackage{ifthen}
\usepackage[latin1]{inputenc}

\newcommand{\nnsection}[1]{
\section*{#1}
\addcontentsline{toc}{section}{#1}
}

\begin{document}

\begin{center}
\vspace{20pt}
\textbf{\large Routy: a small routing protocol}\\
\vspace{10pt}
\textbf{Johan Montelius}\\
\vspace{10pt}
\textbf{\today}
\end{center}

\nnsection{Introduction}

Your task is to implement a link-state routing protocol in Erlang. The
link-state protocol is used in for example OSPF, the most used routing
protocol for Internet routers. The aim of this exercise is that you
should be able to:

\begin{itemize}
\item describe the structure of a link-state routing protocol
\item describe how a consistent view is maintained
\item reflect on the problems related to network failures
\end{itemize}

We will implement routing processes with logical names such as {\tt
london}, {\tt berlin}, {\tt paris} etc. Routers can be connected to
each other with directional (one-way) links  and they can only communicate with
the routers that they are directly connected to.

The routing processes should be able to receive a message of the form
{\tt \{route, london, berlin, "Hello"\}} and determine that it is a
message from {\tt berlin} that should routed to {\tt london}. A
routing process should consult its routing table and determine which
{\em gateway} (a routing process that it has direct connection to) is
best suited to deliver the message. If a message arrives at it's
destination (the router called {\tt london}) it is printed on the
screen. Messages for which paths are not found are simply thrown away,
no control messages are sent back to the sender.

During the seminar you will be divided into groups representing
regions of the world (Europe, Asia, Africa etc). Each Erlang engine
that you run will have a name of a country in that region (Sweden, UK,
France ect). Assume that the Erlang shell named {\tt sweden}, is
running on a machine with the IP address 130.123.112.23. Routing
processes that we create will be registered ar {\tt r1}, {\tt r2}
etc. The adress of a routing process is thus for example:

\begin{verbatim}
      {r1, 'sweden@130.123.112.23'}
\end{verbatim}

The routing process will also have a logical name, for example {\tt
  stockolm}, but this is a name on a different level.

Your task before the seminar will be to have a router up and running. At the
seminar we will connect the routers together.

Before implementing the operations I advice you to study the {\tt
  lists} library and learn how {\tt keyfind/3}, {\tt keydelete/2},
{\tt map/2} and {\tt foldl/3} works. It will make your life easier
but, if you don't understand what {\tt foldl/3} does, then don't even
try to use it.

\section{The map}

Think of a good representation of a directional map where you should
easily be able to update the map and find nodes directly connected to
a given node. We could represent it as a list of entries where each
entry consist of a city with a list of directly connected cities. This
will give us a very quick way of updating the map, simply replace an
entry with a new entry. For our purposes this is fine, in other
situations one might want other operations to be efficient and
therefore need another representation.

In a module {\tt map}, implement and export the
following functions:

\begin{itemize}

\item{\bf new():} returns an empty map (a empty list)

\item{\bf update(Node, Links, Map):} updates the Map to reflect that
  Node has directional links to all nodes in the list Links. The old
  entry is removed.

\item{\bf reachable(Node, Map):} returns the list of nodes directly reachable from Node.

\item{\bf all\_nodes(Map):} returns a list of all nodes in the map,
  also the ones without outgoing links. So if {\tt berlin} is linked
  to {\tt london} but {\tt london} does not have any outgoing links
  (and thus no entry in the list), {\tt london} should still be in the
  returned list.
  
\end{itemize}

\noindent Before going further make sure that your implementation of map
works. In the tests below the map is represented as a list of entries
holding the node and the links. Try the following tests:

\begin{verbatim}
 > map:new().
[]

> map:update(berlin, [london, paris], []).
[{berlin,[london,paris]}]

 > map:reachable(berlin, [{berlin,[london,paris]}]).
[london,paris]

>map:reachable(london, [{berlin,[london,paris]}]).
[]

>map:all_nodes([{berlin,[london,paris]}]).
[paris,london,berlin]

>map:update(berlin, [madrid], [{berlin,[london,paris]}]).
[{berlin, [madrid]}]
\end{verbatim}

\noindent Note that the representation of the map should not be known
by the users of a map. A module using a map should only use the four
functions described above.

\section{Dijkstra}

The Dijkstra algorithm will compute a {\em routing table}. The table
is represented by a list with one entry per node where the entry
describes which gateway, city,  should be used to reach the node. The input to
the algorithm is: 
\begin{itemize}
\item a map 
\item a list of {\em gateways} to which we have direct access.
\end{itemize}

An example of a routing table is:

\begin{verbatim}
[{berlin,madrid},{rome,paris},{madrid,madrid},{paris,paris}]
\end{verbatim}

This table says that if we want to send something to {\tt berlin} we
should send it to {\tt madrid}. Note that we also include information
that in order to reach {\tt madrid} we should send the message to {\tt
  madrid}.

\noindent A router will know its own name, a set of gateways, a map
of the network and a hopefully not to old routing table.

The map will describes how all other nodes, including the gateways,
are connected. The map will not include the router itself. When we
build the router we should see that the map is updated quite
frequently. The routing table is however, only updated once in a
while (when we say so).

\subsection{a sorted list}

In the algorithm we will use a sorted list when we calculate a new
routing table. We will start by implementing operations on a sorted
list and then look at the algorithm itself.

Each entry in the list will hold the name of a node, the length of the
path to the node and the gateway that we should use to reach the
node. An entry showing that {\tt berlin} could be reached in 2 hops
using {\tt paris} as a gateway could look like follows:

\begin{verbatim}
  {berlin, 2, paris}
\end{verbatim}

The list is sorted based on the length of the path. We should be able
to update the list to give a node a new length and a new gateway but
when we do an update it is important that we update an existing entry
and that we actually have an entry in the list to update.

To implement the update procedure it could be an advantage to first
implement two procedures that will help us. 
In a module {\tt dijkstra} implement the
two procedures:

\begin{itemize}

\item {\bf entry(Node, Sorted):} returns the length of the shortest
path to the node or $0$ if the node is not found.

\item {\bf replace(Node, N, Gateway, Sorted):} replaces the entry for Node
in Sorted with a new entry having a new length N and Gateway. The
resulting list should of course be sorted.
\end{itemize}

Note that in {\tt replace/4} we require a entry for the node to be
present in the sorted list. Be careful and make sure that the
resulting list is sorted based on the new entry.

\noindent Now when we have these two procedures it is easier to
implement the update procedure.

\begin{itemize}
\item {\bf update(Node, N, Gateway, Sorted)}: update the list Sorted
given the information that Node can be reached in N hops using
Gateway. If no entry is found then no new entry is added. Only if we
have a better (shorter) path should we replace the existing entry.
\end{itemize}

\noindent The procedure is implemented simply by first calling the
{\tt entry/2} procedure to get the length of the existing path. If we
have a better (shorter) path then we use the {\tt replace/4}
procedure. Why did we make {\tt entry/2} return 0 if the node is not
found?

\begin{verbatim}
> dijkstra:update(london, 2, amsterdam, []).
[]

> dijkstra:update(london, 2, amsterdam, [{london, 2, paris}]).
[{london,2,paris}]

> dijkstra:update(london, 1, stockholm, 
      [{berlin, 2, paris}, {london, 3, paris}]).
[{london,1,stockholm}, {berlin, 2, paris}]
\end{verbatim}

\subsection{the iteration}

This is the heart of the algorithm. We will take a sorted list of
entries, a map and a table that is what we have constructed so far. We have three cases:

\begin{itemize}
\item If there are no more entries in the sorted list then we are done and the
given routing table is complete.  

\item If the first entry is a dummy entry with an infinite path to a
  city we know that the rest of the sorted list is also of infinite length and 
 the given routing table is complete.  

\item Otherwise, take the first entry in
the sorted list, find the nodes in the map reachable from this entry
and for each of these nodes update the Sorted list. The entry that
you took from the sorted list is added to the routing table. 
\end{itemize}

Iterate this until we have no more entries in the sorted list - the
table is then complete.

What is happening here? If the entry says that {\tt berlin} can be
reached in three hops by going through {\tt paris} and the map says
that berlin is directly linked to {\tt copenhagen}, then copenhagen is
reachable in four hops going through {\tt paris}. We might already have
a entry for {\tt copenhagen} using only three hops over {\tt
amsterdam} and then nothing is done, but if we have an entry with more
than four hops we will update the list.

If we have an entry for {\tt copenhagen} with less than three
hops, this entry has already been processes and removed from the
list. This explains why we do not want to add another entry for {\tt
copenhagen}.

Note, since our network is connected by directional links it could
actually be the case that some nodes in our map are not reachable at
all. If {\tt ulanbator} has a link to {\tt beijing} but there is no
link from {\tt beijing} to {\tt ulanbator} then the world will have
{\tt ulanbator} in the map. If all cites in the map are chosen to be
part of the original sorted list that we try to iterate over we will
in the end find an entry:

\begin{verbatim}
{ulanbator, inf, unknown}
\end{verbatim}

as the first element in the list. If we have this situation we can
conclude that the routing table we have is complete and contains all
reachable cities. 

\begin{itemize}
\item {\bf iterate(Sorted, Map, Table)}: construct a table given a sorted
list of nodes, a map and a table constructed so far.
\end{itemize}

The second case is to handle the situation when nodes in the map are 
not reachable. In order to capture this we take a closer look at
the first node in the sorted list. If we have a node with the length
set to infinity, {\tt inf}, then this node (nor any other node after
it since the list is sorted) cannot be reached and need not be part of
the final table.

This is a test of the iterate procedure:

\begin{verbatim}
> dijkstra:iterate([{paris, 0, paris}, {berlin, inf, unknown}], 
       [{paris, [berlin]}], []).
[{paris, paris},{berlin,paris}]
\end{verbatim}

Now in the same module implement the function {\tt table/2} that
should take a list of gateways and a map and produce a routing table
with one entry per node in the map. The table could be a list of
entries where each entry states which gateway to use to find the
shortest path to a node (if we have a path). Follow the outline below
and you will have your program running in no-time.

\begin{itemize}
\item {\bf table(Gateways, Map)}: construct a routing table given the
gateways and a map. 
\end{itemize}

List the nodes of the map and construct a initial sorted list. This
list should have dummy entries for all nodes with the length set to
infinity, {\tt inf}, and the gateway to {\tt unknown}. The entries of
the gateways should have length zero and gateway set to itself. Note
that {\tt inf} is greater than any integer (try). When you have
constructed this list you can call iterate with an empty table. This
is a test of the table procedure:

\begin{verbatim}
> dijkstra:table([paris, madrid], [{madrid,[berlin]}, {paris, [rome,madrid]}]).
[{berlin,madrid},{rome,paris},{madrid,madrid},{paris,paris}]
\end{verbatim}

To complete the dijkstra module we need one more procedures. 

\begin{itemize}
\item {\bf route(Node, Table)} search the routing table and return the
gateway suitable to route messages to a node. If a gateway is found we
should return {\tt \{ok, Gateway\}} otherwise we return {\tt notfound}.
\end{itemize}

The {\tt table/2} and {\tt route/2} are the only procedures that we
need to export. No one outside the module knows how the table is
represented so you can re-implement it and make it even more efficient.

\section{Interfaces}

A router will also need to keep track of a set of interfaces. A
interface is described by the symbolic name ({\tt london}), a {\em process
reference} and a {\em process identifier}. When you implement the router it
will be clear what a process reference is. Implement the following procedures:


\begin{itemize}
\item {\bf new()} return an empty set of interfaces.

\item {\bf add(Name, Ref, Pid, Intf)} add a new entry to the set and
return the new set of interfaces.

\item  {\bf remove(Name, Intf)} remove an entry given a name of an interface,
return a new set of interfaces.

\item  {\bf lookup(Name, Intf)} find the process identifier given a name,
return {\tt \{ok, Pid\}} if found otherwise {\tt notfound}.

\item  {\bf ref(Name, Intf)} find the reference given a name and return {\tt \{ok, Ref\}} or {\tt notfound}.

\item  {\bf name(Ref, Intf)} find the name of an entry given a reference and
return {\tt \{ok, Name\}} or {\tt notfound}.

\item  {\bf list(Intf)} return a list with all names.

\item  {\bf broadcast(Message, Intf)} send the message to all interface processes.
\end{itemize}

\noindent It should be quite straight forward to implement this. 

\section{The history}

When we send link-state messages around we need to avoid cyclic
paths; if we are not careful we will resend messages forever. We can
solve this in two ways, either we set a counter on each message and
decrement the counter in each hop, hoping that it will reach all
routers before the counter reaches zero, or we keep track of what
messages we have seen so far.

We will try the later strategy but to avoid having to keep a copy of
all messages we will tag each constructed message with a per router increasing
message number. If we know that we have seen message {\tt 15} from
{\tt london} then we know that messages from {\tt london} with a lower
number are old and can be thrown away. This strategy not only avoids
circular loops but also prevents old messages from being delayed and
later be allowed to change our view of the network. 

Implement a data structure called history that keeps track of what
messages that we have seen. In module {\tt hist} implement two
procedures.

{\bf new(Name)} Return a new history, where messages from Name will
always be seen as old.

{\bf update(Node, N, History)} Check if message number N from the Node is
old or new. If it is old then return {\tt old} but if it new return
{\tt \{new, Updated\}} where Updated is the updated history.

To determine if a link-state message is old or new one need of course
not store the message itself nor all previously received messages. The
only thing we have to keep track of is the highest counter value
received from each node. Can you create an entry for a node that will
make any message look old?

\section{The router}

The router should be able to, not only route messages through a
network of connected nodes but also, maintain view of the
network and construct optimal routing tables. Each
routing process will have a state:

\begin{itemize}
\item a symbolic name such as {\tt london}
\item a counter
\item a history of received messages
\item a set of interfaces
\item a routing table
\item a map of the network
\end{itemize}

\noindent When a new router process is created it will sett all its
parameters to initial empty values. We will also register the router
process under a uniqe name (unique for the erlang machine it is
running on, for example \verb+r1+, \verb+r2+, etc). 

\begin{verbatim}
-module(routy).

-export([start/2, stop/1, ...]).

start(Reg, Name) ->
    register(Reg, spawn(fun() -> init(Name) end)).
    
stop(Node) ->
    Node ! stop,
    unregister(Node).

init(Name) ->
    Intf = intf:new(),
    Map = map:new(),
    Table = dijkstra:table(Intf, Map),
    Hist = hist:new(Name),
    router(Name, 0, Msgs, Intf, Table, Map).
\end{verbatim}


\noindent To route a message to a node, the router will simply consult the
routing table to find the best gateway and then find the pid of that
gateway given the list of interfaces. This is the easy part; the hard
part is to maintain a consistent view of the networks as interfaces
are added and removed. The algorithm of a links-state protocol is as follows:

\begin{itemize}
\item determine which nodes that you are connected to
\item tell all neighbors in a {\em link-state message}
\item if you receive a link-state message that you have not seen
before pass it along to your neighbors
\end{itemize}

\noindent A node will thus collect link-state messages from all other
routers in the network. The link-state messages are exactly what we
need to build a map. Since we also know which nodes we can reach
directly, our gateways, we can use Dijkstra's algorithm to generate a
routing table. 

In our first effort we will however only implement a process that can
connect or disconnect to other nodes in the system and update its set
of interfaces.

\subsection{adding interfaces}

We will user {\em monitors} to detect if a node is
unreachable; a monitor will send an {\tt 'DOWN'} message to the
process and we can then remove links to the node. A skeleton code for
the router process could look as follows.

\begin{verbatim}

router(Name, N, Hist, Intf, Table, Map) ->
    receive 
%            :
%            :
        {add, Node, Pid} ->
            Ref = erlang:monitor(process,Pid),
            Intf1 = intf:add(Node, Ref, Pid, Intf),
            router(Name, N, Hist, Intf1, Table, Map);

        {remove, Node} ->
            {ok, Ref} = intf:ref(Node, Intf),
            erlang:demonitor(Ref),
            Intf1 = intf:remove(Node, Intf),
            router(Name, N, Hist, Intf1, Table, Map);

        {'DOWN', Ref, process, _, _}  ->
            {ok, Down} = intf:name(Ref, Intf),
            io:format("~w: exit recived from ~w~n", [Name, Down]),
            Intf1 = intf:remove(Down, Intf),
            router(Name, N, Hist, Intf1, Table, Map);

%            :
%            :

        {status, From} ->
            From ! {status, {Name, N, Hist, Intf, Table, Map}},
            router(Name, N, Hist, Intf, Table, Map);                 

        stop ->
            ok
    end.
\end{verbatim}

Note that creating a monitor for a process that does not exist will
fail nor throw an exception. What will happen is that you're
imediately sent a down message. The behaviour is thus the same if you
add a monitor to a process that dies or if you add monitor to a
process that died 10 milliseconds ago.

The {\tt \{status, From\}} message can be used to do a pretty-print of
the state. Add a function that sends a \verb+status+ message to a
process, receives the reply and displays the information.

When we start Erlang shells we will all have to use the same magic
cookie so let's agree on {\tt routy}. We could also use a flag to
reduce the underlying network traffic. The default behavior for
distributed Erlang is to try to connect to all nodes available in the
network. Connecting A with B where B is already connected to C will
create a connection between A and C. Since we will allow our nodes to
 crash we can turn this feature off.

\begin{verbatim}
erl -name sweden@130.123.112.23 -setcookie routy -connect_all false
\end{verbatim}

To try to keep things under control we name Erlang nodes after
countries and routers after names in that country. Start two routers
and send them messages so that they connect to each other. Terminate
one of them and see that things work.

\subsection{link-state messages}

Next we need to implement the link-state message. When this is sent it
is tagged with the counter value. The counter is then updates so
subsequent messages will have a higher value. When receiving a
links-state message a router must check if this is an old or new
message.  The handling of link-state messages can  be implemented as
follows:

\begin{verbatim}
        {links, Node, R, Links} -> 
           case hist:update(Node, R, Hist) of
            {new, Hist1} -> 
                  intf:broadcast({links, Node, R, Links}, Intf), 
                  Map1 = map:update(Node, Links, Map),
                  router(Name, N, Hist1, Intf, Table, Map1); 
            old ->
                 router(Name, N, Hist, Intf, Table, Map) 
        end;
\end{verbatim}

\noindent When we have updated our map we should also update the routing
table. This is where we invoke the Dijkstra algorithm. We should do it
periodically, maybe every time we receive a link-state message or
better every time the map changes. In our experiment we will do it
manually. We add a method {\tt update} that we will send to order the
router to update its routing table.

\begin{verbatim}
        update ->
          Table1 = dijkstra:table(intf:list(Intf), Map),
          router(Name, N, Hist, Intf, Table1, Map);
\end{verbatim}

\noindent We also add a message so that we manually can order our router to
broadcast a link-state message. This should of course be done
periodically or every time a link is added but we want to experiment
with inconsistent maps so we keep this as a manual procedure.

\begin{verbatim}
        broadcast ->
          Message = {links, Name, N, intf:list(Intf)},
          intf:broadcast(Message, Intf),
          router(Name, N+1, Hist, Intf, Table, Map);       
\end{verbatim}


\subsection{testing what we have}

We can now test our protocol by starting several routing processes and
letting them connect to each other. Let's call Erlang machines for countries
and routers for cities. So start a Erlang node with the command:

\begin{verbatim}
erl -name sweden@130.123.112.23 -setcookie routy -connect_all false
\end{verbatim}

Load the {\tt routy} and {\tt dijkstra} module and then start routers
for different cities in Sweden. Then connect the routers by manually
sending them {\tt add} messages. Note that the add message contains
both the logical name ({\tt stockholm}) and the process identifier of
the router (for example {\tt \{r1, 'sweden@130.123.112.23'\}}).

\begin{verbatim}
(sweden@130.123.112.23)>routy:start(r1, stockholm).

(sweden@130.123.112.23)>routy:start(r2, lund).

(sweden@130.123.112.23)>lund ! {add, stockholm, {r1, 'sweden@130.123.112.23'}}.
true
\end{verbatim}

If everything works out ok, you should be able to build a network of
routers. When you send the message {\tt broadcast} to a router the
link-state messages should be generated and after a {\tt update}
message the routing table should be computed. Try it with some Erlang
nodes running on one machine. If you have problems with the long
network names you could start Erlang using short node names {\tt
-sname} or simply have all routers in the same Erlang process.

\subsection{routing a message}

It's now time to implement the actual routing. We have one easy case
and that is when a message has actually arrived to the final
destination.

\begin{verbatim}
     {route, Name, From, Message} ->
          io:format("~w: received message ~w ~n", [Name, Message]),
          router(Name, N, Hist, Intf, Table, Map);
\end{verbatim}

\noindent If the message is not ours we should forward it. If we find a suitable
gateway in the routing table we simply forward the message to the
gateway. If we do not find a routing entry or do not find a interface
of a gateway we have a problem, simply drop the packet and keep
smiling.

\begin{verbatim}
     {route, To, From, Message} ->
          io:format("~w: routing message (~w)", [Name, Message]),
          case dijkstra:route(To, Table) of
             {ok, Gw} -> 
                case intf:lookup(Gw, Intf) of
                   {ok, Pid} ->
                       Pid ! {route, To, From, Message};
                   notfound ->
                       ok
                end;
             notfound ->
                ok
          end,
          router(Name, N, Hist, Intf, Table, Map);         
\end{verbatim}

\noindent In the implementation we make use of the fact that the
routing table contains entries even for our own gateways. Could we
also have a dummy entry for the node itself so that we would not need
to have a special message entry to handle messages directed to the
router itself?

We also add a message so that a local user can initiate the routing of
a message without knowing the name of the local router.

\begin{verbatim}
     {send, To, Message} ->
          self() ! {route, To, Name, Message},
          router(Name, N, Hist, Intf, Table, Map);
\end{verbatim}

This is how far you should have got before the seminar. Write up a two
page report on what was difficult and how you solved it. At the
seminar we should connect as many routers as possible, start killing
nodes and watch how the network is still able to route messages.

\section{The world}

Form a group and be responsible for a region in the world (Europe,
Africa, South America etc); coordinate with other groups so each group
has it's own region. Then start a set of Erlang nodes on each machine
where you give each Erlang node the name of a country (that is in your
region).

In each Erlang node you can now create one or more routers with
registered names of cities in that country. Then start to connect the
routers to each other. Note that all cities in the world must have
unique names so even if there is a Paris in Texas the network will only
allow one node to be called {\tt paris}. Start to send messages to
other nodes and see that it works. Note that since you have not
implemented automatic broadcast and update functionality, you must do
this manually.

When things are working in your region chose two or more routers that
should connect to other parts of the world. Make sensible connections
to make it easier to understand what the network looks like. Can we
send messages from Sydney to Oslo?

If everything works ok, you can try to either stop routers, close
Erlang nodes or simply disable the network card. Will the routing
functionality still work? How long time does it take between a
disabled network card and the delivery of a 'DOWN' message to the
other nodes?


\end{document}
