\documentclass[a4paper, 11pt]{article}

\usepackage{ifthen}
\usepackage[latin1]{inputenc}


\newcommand{\nnsection}[1]{
\section*{#1}
\addcontentsline{toc}{section}{#1}
}

\begin{document}

\begin{center}
\vspace{20pt}
\textbf{\large Failure detectors in Erlang}\\
\vspace{10pt}
\textbf{Johan Montelius}\\
\vspace{10pt}
\today{}
\end{center}

\nnsection{Introduction}

In this assignment you will look at how failure detectors work in
Erlang. You will first set up a system of two processes running in one
Erlang node and then divide the system so that the processes run in
two nodes on different machines.

\section{Send and receive}

Let us start with a two simple processes, one that keeps sending ping
messages every second and one that happily consume these messages. The
producer will be started first and wait for a consumer to attach to
it. Once the consumer has been connected the ping messages can be
sent, each tagged with an increasing number. We should be able to stop
the producer and it will then send a final message to the consumer.


\subsection{the producer}
 The
producer could look like this, we start by declaring the module and the interface.

\begin{verbatim}
-module(producer).

-export([start/1, stop/0, crash/0]).

start(Delay) ->
    Producer = spawn(fun() -> init(Delay) end),
    register(producer, Producer).
    
stop() ->
    producer ! stop.

crash() ->
    producer ! crash.
\end{verbatim}

The function {\tt start/1} will spawn the new process and register it
under the name {\em producer}. It is given a parameter called {\em
  Delay}, that will be the number of milliseconds between
messages. The function {\tt stop/0} will simply send a stop message
to the process registered under the name {\em producer}.

In the initialization of the producer we wait for a consumer to send
us a message. A consumer will send us a message {\tt \{hello,
  Consumer\}} where {\em Consumer} is the process identifier of the
process that wants to receive the messages.

\begin{verbatim}

init(Delay) ->
     receive 
         {hello, Consumer} ->
             producer(Consumer, 0, Delay);
         stop ->
             ok
     end.
\end{verbatim}

The process is then implemented using the {\tt after} construct that
allows us to wait for a message a certain time before carrying on. If
no stop message is received we send a {\em ping message} to the consumer.

\begin{verbatim}
producer(Consumer, N, Delay) ->
    receive 
        stop -> 
            Consumer ! bye;
        crash ->
            42/0  %% this will give you a warning, but it is ok
    after Delay ->
            Consumer ! {ping, N},
            producer(Consumer, N+1, Delay)
    end.

\end{verbatim}

If we receive a {\em stop message} we send a {\em bye message} to the
consumer and terminate the execution. This is the controlled way and
let the consumer know that it should not expect to see any more
messages. If we receive a {\em crash message} we simply terminate
without informing the consumer. We will see that this of course leads
to problems.

\subsection{the consumer}

The consumer is equally simple, it should have the following properties.

\begin{itemize}
\item It should export two functions: {\tt start/1} that takes a
  process identifier or registered name of a producer as argument and
  {\tt stop/} that terminates the process.

\item When initialized it should send a {\em hello message} to the
  producer before entering its recursive state with a {\em expected value} set to 0.

\item It should receive a sequence of {\em ping messages} and check
  that the message contains the {\em expected value}. If it is the
  correct value it should print the number on the screen and
  continue. If a higher value is received it should print a warning
  before continuing. In both cases the next expected value is one more
  than the received value.

\item If the process receives a {\em bye message} (sent by the
  producer) or a {\em stop message} (sent when calling {\tt stop/0})
  the process should terminate.
\end{itemize}


\section{Detecting a crash}

Start the producer and then start the consumer in the same Erlang
shell. The parameter to the consumer is simply {\tt producer} since
this is the local name under which it is registered. 

You should be able to start the producer and then consumer and see how
the messages are received by the consumer. If you {\em stop} the
producer, the consumer is informed through a {\em bye messages} and
terminates gracefully.

The problem is if we simulate a crash and the producer simply
terminates. The consumer will now be stuck, waiting for a ping or bye
message that will never come. This can be solved by implementing the
consumer using a {\em after} construct. If no message has been
received for ten seconds we can expect that the producer is dead and
do something else. This is a rather crude way of solving the problem
and there is a better way.

The Erlang runtime system can help us since it knows if the producer
is alive or not. In Erlang we have a construct called {\em monitor}
that asks the runtime system to give us information of the state of
another process. To use this feature we start a monitor for the
producer and add one additional clause to the recursive definition.

The monitor is started as follows:

\begin{verbatim}
init(Producer) ->
    Monitor = monitor(process, Producer),
    Producer ! {hello, self()},
    consumer(0, Monitor).
\end{verbatim}

The {\em Monitor} that is returned from the call to {\em monitor/2} is
simply a unique reference so that we can determine which monitor that
was triggered. 

Apart from now having one additional parameter, one more message is
added to the recursive definition.

\begin{verbatim}
   {'DOWN', Monitor, process, Object, Info}  ->
        io:format("~w died; ~w~n", [Object, Info]),
        consumer(N, Monitor);
\end{verbatim}

This message is what we should receive if the producer crashes or
terminates. We knows that the message pertains to the monitor that we
started since the second element of the tuple is identical to our {\em
  Monitor}. The {\em Object} element is the process identifier (or
registered name) of the dead process and the {\em Info} element gives
us information in why it terminated.

You might ask why are we doing a recursive call and it is of course
rather pointless. If the producer died we will not receive any more
messages and we are then stuck in the same situations as before. We
will however later use this feature to show some strange behaviour.

Run the producer and consumer and then crash the producer to see that
it is working.

\section{A two node experiment}

Erlang is, as you know, quite transparent when it comes to
distribution. We can quite easily run our producer and consumer in
different Erlang nodes.

\subsection{on the same host}

Use two Erlang nodes, gold and silver, running on the same
machine. Start them in distributed mode and make sure that you start
hem with the same cookie so that they can communicate.

\begin{verbatim}
>erl -sname gold -setcookie foo
\end{verbatim}

If the producer is started on the node {\em silver} the consumer
should be given the argument {\tt \{producer, 'silver@host'\}} where
{\tt host} is the name of the machine we are running. 

No we can not only fake a crashing producer but kill the Erlang
node that it is running in. What is the message that is given as
reason when the node is killed, why?

\subsection{a distributed experiment}

Now things are getting interesting; run each of the nodes on separate
machines. As first things should be as normal and we should be able to
crash the process. What happens if we kill the Erlang node of the producer?

Now unplug the Ethernet cable to the producer and put it back in again
after a few seconds. What happens? Remove the cable for longer time
periods, what happens?

What does it mean that we have received a {\em DOWN message}? When
should we trust it? 

Have you received any messages out of order even without having
received a {\em DOWN message}? What does the manual say about message
delivery guarantees?

The questions raised are the core problems when implementing
distributed systems. 


\end{document}

