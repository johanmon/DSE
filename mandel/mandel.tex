\documentclass[a4paper,11pt]{article}

\usepackage{ifthen}
\usepackage[latin1]{inputenc}

\usepackage{verbatim}

\usepackage[breaklinks=true,
            bookmarksnumbered=true,
            pdftitle={Distributed Computing},
            pdfauthor={Johan Montelius},
            pdfsubject={Distributed Mandelbrot set},
            pdfkeywords={distributed computing erlang}
            ]{}

\newcommand{\nnsection}[1]{
\section*{#1}
\addcontentsline{toc}{section}{#1}
}

\begin{document}

\begin{center}
\vspace{20pt}
\textbf{\large Generating a Mandelbrot image }\\
\vspace{10pt}
\textbf{Johan Montelius}\\
\vspace{10pt}
\today{}
\end{center}

\nnsection{Introduction}

In this exercise you will implement a distributed Mandelbrot set
generator, or rather an image generator. Since calculating Mandelbrot
sets can be quite time consuming we would of course like to make use
of as much resources as possible and then why not try to use all the
laptops we have in a class.

You should do some reading so that you understand the basics of what
the Mandelbrot set is and why it can generate some beautiful
images; this text only contains a minimal explanation.

\section{Mandelbrot}

The Mandelbrot set is defined as the set of complex numbers $c$ for
which the sequence $z_n$ does not approach infinity. The value $z_n$
is defined as follows:

\begin{eqnarray*}
    z_0 &= &0 \\ 
    z_{n+1} & = &z_n� + c
\end{eqnarray*}

If you remember how to do the square of a complex numbers you know
everything there is to know to start:

$$ (x + yi)� = x� - y� + 2xyi$$

How do we know is a complex number $(x + yi)$ belongs to the
Mandelbrot set. We could of course start to compute $z_n$ for higher
values and see if we ever hit infinity but that would of course take
a very long time (to say the least). 

An observation saves us from spending the rest of our lives computing
$z_n$: {\em if $|z_n| >= 2$ then there is no turning back, $z_n$ will
  only increase in size}. The absolute value of a complex number is of
course:

$$|(x + yi)| = \sqrt{x� + y�}  $$


So given a maximum value of $n$, we can for
any complex number $c$ say if {\em definitely does not belong} to the
Mandelbrot set or it {\em could possibly belong} to the set but we are
not quite sure. In the case where we know for sure that the number does
not belong to the set we also have a value $i$ which was the point
where $|z_i| >= 2$. This value $i$ is the color we need to generate a
beautiful Mandelbrot image.

\subsection{complex numbers}

Since we are going to work with complex numbers we might as well start
by implementing a module to handle these. Let's make it simple and
represent a complex number as a tuple with its real and imaginary
values. Create a module {\tt cmplx} that exports the following
functions:

\begin{itemize}
 \item {\tt new(X, Y)} : returns the complex number with real value X and imaginary Y
 \item {\tt add(A, B)} : adds two complex numbers
 \item {\tt sqr(A)} : squares a complex number
 \item {\tt abs(A)} : the absolute value of A
\end{itemize}

You have a choice of how to represent complex numbers but why not make
it easy and represent them as a tuple {\tt \{X, Y\}} that will give us
easy access to its components.

You might want to use the {\tt sqrt/1} function exported from the {\tt
  math} module when calculating the absolute value.

\subsection{the brot client}

For no reason at all we will call our first module {\tt brot}, it will
implement the client process that will do the computation to determine
if a given value $x + yi$ possibly belongs to the set. 

Implement a function {\tt mandelbrot/3} that is given the complex
number and the maximum number of iterations, $m$, that it is
allowed to use. It should return the value $i$ at which $|z_i| >= 2$ or
$0$ if it does not for any $i < m$ i.e. it should always return a
value in the range $0..(m-1)$.

\begin{verbatim}
mandelbrot(C, M) -> 
    Z = cmplx:new(...,...),
    I = 0,
    test(I, Z, C, M).
\end{verbatim}

The {\tt test/4} function should of course test first of all if we
have reached the maximum iteration, in which case it returns zero, or if
the absolute value of {\tt Z} is greater than 4, in which case it returns
{\tt I}. Make sure that you use the functions exported from the {\tt cmplx}
module.

Do some test to see that the function works (here I'm writing the
complex numbers directly knowing that we represent them as tuples,
this is of course cheating but very convenient).

\begin{itemize}
 \item {\tt brot:mandelbrot(\{0.8, 0\}, 30)}
 \item {\tt brot:mandelbrot(\{0.5, 0\}, 30)}
 \item {\tt brot:mandelbrot(\{0.3, 0\}, 30)}
 \item {\tt brot:mandelbrot(\{0.27, 0\}, 30)} 
 \item {\tt brot:mandelbrot(\{0.26, 0\}, 30)}
 \item {\tt brot:mandelbrot(\{0.255, 0\}, 30)}
\end{itemize}

What is happening? Which values could possibly belong to the
Mandelbrot set - how sure are you? Do some more testing, why stop at
thirty iterations?


\subsection{a process}

Once we have a working mandelbrot tester the rest is just
engineering. We will wrap the Mandelbrot test function in a process
that will eagerly contact a server to request for something to
test. The server will reply with a task that also includes a process
identifier to which the result should be sent. 


These are the messages that we will use:

\begin{itemize}
 \item {\tt \{req, From\}}: a request for work from client From
 \item {\tt \{pos, Pr, P, C, M\}}: a position, P, to investigate given
   complex value C and maximum iterations M, result is sent to the
   printer Pr
 \item {\tt \{res, P, Res\}}: the result Res at position P sent to the printer
 \item {\tt done}: no more work to be found, terminate
\end{itemize}

The position is a tuple {\tt \{C, R\}}
i.e. the column and row of the image that we will eventually
generate i.e. the position could be {\tt \{230,435\}} in a
1920x1080 image, while the complex value that we are looking for is
{\tt \{0.34, 1,2\}}. Note that the {\tt brot} module does not really
care what the position is, it will simply pass this information to
the printer process.

Here is some skeleton code that will get you starting:

\verbatiminput{brot.erl}

We now have a process that could be helpful in generating a Mandelbrot
image. We only need a server process that can delegate work and a
printing process that will take the generated image and write it to a file.

\subsection{the server}

Open a new module called {\tt mandel}, this module will implement a
server process that keeps track of and delegate work to the brot
clients. The process will delegate tasks row by row and pixel by
pixel. In the end the printer should be sent the calculated values for
each pixel so it can assemble them into a picture. The client is thus
given the current pixel to calculate but also the corresponding
complex value. We therefor need to map the row and column value to
complex value but this is of course depending on what image we are
looking for.

We solve this by telling the mandel server the lower left complex
value of our image and the increment it should use to calculate the
complex values for the pixels. Assume that we want to generate a 1920
by 1080 image starting at (-0.5,0.2i) we could use an increment of
0.001. The image would then have its upper right corner in (1.980-0.5,
(1.08+0.2)i).

We also need to pass a maximum value of iterations that the brot
client should use. In order to make it easier for the printer function
to choose the colors we tell the printer that we will use a color depth
of each R-G-B component of $d$ while the brot client can use $d�$
iterations. For example, if the color depth is 8 (that is three bits
for each component) the brot client can then use a depth of $8�$ and
will thus reply with a value in the range $0...511$. This value
can then easily be transformed to RGB components. 

This is just one simple way of coloring our picture and if you want to
explore Mandelbrot images further there is a lot to learn in how to do
the coloring right.


\subsection{the printer}

The printer process should when started be given the width, height and
color depth (remember depth of each RGB component) of the image that
should be generated. It will sit and wait for incoming messages that
hopefully are sent in order.

Each incoming value is converted to a RGB components and this could,
as mentioned above, be done in more than one way. In this simple
example the value is calculate modulo the color depth to obtain the R
value. It is then divided by the depth and second modulo operation
gives us the G value, two more operations and we have the B
value. This will give an increasing red color that is then switched to
a dark red color with some green in. Look at the generated images and
you will soon ask yourself how the coloring can be done differently.

When we have all pixels the image is written to a file. We could write
the image incrementally but there is a large overhead in each file
operations so we rather wait as long as possible.

How do we know that messages arrive in order? Do they have to arrive
in order? What do we do if a brot client crashes, should we wait
forever for a value?

The given printer module generates a so call PPM file. This is a very
crude image format that does not compress the image. A better solution
would be to generate a PNG image but this is not for this exercise -
but not as hard as you might think so do some reading and you will
solve it, you will earn a lot about compression.


\end{document}
