\documentclass[a4paper,11pt]{article}


\usepackage{ifthen}
\usepackage[latin1]{inputenc}


\newcommand{\nnsection}[1]{
\section*{#1}
\addcontentsline{toc}{section}{#1}
}

\begin{document}

\begin{center}
\vspace{20pt}{}
\textbf{\large Chordy: a distributed hash table}\\
\vspace{10pt}
\textbf{Johan Montelius}\\
\vspace{10pt}
\today{}
\end{center}

\nnsection{Introduction}

In this assignment you will implement a distributed hash table
following the Chord scheme. In order to understand what you're about
to do you should have a basic understanding of Chord and preferably
have read the original paper. 

The first implementation will only maintain a ring structure; we will
be able to add nodes in the ring but not add any elements to the
store. Once we have a growing ring we will introduce a store where
key-value pairs can be added. Adding and searching for values will
only introduce a few new messages and one parameter to represent the
store. When we have the distributed store we can perform some
benchmarks to see if the distribution actually gives us anything.

Moving forward we will add failure detection to the system. Each node
will keep track of the liveness of its successor and predecessor, if
they fail the ring must be repaired. If a predecessor dies we don't do
very much but if our successor dies we have to contact the next in
line. In the presentation of Chord one will keep a list of potential
successors but to keep things simple we will only keep track of one
more successor.

Maintaining the ring in the face of failures is of course all well but
if a node dies we will loose information. To solve this problem we
will have to replicate the values in the store. We will introduce a
simple replication scheme that might take care of some problems but
does it actually work.

The Chore architecture also define a routing table, called fingers,
for each node. The routing table will allow us to find any given key in
$log(n)$ hops. This is of course important if we have a large ring. In
a network of twenty nodes it will however be quite manageable to
rely on the successor pointers only.

If you also want to implement mutable objects you will be faced with a
challenge. How do you consistently update an object if it's
replicated? Some objects or nodes might be unavailable during node
insertion and failures. To solve this you will have
to do some more reading.

\section{Building a ring}

Start this project implementing a module {\tt node1}. It will be our
first implementation that only handles a growing ring. It could be
good to keep this very simple, to see what functionality
introduces new messages.

\subsection{keys}

Chord uses hashing of names to create unique keys for objects. We will
not use a hash function, instead a random number generated is used
when a new key is generated. We will thus not have any ``names'' only
keys. A node that wants to join the network will generate a random
number and we will hope that this is unique. This is ok for all of our
purposes.

In a module {\tt key} implement two functions: {\tt generate()} and
{\tt between(Key, From, To)}. The function {\tt generate/0} will
simply return a random number from 1 to 1.000.000.000 (30-bits), that
will be large enough for our test. Using a hash function such as SHA-1
would give us 160 bits and allow us to have human readable names on
object but let's keep things simple. Use the Erlang module {\tt
  random} to generate numbers.

The {\tt between/3} function will check if a {\tt Key} is between {\tt
  From} and {\tt To} or equal to {\tt To}, this is called a partly 
closed interval and is denoted $(From, To]$. 

Remember that the we're dealing with a ring so it could be that {\tt
  From} is larger than {\tt To}. What does that mean and how do you
handle it? Also, {\tt From} could be equal to {\tt To} and we will
interpret this as the full circle i.e. anything is in between. 


\subsection{the node}

A node will have the following properties: a key, a predecessor and a
successor (remember that we will wait with the store until later). We
need to know the key values of both the predecessor and the successor
so these will be represented by a tuples {\tt \{Key, Pid\}}. 

The messages we need to maintain the ring are:

\begin{itemize}
\item {\tt \{key, Qref, Peer\}} : a peer needs to know our key
\item {\tt \{notify, New\}} : a new node informs us of its existence
\item {\tt \{request, Peer\}} : a predecessor needs to know our predecessor
\item {\tt \{status, Pred\}} : our successor informs us about its predecessor
\end{itemize}

If you read the original paper that describes the Chord architecture
they of course describe it a bit different. They have a pseudo code
description where they call functions on remote nodes. Since we need
to build everything on message passing, and need to handle the case
where there is only one node in the ring pointing to itself, we need
to make things asynchronous. 

If we delay all the tricky decision to sub-routines the implementation
of the node process could look like this:

\begin{verbatim}
node(Id, Predecessor, Successor) ->
    receive 
        {key, Qref, Peer} ->
            Peer ! {Qref, Id},
            node(Id, Predecessor, Successor);

        {notify, New} ->
            Pred = notify(New, Id, Predecessor),
            node(Id, Pred, Successor);

        {request, Peer} ->
            request(Peer, Predecessor),
            node(Id, Predecessor, Successor);

        {status, Pred} ->
            Succ = stabilize(Pred, Id, Successor),
            node(Id, Predecessor, Succ);
   end.
\end{verbatim}

You could also include handlers to print out some state information,
to terminate and a catch all clause in case some strange messages are
sent. 

\subsection{stabilize}

The periodic stabilize procedure will consist of a node sending a
{\tt \{request, self()\}} message to its successor and then expecting a
{\tt \{status, Pred\}} in return. When it knows the predecessor of its
successor it can check if the ring is stable or if the successor needs
to be notifies about its existence through a {\tt \{notify, \{Id,
  self()\}} message.

Below is a skeleton for the {\tt stabilize/3} procedure. The {\tt Pred}
argument is ours successors current predecessor. If this i {\tt nil}
we should of course inform it about our existence. If it is pointing
back to us we don't have to do anything. If it is pointing to itself
we should of course notify it about our existence.

If it's pointing to another node we need to be careful. The question
is if we are to slide in between the two nodes or if we should place
ourselves behind the predecessor. If the key of the predecessor of our
successor ({\tt Xkey}) is between us and our successor we should of
course adopt this node as our successor and run stabilization
again. If we should be in between the nodes we inform our successor of
our existence.


\begin{verbatim}
stabilize(Pred, Id, Successor) ->
    {Skey, Spid} = Successor,
    case Pred of
        nil ->
             :

        {Id, _} ->
             :
        {Skey, _} ->
             :

        {Xkey, Xpid} ->
            case key:between(Xkey, Id, Skey) of
                true ->
                     :
                false ->
                     :
            end
    end.
\end{verbatim}

If you study the Chord paper you will find that they explain things
slightly different. We use our {\tt key:between/3} function that
allows the key {\tt Xkey} to be equal to {\tt Skey}, not strictly
according to the paper. Does this matter? What does it mean that {\tt
  Xkey} is equal to {\tt Skey}, will it ever happen?


\subsection{request}

The stabilize procedure must be done with regular intervals so
that new nodes are quickly linked into the ring. This can be arranged
by starting a timer that sends a {\tt stabilize} message after a
specific time. 

 The following function will set up a timer and send
the request message to the successor after a predefined interval. In
our scenario we might set the interval to be 1000 ms in order to slowly
trace what messages are sent.


\begin{verbatim}
schedule_stabilize() ->
    timer:send_interval(?Stabilize, self(), stabilize).
\end{verbatim}

The procedure {\tt schedule\_stabilize/1} is called when a node is
created. When the process receives a {\tt stabilize} message it will
call {\tt stabilize/1} procedure. 

\begin{verbatim}
        stabilize ->
            stabilize(Successor),
            node(Id, Predecessor, Successor);
\end{verbatim}


The {\tt stabilize/1} procedure will then simply send a {\tt request}
message to its successor. We could have set up the timer so that it was
responsible for sending the request message but then the timer would
have to keep track of which node was the current successor. 

\begin{verbatim}
stabilize({_, Spid}) ->
    Spid ! {request, self()}.
\end{verbatim}

The request message is picked up by a process and the {\tt request/2}
procedure is called. We should of course only inform the peer that
sent the request about or predecessor as in the code below. The
procedure is over complicated for now but we will later extend it to be
more complex.

\begin{verbatim}
request(Peer, Predecessor) ->
    case Predecessor of
        nil ->
            Peer ! {status, nil};
        {Pkey, Ppid} ->
            Peer ! {status, {Pkey, Ppid}}
    end.
\end{verbatim}

What are the pros and cons of a more frequent stabilizing
procedure? What is delayed if we don't do stabilizing that often?

\subsection{notify}

Being notified of a node is a way for a node to make a friendly
proposal that it might be our proper predecessor. We can not take
their word for it, so we have to do our own investigation.

\begin{verbatim}
notify({Nkey, Npid}, Id, Predecessor) ->
    case Predecessor of
        nil ->
             :
        {Pkey,  _} ->
            case key:between(Nkey, Pkey, Id) of
                true ->
                     :
                false -> 
                     :
            end
    end.
\end{verbatim}

If our own predecessor is set to nil the case is closed but if we
already have a predecessor we of course have to check if the new node
actually should be our predecessor or not. Do we need a special case
to detect that we're pointing to ourselves?

Do we have to inform the new node about our decision? How will
it know if we have discarded its friendly proposal?

\subsection{starting a node}

The only thing left is how to start a node. There are two
possibilities: either we are the first node in the ring or we're
connecting to an existing ring. We'll export two procedures, {\tt
  start/1} and {\tt start/2}, the former will simply call the later
with the second argument set to {\tt nil}.

\begin{verbatim}
start(Id) ->
    start(Id, nil).

start(Id, Peer) ->
    timer:start(),
    spawn(fun() -> init(Id, Peer) end).
\end{verbatim}

In the {\tt init/2} procedure we set our predecessor to {\tt nil},
connect to our successor and schedule the stabilizing procedure; or
rather making sure that we send a {\tt stabilize} message to
ourselves. This also has to be done even if we are the only node in the
system. We then call the {\tt node/3} procedure that implements the message
handling.

\begin{verbatim}
init(Id, Peer) ->
    Predecessor = nil,
    {ok, Successor} = connect(Id, Peer),
    schedule_stabilize(),    
    node(Id, Predecessor, Successor).
\end{verbatim}

The {\tt connect/2} procedure is divided into two cases; are we the
first node or trying to connect to an existing ring. In either case we
need to set our successor pointer. If we're all alone we are of course
our own successors. If we're connecting to an existing ring we send a
{\tt key} message to the node that we have been given and wait
for a reply. Below is the skeleton code for the {\tt connect/2} procedure.

\begin{verbatim}
connect(Id, nil) ->
    {ok, .......};    
connect(Id, Peer) ->
    Qref = make_ref(),
    Peer ! {key, Qref, self()},
    receive
        {Qref, Skey} ->
               :
    after ?Timeout ->
            io:format("Time out: no response~n",[])
    end.
\end{verbatim}

Notice how the unique reference is used to trap exactly the message
we're looking for. It might be over-kill in this implementation but it
can be quite useful in other situations. Also note that if we for
some reason do not receive a reply within some time limit (for example
10s) we return an error.

What would happen if we didn't schedule the stabilize procedure?
Would things still work?

The Chord system uses a procedure that quickly will bring us
closer to our final destination but this is not strictly needed. The
stabilization procedure will eventually find the right position for us.

\subsection{does it work}

Do some small experiments, to start with in one Erlang machine but
then in a network or machines. When connecting nodes on different
platforms remember to start Erlang i distributed mode (giving a -name
argument) and make sure that you use the same cookie (-setcookie). 

To check if the ring is actually connected we can introduce a probe message.

\begin{verbatim}
{probe, I, Nodes, Time}
\end{verbatim}

If the second element, {\tt I}, is equal to the {\tt Id} of the node,
we know that we sent it and can report the time it took to pass it
around the ring. If it is not our probe we simply forward it to our
successor but add our own process identifier to the list of nodes. 

The time stamp is set when creating the probe, use {\tt erlang:system\_time(micro\_seconds)}
to get microsecond accuracy (this is local time so the times-tamp does
not mean anything on other nodes). 

\begin{verbatim}
        probe ->
            create_probe(Id, Successor),
            node(Id, Predecessor, Successor);

        {probe, Id, Nodes, T} ->
            remove_probe(T, Nodes),
            node(Id, Predecessor, Successor);

        {probe, Ref, Nodes, T} ->
            forward_probe(Ref, T, Nodes, Id, Successor),
            node(Id, Predecessor, Successor);
\end{verbatim}

If you run things distributed you must of course register the first
node under a name, for example {\tt node}. The remaining nodes will
then connect to this node using for example the address: 

\begin{verbatim}
{node,  'chordy@192.168.1.32'}
\end{verbatim}

The connection procedure will send a name to this registered node and
get a proper process identifier of a node in the ring. If we had
machines registered in a DNS server we could make this even more
robust and location independent.

\section{Adding a store}

We will now add a local store to each node and the possibility to add
and search for key-value pairs. Create a new module, {\tt node2}, from
a copy of {\tt node1}. We will now add and do only slight
modifications to or existing code.

\subsection{a local storage}

The first thing we need to implement is a local storage. This could
easily be implemented as a list of tuples {\tt \{Key, Value\}}, we can
then use the key functions in the {\tt lists} module to search for
entries. Having a list is of course not optimal but will do for our
experiments.

We need a module {\tt storage} that implements the following functions: 

\begin{itemize}
 \item {\tt create()}: create a new store
 \item {\tt add(Key, Value, Store)}:  add a key value pair, return the updated store
 \item {\tt lookup(Key, Store)}: return a tuple \{Key, Value\} or the atom false

 \item {\tt split(From, To, Store)} return a tuple \{Updated, Rest\} where the updated store only contains the key value pairs requested and the rest are found in a list of key-value pairs

 \item {\tt merge(Entries, Store)}: add a list of key-value pairs to a store
\end{itemize}

The split and merge functions will be used when a new node joins the ring and should take over part of the store.

\subsection{new messages} 

If the ring was not growing we would only have to add two new
messages: {\tt \{add, Key, Value, Qref, Client\}} and {\tt \{lookup, Key, Qref,
  Client\}}. As before we implement the handlers in separate
procedures. The procedures will need information about the
predecessor and successor in order to determine if the message is
actually for us or if it should be forwarded.

\begin{verbatim}
        {add, Key, Value, Qref, Client} ->
            Added = add(Key, Value, Qref, Client, 
                        Id, Predecessor, Successor, Store),
            node(Id, Predecessor, Successor, Added);

        {lookup, Key, Qref, Client} ->
            lookup(Key, Qref, Client, Id, Predecessor, Successor, Store),
            node(Id, Predecessor, Successor, Store);
\end{verbatim}

The {\tt Qref} parameters will be used to tag the return message to the
{\tt Client}. This allows the client to identify the reply message and
makes it easier to implement the client.

\subsection{adding an element}

To add a new key value we must first determine if our node is the node
that should take care of the key. A node will take care of all keys
from (but not including) the identifier of its predecessor to (and
including) the identifier of itself. If we are not responsible we simply
send a add message to our successor.

\begin{verbatim}
add(Key, Value, Qref, Client, Id, {Pkey, _}, {_, Spid}, Store) ->
    case .......................  of
        true ->
              Client ! {Qref, ok},
              :
        false ->
              :
              :
    end.
\end{verbatim}

\subsection{lookup procedure}

The lookup procedure is very similar, we need to do the same test to
determine if we are responsible for the key. If so we do a simple
lookup in the local store and then send the reply to the requester. If
it is not our responsibility we simply forward the request.

\begin{verbatim}
lookup(Key, Qref, Client, Id, {Pkey, _}, Successor, Store) ->
    case ................. of
        true ->
            Result = storage:lookup(Key, Store),
            Client ! {Qref, Result};
        false ->
            {_, Spid} = Successor,
              :
    end.
\end{verbatim}

\subsection{responsibility}

Things are slightly complicate by the fact that new nodes might join
the ring. A new node should of course take over part of the
responsibility and must then of course also take over already added
elements. We introduce one more message to the node, {\tt \{handover,
  Elements\}}, that will be used to delegate responsibility.

\begin{verbatim}
        {handover, Elements} ->
            Merged = storage:merge(Store, Elements),
            node(Id, Predecessor, Successor, Merged);       
\end{verbatim}

When should this message be sent? It's a message from a node that has
accepted us as their predecessor. This is only done when a node
receives and handles a {\tt notify} message. Go back to the
implementation of the {\tt notify/3} procedure. Handling of a notify
message could mean that we have to give part of a store away; we need
to pass the store as an argument also return a tuple
{\tt\{Predecessor, Store\}}. The procedure {\tt notify/4} could look
like follows:

\begin{verbatim}
notify({Nkey, Npid}, Id, Predecessor, Store) ->
    case Predecessor of
        nil ->
            Keep = handover(Id, Store, Nkey, Npid),
                  :
        {Pkey,  _} ->
            case key:between(Nkey, Pkey, Id) of
                true ->
                      :
                      :
                false -> 
                      :
            end
    end.
\end{verbatim}

So, what's left is simply to implement the {\tt handover/4}
procedure. What should be done: split our {\tt Store} based on the
{\tt NKey}. Which part should be kept and which part should be handed
over to the new predecessor? You have to check how your split function
works, remember that a store contains the range $(Pkey, Id]$, that is
from (not including $Pkey$ to (including) $Id$. What part should be
handed over to our new predecessor?

\begin{verbatim}
handover(Id, Store, Nkey, Npid) ->
    {...., ....} = storage:split(Id, Nkey, Store),
    Npid ! {handover, Rest},
    Keep.
\end{verbatim}

\subsection{performance}

If we now have a distributed store that can handle new nodes that are
added to the ring we might try some performance testing. You need to
be a group with several machine to do this. Assume that we have eight
machines and that we will use four in building the ring and four in
testing the performance. 

As a first test we can have one node only in the ring and let the four
test machines add 1000 elements to the ring and then do a lookup of
the elements. Does it take longer for one machine to handle 4000
elements rather than four machines that do 1000 elements each. What
is the limiting factor?

Implement a test procedure that adds a number of random key-value
pairs into the system and keeps the keys in a list. You should then be
able to do a lookup of all the keys and measure the time it takes. The 
lookup test should be given the name of a node to contact.

Now what happens if we add another node to the ring, how does the
performance change? Does it matter if all test machines access the same
node? Add two more nodes to the ring, any changes? How will things
change if we have a ten thousand elements?


\section{Handling failures}

To handle failures we need to detect if a node has failed. Both the
successor and the predecessor need to detect this and we will use the
Erlang built-in procedure to {\tt monitor} the health of a node. Start a
new module {\tt node3} and copy what we have from {\tt node2}. As you
will see we will not have to do large changes to what we have.

\subsection{successor of our successor}

If our successor dies we need a way to repair the ring. A simple
strategy is to keep track of our successors successor; we will call
this node the {\tt Next} node. A more general scheme is to keep track
of a list of successors to make the system even more fault
tolerant. We will be able to survive from one node crashing at a time
but if two node in a row crashes we're doomed. That's ok, it makes
life a bit simpler.

Extend the {\tt node/4} procedure to a {\tt node/5} procedure,
including a parameter for the {\tt Next} node. The {\tt Next} node
will not change unless our successor informs us about a change. Our
successor should do so in the {\tt status} message so we extend this to
{\tt\{status, Pred, Nx\}}. The procedure {\tt stabilize/3} must now be
replaced by a {\tt stabilize/4} procedure that also takes the new {\tt
  Nx} node and returns not only the new successor but also the new
next node.

\begin{verbatim}
        {status, Pred, Nx} ->
            {Succ, Nxt} = stabilize(Pred, Nx, Id, Successor),
            node(Id, Predecessor, Succ, Nxt, Store);
\end{verbatim}

Now {\tt stabilize/4} need to do some more thinking. If our successor
does not change we should of course adopt the successor of our
successor as our next node. However, if we detect that a new node has
sneaked in between us and our former successor then ... yes then what?
Do the necessary changes to {\tt stabilize/4} so that it returns the
correct successor and next node. 

Almost done but who sent the {\tt status} message? This is sent by the
{\tt request/2} procedure. This procedure must be changed in
order to send the correct message, a small change and your done.

\subsection{failure detection}

We will the {\tt erlang:monitor/2} procedures to detect failures. The
procedure returns a unique reference that can be used to determine which
{\tt 'DOWN'} message belong to which process. Since we need to keep
track of both our successor and our predecessor we will extend the
representation of these nodes to a tuple {\tt \{Key, Ref, Pid\}} where
the {\tt Ref} is a reference produced by the monitor procedure. To
make the code more readable we add wrapper functions for the built-in
monitor procedures.

\begin{verbatim}
monitor(Pid) ->
    erlang:monitor(process, Pid).

drop(nil) ->
    ok;
drop(Pid) ->
    erlang:demonitor(Pid, [flush]).
\end{verbatim}

Now go through the code and change the representation of the
predecessor and successor to include also the monitor reference. In the
messages between node we still send only the two element tuple {\tt
  \{Key, Pid\}} since the receiving node has no use of the reference
element of the sending node. When a new nodes is adopted as successor
or predecessor we need to de-monitor the old node and monitor the new.

There are only four places where we need to create a new monitor
reference and only two places where we de-monitor a node. 

\subsection{Houston we have a problem}

Now to the actual handling of failures. When a process is detected as
having crashed (the process terminated, the Erlang machine died or the
computer stopped replying on heart beats) a message will be sent to a
monitoring process. Since we now monitor both our predecessor and
successor we should be open to handle both messages. Let's follow our
principle of keeping the main loop free of gory details and handle all
decisions in a procedure. The extra message handler could then look
like follows:

\begin{verbatim}
        {'DOWN', Ref, process, _, _} ->
            {Pred, Succ, Nxt}  = down(Ref, Predecessor, Successor, Next),
            node(Id, Pred, Succ, Nxt, Store);
\end{verbatim}

The {\tt Ref} obtained in the {\tt 'DOWN'} message must now be compare
to the saved references of our successor and predecessor. For clarity we
break this up into two clauses.

\begin{verbatim}
down(Ref, {_, Ref, _}, Successor, Next) ->
          :
down(Ref, Predecessor, {_, Ref, _}, {Nkey, Npid}) ->
          :
          :
    {Predecessor, {Nkey, Nref, Npid}, nil}.
\end{verbatim}

If our predecessor died things are quite simple. There is no way for us
to find the predecessor of our predecessor but if we set our
predecessor to {\tt nil} someone will sooner or later knock on our
door and present them self as a possible predecessor. 

If our successor dies, things are almost as simple. We will of course
adopt our next-node as our successor and then only have to remember two
things: monitor the node and make sure that we run the stabilizing
procedure.

You're done you have a fault tolerant distributed storage...... well
almost, if a node dies it will bring with it a part of the storage. If
this is ok we could stop here, if not we have to do some more work.

Another thing to ponder is what will happen if a node is falsely
detected of being dead? What will happen if a node has only been
temporally unavailable (and in the worst case, it might think that the
rest of the network is gone). How much would you gamble in trusting
the {\tt 'DOWN'} message?


\section{Replication}

The way to maintain the store in face of dying nodes is of course to
replicate information. How to replicate is a research area of its own
so we will only do something simple that (almost) works. 

\subsection{close but no cigar}

We can handle failures of one node at a time in our ring so let's
limit the replication scheme to be on the same level. If a node dies
its local store should be replicated so it's successor can take over the
responsibility. Is there then a better place to replicate the store
than at the successor?

When we add an key-value element to our own store we also forward it
to our successor as a {\tt \{replicate, Key, Value\}} message. Each
node will thus have a second store called the {\tt Replica} where it
can keep a duplicate of its predecessors store. When a new node joins
the ring it will as before takeover part of the store but also part of
the replica.

If a node dies it's successor is of course responsible for the store
held by the node. This mean that the {\tt Replica} should be merged
with its own {\tt Store}. Sounds simple does it not there are however
some small details that makes our solution less than perfect. 

\subsection{the devil in the detail}

What does it mean that a element has been added to the store. Can we
send a confirmation to the client and then replicate the element, what
if we fail? If we want to handle failures we should make sure that a
confirmation only is sent when an element has been properly added and
replicated. This should not be too difficult to implement, who should
send the confirmation?

A client that does not receive a confirmation could of course
choose to re-send the message. What happens if the element was
added the first time but that the confirmation message was lost? Will
we handle duplicates properly?

Another problem has to do with a joining node in combination with
adding of a new value. You have to think about this twice before
realizing that we have a problem. What would a solution look like?

Are there more devils? Will we have a implementation with no obvious
faults or an implantation with obviously no faults? Take a copy of
{\tt node3}, call it {\tt node4} and try to add a replication
strategy.

\section{Carrying on}

There are more things that we could add or change and there is often
not only one way of doing things. It becomes a trade-off between
properties that we want and efficiency in the implementation. Sometimes
the properties are driving in opposite directions, such as high
availability and consistence, and one has to make up ones mind what
property is actually needed most.

\subsection{routing}

Routing is one thing that we have left out completely. If we only have
twenty nodes it is less of a problem but if we have hundred nodes it
does become important.

If network latency is high (think global Internet distances) then
we need to do something. Should we even try to take network distances
into account and route to nodes that are network wise close to us
(remember that the ring is an over lay and does not say anything about
network distance). 

\subsection{replication}

Is one replica enough or should we have two or three replicas? On what
does this depend? It is of course related to how reliable nodes are
and what reliability we need to provide, is it also dependent on the
number of nodes in the ring?

Can we use the replicas for read operations and thus make use of the
redundancy. How are replicas found and can we distribute them in the
ring to avoid hot-spots?

\subsection{mutable object}

As long as we only have one copy of an object things are simple, but
what if we want to update objects and some replicas are not
updated. Do we have to use a two-phase-commit protocol to update all
replicas in a consistent way? Could we have a trade off between the
expense of read and write operations? How relaxed can we be and how
does this relate to a shopping cart?

\section{Conclusions}

If you have followed this tutorial implementation you should have a
better understanding of how distributed hash tables work and how they
are implemented. As you have seen its not that hard to maintain a ring
structure even in the face of failures. A distributed store also seams
easy to implement and replication could probably be
solved. Consistency is a problem, can we guarantee that added values
never are lost (given a maximum number of failed nodes). 

When things get complicate to implement the performance might
suffer. What is the advantage of the distributed store, is it
performance or fault tolerance? What should we optimize if we have to
choose between them.


\end{document}
