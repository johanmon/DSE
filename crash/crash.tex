\documentclass[a4paper, 11pt]{article}

\usepackage{ifthen}
\usepackage[latin1]{inputenc}

\newcommand{\nnsection}[1]{
\section*{#1}
\addcontentsline{toc}{section}{#1}
}

\begin{document}

\begin{center}
\vspace{20pt}
\textbf{\large An Erlang primer}\\
\vspace{10pt}
\textbf{Johan Montelius}\\
\vspace{10pt}
\today{}
\end{center}

\nnsection{Introduction}

This is not a crash course in Erlang since there are plenty of
tutorials available on the web. I will however describe the tools that
you need so that you can get a programming environment up and
running. I will take for granted that you know some programming
languages, have heard of functional programing and that recursion is
not a mystery to you.

\section{Getting started}

If you run on your own computer you need to install the Erlang
development environment. If you're running a one of the KTH computers
this is probably already done.

The first thing you need to do is download the Erlang SDK. This is
found at www.erlang.org and is available both as a Windows binary and
as source that you can make with the usual tools available on a Unix
or even MacOs platform. If your using for example Ubuntu then the
Erlang system is avialable in the repository.

The SDK includes a compiler, all the programming libraries and virtual
machine. It does not include a development environment. You can also
download the Erlang manual in HTML.

\subsection{Erlang}

When you have installed Erlang you should be able to start an Erlang
shell. You do this either by starting Erlang found in the regular
Windows program listing or by hand from a regular shell. Since you
later will need to set runtime parameters you need to learn how to
start erlang by hand. Open a regular shell and type:

\begin{verbatim}
erl
\end{verbatim}

\noindent This should start the Erlang shell and you should see someting like this:

\begin{verbatim}
Erlang (BEAM) emulator version 5.6.1 [source] [smp:4] 
[async-threads:0] [hipe] [kernel-poll:false]

Eshell V5.6.1  (abort with ^G)
1>  
\end{verbatim}

\noindent Type {\tt help().} (note the dot) followed by return to see all the
shell commands and {\tt halt().} to exit the shell.


\subsection{Emacs}

As an development environment you need a text editor and what is
better than Emacs. Download it from www.gnu.org/software/emacs/,
available both for Unix, Mac and Windows. Ubuntu users will find it in
the repository.

You need to add the code below to your {\tt .emacs} file (provided
that you have Erlang installed under {\tt C:/Program Files/}). This
will make sure that the Erlang mode is loaded as soon as you open a
{\tt .erl} file and that you can start an Erlang shell under Emacs
etc. Change the {\tt <Ver>} and {\tt <ToolsVer>} to what is right in
your system. On Linux it will look similar but the install
directory is something like {\tt /usr/local/lib/erlang/}.

Note! If you cut and past this text the '-character will not be a
'-character, if you see what I mean. When you load the file in emacs
you will have errors. If you do cut and paste then write in by hand
``'erlang-start''. This is true for all cut and paste from pfd
documents, things might not be what they appear to be so be careful.

\begin{verbatim}
  (setq load-path 
    (cons  
      "C:/Program Files/erl<Ver>/lib/tools-<ToolsVer>/emacs" 
      load-path))
  (setq erlang-root-dir 
      "C:/Program Files/erl<Ver>")
  (setq exec-path 
    (cons 
       "C:/Program Files/erl<Ver>/bin" 
       exec-path))
  (require 'erlang-start)
\end{verbatim}

\noindent You will have to find your emacs home directory where the file should
be placed. If you're on the KTH student computers then the home area
is ``h:'' but you then need to set the ``HOME'' environment variable
so that emacs finds its way. 

If everything works you should be able to start an Erlang shell inside
Emacs by {\tt M-x run-erlang} ({\tt M-x} is $<escape>$ followed by
x). A shell inside Emacs will allow you to quickly compile and run
programs but when we experiment with distributed applications you need
to run these in separate shells so make sure that you also know how
start an Erlang shell manually.

\subsection{Eclipse}

If you prefer to use Eclipse you can install a Erlang plugin and do
your development inside Eclipse. Fell free to use whatever
environment you want.

\section{Hello World}

Open a file {\tt hello.erl} and write the following:

\begin{verbatim}
-module(hello).

-export([world/0]).

world()->
  "Hello world!".
\end{verbatim}

\noindent Now open a Erlang shell,  compile and load the file with the command
{\tt c(hello).} and, call the function {\tt hello:world().}.  Remember
to end commands in th shell with a dot. If things work out you have
successfully written, compiled and executed your first Erlang program.

Find the Erlang documentation and read the ``Getting started''
section. 

\section{Concurrent Programming}

Erlang was designed for concurrent programming. You will quickly learn
how to divide your program into communicating processes and thereby
give it far better structure. Try the following:

\begin{verbatim}
-module(wait).
-export([hello/0]).

hello() ->
    receive
      X -> io:format("aaa! surprise, a message: ~s~n", [X])
    end.
\end{verbatim}

\noindent The {\tt io:format} procedure will output the string to the stdout and
replace the control characters (characters preceded by a tilde) with
the elements in the list. The {\tt ~s} means that the next element in
the list should be a string, {\tt ~n} simply outputs a newline. Load
the above module and execute the command:

\begin{verbatim}
   P = spawn(wait, hello, []).
\end{verbatim}

\noindent The variable {\tt P} is now bound to the {\em process identifier} of
spawned process. The process was created and called the procedure {\tt
  hello/0} (this is how we name a function with zero arguments). It is
now suspended waiting for incoming messages. In the same Erlang shell
execute the command:

\begin{verbatim}
   P ! "hello".
\end{verbatim}

\noindent This will send a message, in this case a string ``hello'', to the
process that now wakes up and continues the execution.

To make life easier one often register the process identifiers under
names that can be access by all processes. If two processes should
communicate they must know the process identifier. Either a process is
given the identifier to the other process when it is created, in a
message or, through the registered name of the process. Try this in a
shell (first type {\tt f()} to make the shell forget the previous
binding to P):

\begin{verbatim}
   P = spawn(wait, hello, []).
\end{verbatim}

\noindent Now register the process identifier under the name ``foo''.

\begin{verbatim}
   register(foo, P).
\end{verbatim}

\noindent And then send the process a message.

\begin{verbatim}
   foo ! "hello".
\end{verbatim}

\noindent In this example the only thing we sent was a string but we can send
arbitrary complex data structures. The {\tt receive} statement can
have several clauses that try to match incoming messages. Only if a
match is found will a clause be used. Try this:

\begin{verbatim}
-module(tic).
-export([first/0]).

first() ->
    receive
      {tic, X} -> 
          io:format("tic: ~w~n", [X]),
          second()
    end.

second() ->
    receive
      {tac, X} -> 
          io:format("tac: ~w~n", [X]),
          last();
      {toe, X} -> 
          io:format("toe: ~w~n", [X]),
          last()
    end.

last() ->
    receive 
      X ->
          io:format("end: ~w~n", [X])
    end.
\end{verbatim}


\noindent Then in a shell execute the following commands:

\begin{verbatim}

P = spawn(tic, first, []).

P ! {toe, bar}.

P ! {tac, gurka}.

P ! {tic, foo}.

\end{verbatim}

\noindent In what order where they received by the process. Note how messages
are queued and how the process selects in what order to process them.


\section{Distributed Programming}

Distributed programming is extremely easy in Erlang, the only problem
we will have is finding the name of our node. The Erlang distributed
systems normally work using domain names rather than explicit IP
addresses. This could be a problem since we're working with a set of
laptops that are not regularly given names in the DNS. There is a work
around for this that we will use.

Connect to the WLAN, login and make sure that you have a access to the
Internet. Run {\tt ipconfig} or {\tt ifconfig} to find out what IP address
you have been allocated. You will use this IP address explicitly when 
starting an Erlang node.

\subsection{node name}

When you start Erlang you can make it network aware by providing a
name. You also would like to give it a secret cookie. Any node that
can prove to have knowledge of the cookie will be trusted to do just
about anything. This is of course quite dangerous and it's very easy
for a malicious node to close a whole network down.

\noindent Start a new Erlang shell with the following command, replacing the IP
address to whatever you have.

\begin{verbatim}
  erl -name foo@130.237.250.69 -setcookie secret
\end{verbatim}

\noindent In the Erlang shell you can now find the name of your node with the
bif {\tt node()}. It should look something like {\tt 'foo@130.237.250.69'}. 

Doing the same if you're running Erlang under Emacs is slightly
more tricky. You have to set a lisp variable that is used when Emacs
starts Erlang. Type {\tt M-x eval-expression} and evaluate the function.

\begin{verbatim}
(set 'inferior-erlang-machine-options 
    '("-name" "foo@130.237.250.69" "-setcookie" "secret"))
\end{verbatim}

\noindent You can also set the environment variable {\tt ERL\_FLAGS} to the same
string or include it in your {\tt .emacs} file. This will however
prevent you from running multiple Erlang shells on the same node. You
would also have to change this every time you get a new IP address. 

Start a second Erlang shells now using the name {\tt bar} on the same
or another machine. Load and start the suspending hello process that
we defined before on the {\em foo-node} and register it under the name
{\tt wait}.

Now on the {\em bar-node} try the following:

\begin{verbatim}
   {wait, 'foo@130.237.250.69'} ! "a message from bar".
\end{verbatim}

\noindent Note how interprocess communication in Erlang is handled in the same
way regardless if the process is executing in the same shell, another
shell or on another host. The only thing that has to be changes is how
the Erlang shell is started and how to access external registered processes.

{\it If you run on a computer where you can not open ports for
  communication you will need to run Erlang in local distribution
  mode. You then start Erlang with a short name as follows:}

\begin{verbatim}
erl -sname foo -setcookie secret
\end{verbatim}

{\it The rest is the same but use foo as the name without the IP address.}

\subsection{ping-pong}

\noindent If we could only send messages to registered processes it
would not be a transparent system. We can send messages to any process
but the problem is of course to get hold of the process
identifier. There is no way to write this down and you can not use the
cryptic ``$<$0.70.0$>$'' that the Erlang shell uses when it prints the
value of a process identifier.

The only processes that know the process identifier of a process is
the creator of the process and the process itself. The process it self
can find out by the built-in procedure {\tt self()}. Now we are of
course allowed to pass the process identifiers around and even send
them in a message so we can let other processes know. Once a process
knows it can use the identifier and does not have to know if it is a
local or remote process.

Try to define a process on one machine {\tt ping}, that sends a
message to a registered process on another machine with its own
process identifier in the message. The process should the wait for a
reply. The receiver on the other machine should look at the message
and use the process identifier to send a reply.

A world of warning, the send primitive {\tt !} will accept both
registered names, remote names and process identifiers. This can
sometime cause a problem. If you implement a concurrent system that is
not distributed and have a registered processes under the name {\tt
foo}, you could pass the atom {\tt foo} around and anyone could treat
it as a process identifier. Now if we distribute this application a
remote process will not be able to use it as a process identifier
since the registration process is local to a node. So keep track of
what you pass around, is it a process identifier (that can be used
remotely) or is it a name under which a process is registered.


\end{document}
