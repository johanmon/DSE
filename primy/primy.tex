\documentclass[a4paper,11pt]{article}

\usepackage{ifthen}
\usepackage[latin1]{inputenc}


\newcommand{\nnsection}[1]{
\section*{#1}
\addcontentsline{toc}{section}{#1}
}

\begin{document}

\begin{center}
\vspace{20pt}
\textbf{\large Primy: finding a large prime}\\
\vspace{10pt}
\textbf{Johan Montelius}\\
\vspace{10pt}
\today{}
\end{center}

\nnsection{Introduction}

Your task will be to implement a distributed system that will find
large primes. The system should have one server that is in control of the
computation and a dynamic set of workers that are assigned numbers to
test for primarity.


\section{Finding a prime}

This is a test that we should use in a coming exercise. The task is to
test if a number is prime. A complete algorithm is quite expensive to
execute for large numbers so we use an algorithm that will detect if
number is a prime with very high accuracy. The algorithm is from
Fermat and to compute it we need a fast implementation of modular
exponentiation. Open up a new file {\tt fermat.erl} and declare the
module {\tt fermat}.


\begin{verbatim}	    
mpow(N, 1, _) ->
    N;
mpow(N, K, M) -> 
    mpow(K rem 2, N, K, M).

mpow(0, N, K, M) -> 
    X = mpow(N, K div 2, M),
    (X * X) rem M;
mpow(_, N, K, M) ->
    X = mpow(N, K - 1, M),
    (X * N) rem M.
\end{verbatim}	    

This algorithm will calculate $N^K \bmod M$ either by $N^{K/2}*N^{K/2}
\bmod M$ if $K$ is even, or by $N^{K-1}*N \bmod M$ if $K$ is odd. Modular
multiplication is nice since you can apply the modular operation on
both terms and the result will be the same. Try it and see if it works. 

Next we implement the test by Fermat. If a random number $R$, less than
$P$, raised to $P-1$ modulo $P$ is equal to $1$ e.g. if $R^{P-1}$ is
relative prime to $P$, then it is very likely that $P$ is prime.

\begin{verbatim}
fermat(1) ->
    ok;
fermat(P) ->
    R = random:uniform(P-1),
    T = mpow(R,P-1,P),
    if
       T == 1 ->
         ok;
       true ->
         no
    end.
\end{verbatim}	    

Note that it is only likely! We want to perform this test several times 
using different random numbers. 

\begin{verbatim}	    
test(_, 0) ->
    ok;
test(P, N) ->
    case fermat(P) of
     ok ->
       test(P, N-1);
     no ->
       no
     end.
\end{verbatim}	    

How many times do we have to perform the test, well it depends on how
many false primes you want to find. Note that most numbers are not
prime and fail the test in one try, if we stumble on a prime we might
as well run the test a couple of time since it will not slow down the
overall performance that much. 


That's it. Compile, load and do some experiments. Why not build a
prime generator, start with an integer and test if it's a prime. If
it's not, then move on to the next integer. Notice that Erlang is
implemented using a big-num packet and can handle integers of (almost)
arbitrary size. This is (probably) a prime that I found after some
searching:

\begin{verbatim}
    75654596987987976987
    68756756757657656987
    98789798798789796546
    54654564217541236547
    65421378512736521765
    73658765123765123786
    512378657852319179
\end{verbatim}

Now lets' implement network of nodes that collaborate in the search process.

\section{The server}

We should implement a server that
keeps track of the highest prime number found so-far and the number
next in turn to examine. The server should accept request from workers
and hand out numbers that should be examined. If the workers can
determine that it is a prime number it will return it to the server. 

We will not handle the situation were workers accept a number and then
dies nor workers that are malicious and report prime numbers without
proper checking.

\section{Speed it up}

There are two ways to speed up the process. One is to implement a more
efficient prime test function on the worker, another is to do part of
the prime checking on the server before delegating a number to a
worker. How much should be done at the server? We don't want the
server to spend it's time doing pre-checking of numbers if the workers
are idle. At the same time replying on a request takes time so we
should reduce the number of requests as much as possible.

What's the bottle neck in our system: the number of workers, the
server or the fact that we're communicating over a shared WLAN? Does
this change as numbers get larger?

\section{Making it robust}

How can we handle the situation with dying workers. Can we have
redundancy in the system so dead workers are detected and the assigned
number is given to another worker?

Can we handle malicious workers? Should we send all numbers to more
than one worker? Is it more important to find all numbers or making
sure that reported primes are actually prime?

\end{document}

