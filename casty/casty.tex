\documentclass[a4paper,11pt]{article}

\usepackage{ifthen}
\usepackage[latin1]{inputenc}

\newcommand{\nnsection}[1]{
\section*{#1}
\addcontentsline{toc}{section}{#1}
}

\begin{document}

\begin{center}
\vspace{20pt}
\textbf{\large Casty: a streaming media network}\\
\vspace{10pt}
\textbf{Johan Montelius}\\
\vspace{10pt}
\today{}
\end{center}

\nnsection{Introduction}

In this assignment you will build a streaming media network. We will
play around with shoutcast streams and build proxies, distributors and
peer-to-peer clients. You will use the Erlang bit-syntax to implement
a communication protocol over HTTP. The parser will be implemented
using higher order functions to hide the socket interface. You will
learn how to decode a mp3 audio stream and make it available for
connecting media players. Sounds fun? - Let's go!

\section{Shoutcast - ICY}

You first need to understand how shoutcast works. It's very simple
protocol for streaming audio content built using HTTP. It was at first
called ``I Can Yell'' and you therefore see a lot of {\tt icy} tags in
the HTTP header; we will also refer to it as the ICY protocol. 

\subsection{request - reply}

A media client such as Amarok (or VLC) connects to a server by sending a HTTP
GET request. In this request the client asks for a specific feed in the
same way as it would ask for a web page. In the request header it also
announce if it can handle meta-data, the name of the player etc. A
request could look like this:


\begin{verbatim}
GET / HTTP/1.0<cr><lf>
Host: mp3-vr-128.smgradio.com<cr><lf>
User-Agent: Casty<cr><lf>
Icy-MetaData: 1<cr><lf>
<cr><lf>
\end{verbatim}


The {\tt Icy-Metadata} header is important since it signals that our
client is capable of receiving meta data in an audio stream. To see that
this actually works you could use {\tt wget} and look at what for
example Virgin Radio returns when you try to connect. Try the
following command in a shell but terminate it with ctrl-c or it will
keep reading the audio stream.

\begin{verbatim}
wget --header="Icy-MetaData:1" -S -O reply.txt http://mp3-vr-128.smgradio.com
\end{verbatim}

Now look at the file (you did terminate the loading right)
reply.txt. Decode the reply and try to figure out what the different
header tag means. One reply header that is very important for us is the
{\tt icy-metaint} header. It typically has the value 8192 which is the
number of bytes in each mp3-block that is sent. Since the bit rate for
this audio stream is 128Kb/s a 8192 byte block will contain half a
second of music. Look at position 8192 in the body, do you find the
meta data?


\subsection{meta-data}

The meta data comes as a sequence of characters after each audio
block. The length of the sequence is coded in one integer $k$ in the
first byte of the meta data block. The length of the sequence is $16k$
bytes (not including the k-byte); the smallest meta data block is thus
simply one k-byte of zero. If the text message is not an even multiple
of $16$ it is padded with trailing zeros. In Erlang bit-syntax a meta
data segment could be written as follows:

\begin{verbatim}
<<1,104,
  101,108,
  108,111,
  0,0,0,0,
  0,0,0,0,
  0,0,0>>
\end{verbatim}

The first byte is a the length of the following padded string. The
next five bytes spell out ``hello'' and the padding consist of 11
zeros. This could also be written:

\begin{verbatim}
<<1,"hello", 0:(11*8)>>
\end{verbatim}


When you attach to a shoutcast radio station you will see that most
meta data blocks are empty. When a new song i played they use the
meta data to send the name of the artist, title etc. 

\subsection{encoding of messages}

One module, {\tt icy}, will implement all details of how the ICY
protocol is encoded and decoded. To make it more interesting we will
use higher order functions and make the module both able to handle
incomplete sequences and being unaware of how byte sequences are
actually read or written.

We will of course communicate using sockets but why build this into
the icy module; we pass a function along as an argument that the icy
procedures can use to send a binary encoded segment when it is ready.

\subsubsection{a request}

Sending a request is simple an we will do with this almost hard coded
version. The host is the name of the server we're contacting and the
feed is the resource, typically ``/''. 

\begin{verbatim}
send_request(Host, Feed, Sender) ->
    Request = "GET " ++ Feed ++" HTTP/1.0\r\n" ++ 
    "Host: " ++ Host ++ "\r\n" ++
    "User-Agent: Ecast\r\n" ++
    "Icy-MetaData: 1\r\n" ++ "\r\n",
    Sender(list_to_binary(Request)).
\end{verbatim}

The third argument is the function that we apply to the final
binary. It is up to the caller of {\tt send\_request/3} to provide a
function that does something useful with the binary. When using
socket communication we will call this procedure as follows:

\begin{verbatim}
Sender = fun(Bin) -> gen_tcp:send(Socket, Bin) end,
icy:send_request("mp3-vr-128.smgradio.com", "/", Sender)
\end{verbatim}

We could however use it like this if we're debugging our code:

\begin{verbatim}
Sender = fun(Bin) -> Bin end,
icy:send_request("mp3-vr-128.smgradio.com", "/", Sender)
\end{verbatim}

Or why not like this:

\begin{verbatim}
Sender = fun(Bin) -> io:format("Request:~n~s~n", [Bin]) end,
icy:send_request("mp3-vr-128.smgradio.com", "/", Sender)
\end{verbatim}

\subsubsection{a reply}

A complete ICY reply consist of a status line, a sequence of headers
and, a body. We divide the encoding into one procedure that encodes the
status line and headers, and one procedure that encodes a segment of
the body. Encoding the status line and headers is quite simple.

\begin{verbatim}
send_reply(Header, Sender) ->
    Status = "ICY 200 OK\r\n",
    Reply = Status ++ header_to_list(Header),
    Sender(list_to_binary(Reply)).
\end{verbatim}

We will represent headers by a list of tuples containing a header (an
atom) and a value (a string). A ICY header could look as follows:

\begin{verbatim}
[{'icy-notice', "This stream requires Winamp .."}, 
 {'icy-name', "Virgin Radio ..."}, 
 {'icy-genre', "Adult Pop Rock"}, 
 {'icy-url', "http://www.virginradio.co.uk/"}, 
 {'content-type', "audio/mpeg"},
 {'icy-pub', "1"}, 
 {'icy-metaint' "8192"}, 
 {'icy-br', "128"}]
\end{verbatim}

The encoding of the headers, implemented in {\tt header\_to\_list/1},
is done simply by turning the list of tuples into a byte sequence with
the name and value of each header separated by a colon and terminated
by a {\tt \textless cr\textgreater\textless lf\textgreater } (in Erlang
written as {\tt "\textbackslash r\textbackslash n" }. The whole header
sequence is terminated by an additional {\tt \textless
  cr\textgreater\textless lf\textgreater }. We can assume that the
headers are valid headers so we don't have to check every header. The
implementation is left as an exercise. The built-in function {\tt
  atom\_to\_list/1} will come at handy.

\begin{verbatim}
header_to_list([]) ->
     :
header_to_list([{Name, Arg}|Rest]) ->
      :
   ... ++ header_to_list(Rest).
\end{verbatim}


\subsubsection{the data segments}

A data segment will be represented by a tuple with audio data and a
(possibly empty) string that should be our meta data. The audio data
is, as will soon be clear, coded as a list of binaries. The total
length of all binaries must be equal to the {\tt metaint} header
information. We assume that this is the case and do not check this for
every segment.


\begin{verbatim}
send_data({Audio, Meta}, Sender) ->
    send_audio(Audio, Sender),
    send_meta(Meta, Sender).
\end{verbatim}

The implementation of {\tt send\_audio/2} is left as an
exercise. Sending the meta data is slightly more complicated. We need
to add the padding to the text string and calculate the k-value.

\begin{verbatim}
send_meta(Meta, Sender) ->
    {K, Padded} = padding(Meta),
    Sender(<<K/integer, Padded/binary>>).
\end{verbatim}

Implementing the {\tt padding/1} function is left as an exercise. You
will need the arithmetic constructs {\tt N rem 16} which will give you
the reminder when dividing with 16, and {\tt N div 16} that is the
integer division with 16. Calculate how large padding is needed,
and that number of zeros to the end of the sequence and turn it into a binary.

\begin{verbatim}
padding(Meta) ->
    N = length(Meta),
     :
     :
    end.
\end{verbatim}

This completes the encoding of messages, now for the slightly more
complex task of decoding. 

\subsection{sockets and continuations}

To understand the structure of our parser one must understand the
problem with reading from a stream socket. Reading from a socket is
very different from reading from a file. When reading from a socket we
could be stuck half-way through a structure since the rest of the
message has not yet arrived. In a concurrent system we do not want to
block a process suspending on a socket. 

In Erlang there is a solution to this problem. Instead of ``reading''
from the socket we have the socket send us segments as they arrive. We
can thus go into a receive statement and wait for either more segments
or any other message. A process that is receiving segments from socket
could then be programmed as follows:

\begin{verbatim}
reader(Socket, Sofar) ->
   receive
       {tcp, Socket, Next} ->
           reader(Socket, [Next|Sofar]);
       {tcp_closed, Socket} ->
           {done, lists:reverse(Sofar)};
       stop ->
           {aborted, Sofar}
   end.
\end{verbatim}
           

The process that calls {\tt reader/2} can now be aborted by sending a
stop message. A flexible solution, but do we have to build this into
the parser? Can we not simply read everything there is to read from
the socket and then pass everything to the parser? The problem is that
we do not always know (especially true for HTTP) how long the message
is unless we start to parse it; only by parsing can we determine if the
message is complete.

If we try to build this reading strategy into the parse the parser
would have to be aware of that it is reading from a socket and that it
must be open to receive other messages and not just the TCP related
messages. This would of course make the parser complicated and less
flexible. An alternative is to use ``continuations''; take a look at this.

The parser is given a, possibly not complete, segment to parse and
could return either:

\begin{itemize}

 \item {\tt \{ok, Parsed, Rest\}} : if the parsing was successful,
   Parsed is the result of the parser and Rest is what is left of the
   segment.

 \item {\tt \{more, Continuation\}} : if more segments are needed,
   Continuation is a function that should be applied to the next segment.

 \item {\tt \{error, Error\}} : if the parsing failed.

\end{itemize}

We then define a general purpose user of the parser that uses a zero
argument function. The function is applied without arguments and
could then result in a parsed result, a request for more or an error. 

\begin{verbatim}
reader(Parser, Socket) ->
    case Parser() of
        {ok, Parsed, Rest} ->
            {ok, Parsed, Rest};
        {more, Cont} ->
            receive
                {tcp, Socket, More} ->
                    reader(fun() -> Cont(More) end,  Socket);
                {tcp_closed, Socket} ->
                    {error, "server closed connection"};
                stop ->
                    aborted
            end;
        {error, Error} ->
            {error, Error}
    end.
\end{verbatim}

If more information is needed we go into the receive statement and
wait for the next segment. If we receive a new TCP block we construct
a new function and apply {\tt reader/2} recursively. To read and parse
a message from a socket we could call the reader as follows:

\begin{verbatim}
reader(Socket) ->
    reader(fun() -> parser(<<>>) end, Socket).
\end{verbatim}


We will not use the code above but this is the strategy that we will
use when implementing our ICY parser. The user of the parser will be
the client and proxy processes that we will define later.


\subsection{the parser}

So now let's look at the parser. We will need to parse two types of
messages: a request and a reply. The request will be sent by a media
client to one of our client proxies and the reply will be sent from
the server to our server proxy.  The reply, that consist of a status
line, headers and a body will, in the same way as the the encoding
procedures, be broken up into two parts. First we will parse the the
status line and headers and then we will parse the body. The body will
then be broken up into a sequence of data segments.

The parser will work on binaries; this will make it more efficient
when handling the larger audio data. 

\subsubsection{a request}

Parsing a request is quite simple, we read the first line (all lines
are terminated by \textless cr\textgreater \textless lf\textgreater)
and check that it is equal to ``GET / HTTP/1.0''. Now this is of
course a simplification; a request could of course include something
more interesting than ``/'' but it will do for now.

The {\tt line/2} function will look for the end-of-line characters but
if these are not found it will return {\tt more}. The request parser
will then return a continuation in the form of a function that should
be applied to the next segment. We could make this more complicated
but why not simply return a function that appends the segment that we
have to the next segment and tries to redo the parsing of the request.

\begin{verbatim}
request(Bin) ->
    case line(Bin) of
        {ok, "GET / HTTP/1.0", R1} ->
            case header(R1, []) of
                {ok, Header, R2} ->
                    {ok, Header, R2};
                more ->
                    {more, fun(More) -> request(<<Bin/binary, More/binary>>) end}
            end;
        {ok, Req, _} ->
            {error, "invalid request: " ++ Req};
        more ->
            {more, fun(More) -> request(<<Bin/binary, More/binary>>) end}
    end.
\end{verbatim}

Note how the scoping rules work; Erlang uses lexical scoping and the
variable {\tt Bin} will have it's binding when the function {\tt
  request/1} is called. The returned function could now be called in
any environment and the {\tt Bin} variable will maintain it's
binding. This is called static or lexical scoping. Some functional
programming languages use dynamic scoping where the value of {\tt Bin}
would depend on the environment in which it is used.

Once we have seen the request line we continue to parse the
headers. The parsing of the headers will either succeed or result in a
request of more information. If we succeed we return the Headers and
also what is left of the segment that we parsed. In practice the rest
will be an empty segment and probably ignored by the caller but why
not be polite and return what is left.

Note that we have made a simplification in the implementation of the
parse. If the function {\tt line/1} or {\tt header/2} has
stepped through a hundred characters only to find out that more
characters are needed, they simply return the atom {\tt more}. The
next time they are called they will have to run through the same
hundred characters again. It would of course be nice if we could save
exactly the position that we're in and continue only with the new
segment. This would however make the function {\tt request/1} more
complicated. We would have to keep track of if we should continue
reading a line or a header. In practice, the original binary will
always contain both the request line and all headers so the question
becomes academic.

Reading a line is a simple pattern matching exercise that is left as
an exercise. This is skeleton to work on: (\textless cr \textgreater
 could be written as {\tt 13} or {\tt \$\textbackslash r} and \textless
lf \textgreater as {\tt 10} or {\tt \$\textbackslash n})

\begin{verbatim}
line(Bin) ->
    line(Bin, ...).

line(..., _) ->
    ...;
line(<<..., ..., Rest/binary>>, Sofar) ->
    {ok, ..., ...};
line(<<..., Rest/binary>>, Sofar) ->
    line(..., ...).
\end{verbatim}

\subsubsection{a reply}

Parsing a reply is very similar. We apply the same strategy as before,
try to parse as much as possible and start from the beginning if you
need more. The thing that is slightly different is when we have
successfully parsed the headers. Now we return not only the headers
but also a continuation that when applied will generate the first data
segment. Since the decoding of the data segments needs information of
the length of the audio part we first extract this information from
the headers. The implementation of {\tt metaint} is left as an
exercise.


\begin{verbatim}
reply(Bin) ->
    case line(Bin) of
        {ok, "ICY 200 OK", R1} ->
            case header(R1, []) of
                {ok, Header, R2} ->
                    MetaInt = metaint(Header),
                    {ok, fun() -> data(R2, MetaInt) end, Header};
                 more ->
                    {more, fun(More) -> reply(<<Bin/binary, More/binary>>) end}
            end;
        {ok, Resp, _} ->
            {error, "invalid reply: " ++ Resp};
        more ->
            {more, fun(More) -> reply(<<Bin/binary, More/binary>>) end }
    end.
\end{verbatim}


Parsing a data segment consists of two parts. First we read $M$
number of bytes from the input stream and then we decode a meta data
section. We will stick to our continuation strategy and return either:

\begin{itemize} 

\item {\tt \{more, Continuation\}}: where Continuation is a function
  that should be applied to the next segment if more segments are
  needed or

\item {\tt \{ok, \{Audio, Meta\}, Continuation\}}: once a complete
  data segment has been read. The continuation will give us the next
  data segment when applied without arguments. 

\end{itemize}

It is not very likely that we will be able to read a whole audio
segment in one go. An audio segment is typically $8192$ bytes long and
TCP packets have a maximum size of 1460 bytes over regular
Ethernet. Each audio segment will typically consist of six
chunks. Since we do not want to decode those there is no point in
appending them into one binary. Instead we keep them in a
list. Eventually it is up to the {\tt send\_audio/2} procedure to send
these chunks one by one.

\begin{verbatim}
data(Bin, M) ->
    audio(Bin, [], M, M).

audio(Bin, Sofar, N, M) ->
    Size = size(Bin),
    if 
        Size >= N ->
            {Chunk, Rest} = split_binary(Bin,N),
            meta(Rest, lists:reverse([Chunk|Sofar]), M);
        true ->
            {more, fun(More) -> audio(More, [Bin|Sofar], N-Size, M) end}
    end.
\end{verbatim}

It's important to understand how we avoid parsing the same binaries
every time we request more information. We return a function that
should be applied to the next segment but we remember what we have
seen so far and how much more we need to see. This is different from
the implementation of {\tt request/1} where we simply started from the
beginning.

Parsing the meta data segment is slightly more tricky. We first have
to read the k-byte so that we know the length of the following
string. The string consist of a text padded with zeros to a multiple
of 16. We read the k-byte (could be zero), the following string and
remove the padding. We could have kept the padding since we will only
have a problem when we should send the segment to a proper client but
it could be nice to have a proper string as the meta data. 

\begin{verbatim}
meta(<<>>, Audio, M) ->
    {more, fun(More) -> meta(More, Audio, M) end};
meta(Bin, Audio, M) ->    
    <<K/integer, R0/binary>> = Bin,
    Size = size(R0),
    H = K*16,
    if
        Size >= H ->
            {Padded, R2} = split_binary(R0,H),
            Meta = [C || C <- binary_to_list(Padded), C > 0],
            {ok, {Audio, Meta},  fun() -> data(R2, M) end};
        true ->
            {more, fun(More) -> meta(<<Bin/binary, More/binary>>, Audio, M) end}
    end.
\end{verbatim}

You might not have seen the construct used to remove the padding. It's
called {\em list comprehension} and could be read - ``give me the list
of C's where C is taken from {\tt binary\_to\_list(Padded)} and C is greater
than 0''. 


\subsection{does it work}

Complete a module {\tt icy} that exports the above described
functions: {\tt request/1}, {\tt reply/1}, {\tt send\_request/1},
{\tt send\_reply/1} and {\tt send\_data/1}. You can then test your
implementation with the following test examples:

\begin{verbatim}
icy:send_request("www.host.com", "/", 
       fun(Bin) -> io:format("~s~n", [Bin]) end).
\end{verbatim}

\begin{verbatim}
icy:send_reply([{key, "value"}], 
       fun(Bin) -> io:format("~s~n", [Bin]) end).
\end{verbatim}

\begin{verbatim}
icy:send_data({[<<"hello">>], "Message"}, 
       fun(Bin) -> io:format("~w~n", [Bin]) end).
\end{verbatim}

Experimenting with the parser is equally simple. It's operating on
binaries but the binary syntax makes it very easy to construct the
segments that we need.

\begin{verbatim}
icy:request(<<"GET / HTTP/1.0\r\nkey:value\r\n\r\n">>).
\end{verbatim}

Let's try parsing something incomplete.

\begin{verbatim}
{more, F} = icy:request(<<"GET / HTTP/1.0\r\nkey:value\r\n">>).
\end{verbatim}

\begin{verbatim}
F(<<"\r\n">>).
\end{verbatim}

A reply will always give us a function that should be applied
without arguments to give us the data segments.

\begin{verbatim}
{ok, Data, H} = icy:reply(<<"ICY 200 OK\r\nicy-metaint: 5\r\n\r\n123">>).
\end{verbatim}

Since the body did not contain 5 audio byte nor a meta data section we
should get a request for more if we apply the continuation.

\begin{verbatim}
{more, More} = Data().
\end{verbatim}

Now let's apply this continuation on the rest of the section. The
``1'' indicates a total of 16 bytes. The message ``hello'' is then
padded with 11 bytes of zeros (an integer 0 encoded in 11*8 bits).

\begin{verbatim}
More(<<"45", 1, "hello", 0:(11*8)>>).
\end{verbatim}


\section{The architecture}

Our goal is now to build a proxy (server proxy), that is connected to
a shoutcast server, and a client (client proxy) that accepts
connections from a media player. The client should know about the
proxy and communicate with it using Erlang messages. We will use the
following messages between the client and the proxy.

\begin{itemize}
\item {\tt \{request, Client\}}: where the Client is the Erlang
  process that wants to connect.

\item {\tt \{reply, N, Context\}}: where Context is the header
  information received from the source. N is the number on the data
  segment that should arrive next.

\item {\tt \{data, N, Data\}}: where N is an integer to number all
  segments and Data is the audio and meta data that we have received. 
\end{itemize}

The numbering of data segments is to keep track of which segments we
need once we start to build more complicated distribution networks. In
the beginning the FIFO-order of Erlang messaging will give us the
messages in the right order.

\subsection{the client}

The client will listen on a TCP port and wait for incoming
connection. Once a connection is accepted a request is read from the
socket. The content of the request is not important for our needs, we
will happily connect any media client to a predefined proxy. 

\begin{verbatim}

init(Proxy, Port) ->
    {ok, Listen} = gen_tcp:listen(Port, ?Opt),
    {ok, Socket} = gen_tcp:accept(Listen),
    case read_request(Socket) of
        {ok, _, _} ->
            case connect(Proxy) of
                {ok, N, Context} ->
                    send_reply(Context, Socket),
                    {ok, Msg} = loop(N, Socket),
                    io:format("client: terminating ~s~n", [Msg]);
                {error, Error} ->
                    io:format("client: ~s~n", [Error])
            end;
        {error, Error} ->
            io:format("client: ~s~n", [Error])
    end.
\end{verbatim}

When creating the listen socket we specify the properties in a options
list. The options we will use are:

\begin{itemize}

\item {\tt binary}: more efficient since there is no point in handling
  the mp3 audio as list of integers.

\item {\tt \{packet, 0\}}: framing messages with the length of the
  message is very useful but we're dealing with the ICY protocol and not our own.

\item {\tt \{reuseaddr, true\}}: allow us to reuse the port (and we will) 

\item {\tt \{active, true\}}: the socket process will send us segments
  as they arrive, we do not have to suspend reading the socket.

\item {\tt \{nodelay, true\}}: send segments as soon as possible

\end{itemize}

Notice how we here choose to have a time-out when waiting for TCP
messages. We could also have chosen to accept a {\tt stop} message or
similar; the client module is in control.

Once we have read a proper request (we don't really care what is
requested) we try to connect to the proxy. The proxy will reply with a
context (the header information sent by the server) that we can send
to the client and then go into a loop that continuously deliver data
segments from the proxy to the media player.


\begin{verbatim}
connect(Proxy) ->
    Proxy ! {request, self()},
    receive 
        {reply, N, Context} ->
            {ok, N, Context}
    after ?TimeOut ->
            {error, "time out"}
    end.
\end{verbatim}

The loop is simple and will continue to deliver data segments as long
as the TCP connection to the media player is not closed nor a time-out
occurs. At this point we do not check that we actually receive all
data segments but this could of course easily be done. 

\begin{verbatim}
loop(_, Socket) ->
    receive
        {data, N, Data} ->
            send_data(Data, Socket),
            loop(N+1, Socket);
        {tcp_closed, Socket} ->
            {ok, "player closed connection"}
    after ?TimeOut ->
            {ok, "time out"}
    end.
\end{verbatim}

This is the interface that we use to the icy module. Sending is quite
simple, we use the exported procedures and supply a tcp send function
that should be used to send the constructed binaries over the socket
interface.

\begin{verbatim}
send_data(Data, Socket) ->
    icy:send_data(Data, fun(Bin)-> gen_tcp:send(Socket, Bin) end).

send_reply(Context, Socket) ->
    icy:send_reply(Context, fun(Bin)-> gen_tcp:send(Socket, Bin) end).
\end{verbatim}

The interface to the parser uses a general purpose reader that will
will return {\tt \{ok, Parsed, Rest\}} or {\tt \{error, Error\}}. It
is given a zero argument function and a socket as input arguments. The
function is applied and results either in a successful parsing, a
request for more or an error message. The request for more input can
now be handled outside of the parser module. The reader will go into a
receive statement and wait for more TCP messages. We have a time-out
so that we do not get stuck waiting for segments that will never come.

\begin{verbatim}
reader(Cont, Socket) ->
    case Cont() of
        {ok, Parsed, Rest} ->
            {ok, Parsed, Rest};
        {more, Fun} ->
            receive
                {tcp, Socket, More} ->
                    reader(fun() -> Fun(More) end,  Socket);
                {tcp_closed, Socket} ->
                    {error, "server closed connection"}
            after ?TimeOut ->
                    {error, "time out"}
            end;
        {error, Error} ->
            {error, Error}
    end.
\end{verbatim}

Reading a request can now be defines like this:

\begin{verbatim}    
read_request(Socket) ->
    reader(fun()-> icy:request(<<>>) end, Socket).
\end{verbatim}

Implement the module {\tt client} and export the procedure {\tt
  init/2}.  That is the whole client process. Now let's turn to the
proxy.


\subsection{the proxy}

The proxy is even simpler to implement. When we start the proxy we
will give it a shoutcast stream to connect to. The stream is defined
by a tuple:

\begin{itemize}
\item {\tt \{cast, Host, Port, Feed\}}.
\end{itemize}

The source of Virgin Radio would be: 

\begin{itemize}
\item {\tt \{cast, "mp3-vr-128.smgradio.com", 80, "/"\}}.
\end{itemize}

We will first wait for a client to request a connection before
attaching to the shoutcast server. Since we will not hear anything
from the client we add a monitor. If the client terminates we will
receive a message and can then decide what to do (die is one
option). This is not strictly necessary but it will reduce the number
of zombie proxies when experimenting.

\begin{verbatim}
init(Cast) -> 
    receive 
        {request, Client} ->
            io:format("proxy: received request ~w~n", [Client]),
            Ref = erlang:monitor(process, Client),
            case attach(Cast, Ref) of 
                {ok, Stream, Cont, Context} -> 
                    io:format("proxy: attached ~n", []),
                    Client ! {reply, 0, Context}, 
                    {ok, Msg} = loop(Cont, 0, Stream, Client, Ref),
                    io:format("proxy: terminating ~s~n", [Msg]);
                {error, Error} -> 
                    io:format("proxy: error ~s~n", [Error])
            end
    end.
\end{verbatim}

A connection to a server consist of: a stream in the form of a open
socket, a continuation, from which we can receive data segments, and a
context (the header information). Once attached we send a reply and
start relaying data segments.

\begin{verbatim}
loop(Cont, N, Stream, Client, Ref) ->
    case reader(Cont, Stream, Ref) of
        {ok, Data, Rest} ->
            Client ! {data, N, Data},
            loop(Rest, N+1, Stream, Client, Ref);
        {error, Error} ->
            {ok, Error}
    end.
\end{verbatim}

Attaching to a server requires some socket programming. We connect to
the server and send an ICY request If the sending is successful we continue with reading the
reply that either results in a valid connection or and error message.


\begin{verbatim}
attach({cast, Host, Port, Feed}, Ref) ->
    case gen_tcp:connect(Host, Port, [binary, {packet, 0}]) of
        {ok, Stream} ->
            case send_request(Host, Feed) of
                ok ->
                    case reply(Stream, Ref) of
                        {ok, Cont, Context} ->
                            {ok, Stream, Cont, Context};
                        {error, Error} ->
                            {error, Error}
                    end;
                _ ->
                    {error, "unable to send request"}
            end;
        _ ->
            {error, "unable to connect to server"}
    end.
\end{verbatim}

When sending the request we supply the {\tt gen\_tcp:send/2} function
to the {\tt icy:send\_request/3} function.

\begin{verbatim}
send_request(Host, Feed) ->
     icy:send_request(Host, Feed, fun(Bin) -> gen_tcp:send(Stream, Bin) end).
\end{verbatim}

The {\tt reader/3} function is slightly different from the one we used
for the client. We now take advantage of the fact that we can act on
other messages besides the once from the tcp-process. A {\tt DOWN}
message is sent if the monitored client process dies or becomes
unavailable. We could of course suspend and wait for a new client
connection but as you will we might as well die.


\begin{verbatim}
reader(Cont, Stream, Ref) ->
    case Cont() of
        {ok, Parsed, Rest} ->
            {ok, Parsed, Rest};
        {more, Fun} ->
            receive
                {tcp, Stream, More} ->
                    reader(fun() -> Fun(More) end,  Stream, Ref);
                {tcp_closed, Stream} ->
                    {error, "icy server closed connection"};            
                {'DOWN', Ref, process, _, _}  ->
                    {error, "client died"}
            after ?TimeOut ->
                    {error, "time out"}
            end;
        {error, Error} ->
            {error, Error}
    end.
\end{verbatim}

Reading a reply can now be coded like this.

\begin{verbatim}
reply(Stream, Ref) ->
    reader(fun()-> icy:reply(<<>>) end,  Stream, Ref).
\end{verbatim}

Implement the {\tt proxy} module and export the {\tt init/1}
function. If all is well we should now be able to connect a media
player to a client, a client to a proxy and a proxy to a server.


\subsection{streaming audio}

We now have all the pieces of the puzzle to start streaming audio
across our network. Create a {\tt test} module and write some
functions that we will use in our experiments. First we will create a
proxy and client process in the same Erlang node and see that things
work, then we will have the two processes running on two computers.

\begin{verbatim}
direct() ->
    Proxy = spawn(proxy, init, [?Cast]),
    spawn(client, init, [Proxy, ?Port]).
\end{verbatim}

Start the processes and then direct you media client (Amarok or VLC
should work) to the stream {\tt http://localhost:8080/} (or whatever
port you're using). If you have added some trace print statement you
will see how the client accepts the request from the player, sends it
to the proxy that connects to the read server. Every data segment will
contain half a second of audio information. Note how the media player
decodes the meta data and uses it to describe the channel.

\begin{indent}
We have two loop constructs one on the proxy side and one on the
client side. The recursive call on the proxy is a call to {\tt
  stream/5}, now if this this was a call to {\tt proxy:stream/5} we
would use the latest loaded version of the function in every
recursion. Try this - change the loop to using {\tt proxy:stream/5},
compile load and start playing, edit the source and include a printout
every loop, compile and load the module while still playing. See how
we can update our code while not loosing a single audio packet.
\end{indent}

If everything works it's time to start a proxy on one computer and let
the client run on another. You can even let the media player run on a
third computer. When running several Erlang node make sure that you
start them with a proper name and the same cookie.

\begin{verbatim}
erl -name 'proxy@130.237.215.255' -setcookie C00l3r
\end{verbatim}

Check the processor load on the machines. We're handling a 128Kbps
audio stream, does it show? What happens of we remove the cables, can
we survive one or two seconds of network failures? Drop every tenth
packet and see if things still work. Are processes terminated as
expected when the media player stops playing? Do some experiments
before going further.

\section{A distribution server}

So now we have solved the tricky parts of communicating with a media
player and a Shoutcast server. Handling the messages in Erlang is a lot
simpler, the only messages that we need to keep track of are:

\begin{itemize}
\item {\tt \{request, Client\}}: a request from a client
\item {\tt \{reply, N, Context\}}: the reply to a request, expect N to be
  the next data packet
\item {\tt \{data, N, Data\}}: only have to look at N if we want to
\end{itemize}

Our current system is limited since we only allow one client to
connect each proxy. We could extend the proxy to handle more clients
but we could also implement this as a separate process. This is the
skeleton code for a a module {\tt dist} that will do just this.

\begin{verbatim}
init(Proxy) ->
    Proxy ! ...
    receive 
        ...  ->
            :

    after ?TimeOut ->
            ok
    end.


loop(Clients, N, Context) ->
    receive 
        {data, N, Data} ->
             :
            loop(Clients, N+1, Context);
        {request, From} ->
            Ref = erlang:monitor(process, From),
            From ! ... ,
            loop(... , N, Context);
        {'DOWN', Ref, process, Pid, _} ->
            loop(... , N, Context);
        stop ->
            {ok, "stoped"};
        stat ->
            io:format("dist: serving  clients~n", [length(Clients)]),
            loop(Clients, N, Context)
    end.
\end{verbatim}

Complete the missing parts connect a distribution process to a
proxy. Then start clients one by one and make them connect to the
distributor. Since the distributor needs a media player before it
connects it could be useful with a dummy client.


\begin{verbatim}
init(Proxy) ->
    Proxy ! {request, self()},
    receive 
        {reply, N, _Context} ->
            io:format("dummy: connected~n", []),
            {ok, Msg} = loop(N),
            io:format("dummy: ~s~n", [Msg]),
    after 5000 ->
            io:format("dummy: time-out~n", [])
    end.

loop(N) ->
    receive 
        {data, N, _} ->
            loop(N+1);
        {data, E, _} ->
            io:format("dummy: received ~w, expected ~w~n", [E,N]),
            loop(E+1);
        stop ->
            {ok, "stoped"}
    after ?TimeOut ->
            {ok, "time out"}
    end.
\end{verbatim}


How many dummy client can you run before your machine chokes? If
you're sitting on anything close to what I'm sitting at don't spend to
much time starting dummy clients one by one. Write a function that
starts $n$ clients and then see how many you can start. Also try to
run the distributor on one machine and the dummy clients on other
machines - is it the processing power or the network that is the
limiting factor?

\section{Build a tree}

Let's try to build a distribution tree dynamically as clients want to
connect. We will use two new processes that are so similar that we'll
implement them in the same module. On is a {\tt root} process that
will connect to a proxy and then wait for branches to connect. It will
only allow two branches to connect and will redirect other branches to
the two that it has connected.

The other process is a {\tt branch} process. It will wait for a client
to connect and then connect to a {\tt root}. It must be prepared to be
redirected to another branch who it should then try to connect
to. Once connected it must also be open to serve two other branches
and redirect others to these two branches. 

We will now have six elements in our shoutcast architecture. A media
player is connected to a client proxy. The client proxy thinks it is
connecting to a proxy but it is actually connecting to a branch
process. The branch process know that it is connecting to a root (or
another branch) and the root is connected to a server proxy. The
server proxy is as before connected to the real shoutcast server.

In a real implementation one would probably collapse the proxy and root
process and the client and branch process but note how the separation
of the processes makes each description more easy to follow. In a true
concurrent language creating and running a separate process is as
normal as handling complexity with procedures and libraries. In the
same way, as we trade efficiency for clarity when hiding
implementation details in a module, we use processes to make our
system easier to implement.

\subsection{the messages}

In the final project the goal is to connect all computers in the class
in a distribution tree. We will have one dedicated node that runs the
proxy and root process. All other nodes will run a media player, a
client process and a branch process. Since we're now implementing the
tree module independently and a branch process on one node will not
run the same code as a branch process on another node, it is
important to specify the message interface.

\begin{itemize}

\item {\tt \{request, Pid\}}: a request sent to a root or a branch
  process. The process (Pid) must be able to handle request messages
  once connected.


\item {\tt \{reply, N, Context\}}: a reply from a root or a branch
  process. The integer {\tt N} is the number of the next data segment,
  {\tt Context} should be handled by the {\tt icy} module.

\item{\tt \{redirect, Pid\}}: this is the message given as a reply to a
  request message when the root or branch can not connect more
  branches. The {\tt Pid} is a process identifier to another branch
  process that might have a available slot or that will redirect us
  again.

\item {\tt \{data, N, Data\}}: the n'th data segment, the data itself
  should be handled by the {\tt icy} module.

\end{itemize}

If we all stick to these messages things might actually work on the first try.

\subsection{the root}

You need your own root to do some initial experiments. This is a
skeleton that you can easily complete. The first procedure will
connect to the proxy.

\begin{verbatim}
root(Proxy) ->
    Proxy ! ..... ,
    receive 
        ..... ->
            loop(... , ... , ...., Context)
    after ?TimeOut ->
            ok
    end.
\end{verbatim}

The loop procedure will accept data messages from the proxy or request
messages from branches processes that try to connect. We will accept
the two first branch processes but redirect all other.

\begin{verbatim}
loop(Clients, N,  Context) ->
    receive 
        {data, N, Data} ->
              :
            loop(Clients, N+1, Context);

        {request, From} ->
            L = length(Clients),
            if 
                L < 2 ->
                    From ! ..... ,
                    loop([From|Clients], N+1, Context);
                true ->
                      :
                    From ! {redirect, ....} ,
                    loop(...,  .... , ...)
            end

    end.
\end{verbatim}

Note that our tree will look less like a tree if we always redirect
requesting processes down the left branch. To keep the tree
balanced we should redirect every second process to the left and every
second to the right.


\subsection{the branch}

The branch process will look very similar to the root process. The
difference is that the branch process should first wait for a client
process to connect. When it receives a request from a client it should
try to connect to the known root of the tree. It could be redirected
several times but it should finally be connected.

Once connected it should forward all data packets to its own client. It
should also be open for requests from other branches. Similar to the
root process it should accept the two first branches and redirect the rest.

Note - a branch process must separate the client process from connected
branches. Al should receive copies of the data segments but we can not
redirect a connecting branch to the client; a client is not prepared
to handle request messages.

\subsection{error handling}

Can we make this structure more stable or self repairing. Can we detect
that our up-stream source is not delivering as it should? If the root
process dies then there is not much to do but if it was a peer branch
process we could try a new attempt at connecting to the root. Do we
have time to do this before any of the branches connected to us will
have time to find out? 

If we are directly connected to the root we should expect to have
data segments delivered every 500 ms. Should we give the root some
slack and set a time out after 600ms? What happens if all branches
below us has the same time out? 

Can we use Erlang monitors to detect that nodes are down or do we have
to do our own failure detection? 

Assume our up-stream source fails to deliver and we manage to
reconnect. If the old up-stream source now resumes transmission and
delivers the data segments we will have two processes delivering the
same stream. How do we prevent this? Can we introduce a control
message to stop a transmission?

\section{A BitTorrent architecture}

How hard would it be to implement distribution network using a
bitTorrent protocol? What are the problems that we need to solve?
Would it be better than our tree distribution network? Pros and cons?

\end{document}



