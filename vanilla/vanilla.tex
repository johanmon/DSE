\documentclass[a4paper,11pt]{article}

\usepackage{ifthen}
\usepackage[latin1]{inputenc}

\newcommand{\nnsection}[1]{
\section*{#1}
\addcontentsline{toc}{section}{#1}
}

\begin{document}

\begin{center}
\vspace{20pt}
\textbf{\large Groupy: a group membership service}\\
\vspace{10pt}
\textbf{Johan Montelius}\\
\vspace{10pt}
\today{}
\end{center}

\nnsection{Introduction}

This is an assignment were you will implement a group membership
service that provides atomic multicast. The aim is to have several
application layer processes with a coordinated state i.e. they should all
perform the same sequence of state changes. A node that wish to
perform a state change must first multicast the change to the group
so that all nodes can execute it. Since the multicast layer provides
total order, all nodes will be synchronized.

The problem in this assignment is that all nodes need to be
synchronized even though nodes may come and go (crash). As you will
see it is not as trivial as one might first think.


\section{The architecture}

We will implement a group membership service that provides atomic
multicast in view synchrony. The architecture consist of a set of nodes
where one is the elected leader. All nodes that which to multicast a
message will send the message to the leader and the leader will so a
basic multicast to all members of the group. If the leader dies a new
leader is elected.

A new node that wish to enter the group will contact any node in the
group and request to be join the group. The leader will determine
when the node is to be included and deliver a new view to the
group. 

Each application layer process will have its own group process that it
communicates with. The application layer will send multicast messages to the
group process and receive all multicasted messages from it. The application
layer must also be prepared to decide if a new node should be allowed
to enter the group and also decide the initial state of this node. 

Note that we will not deliver any views to the application layer. We
could adapt the system so that it did report any view changes but for
the application that we are targeting this is not needed. We will keep
it as simple as possible and then discuss extensions and how much they
would cost.

\subsection{view synchrony}

Each node in the group should be able to multicast messages to the
members of the group. The communication is divided into views and
messages will be said to be delivered in a view. For all messages in a
view we will guarantee the following:

\begin{itemize}
\item in FIFO order: in the order that they were sent by a node
\item in total order: all nodes see the same sequence
\item reliably: if a correct node delivers a message, all correct nodes delivers the message
\end{itemize}

delivered The last statement seams to be a bit weak, what do we mean
by a correct node? A node will fail only by crashing and will then
never be hear from again. A correct node is a node that does not fail
during a view i.e. it survives to install the next view.

It will not be guaranteed that sent messages are delivered, we will
use asynchronous sending without acknowledge and if we have a
failing leader a sent message might disappear.

\subsection{the leader}

A node will either play the role of a leader (let's hope there is only
one) or a slave. All slaves will forward messages to the leader and
the leader will tag each message with a sequence number and multicast
it to all nodes. The leader can also accept a message directly from
its own master i.e. the application layer. The application layer is
unaware of if its group process is acting as a leader or a slave.


\subsection{a slave}

A slave will receive messages from its application layer process and
forward them to the leader. It will also receive messages from the
leader and forward them to the application layer. If nodes could not
fail this would be the easiest job in the world but since we must be
able to act if the leader dies we need to do some book keeping.

In our first version of the implementation we will not deal with
failures but only adding new nodes to the system. This is complicated
enough to start with.

\subsection{the election}

The election procedure is very simple. All slaves have the same list
of peers and they all elect the first node in the list as the
leader. A slaves that detects that it is the first node will of course
adopt the role as leader. 

The leader will have to resend the last message that it received and
the slaves will have to monitor the leader. Note that we do not send
out a new view to the members of the group. We could have don this but
it is not needed in the system that we are building.

\subsection{the application layer}

An application process will create a group process and contact any
other application process it knows of. It will request to join the
group providing the process identifier of its group process. It will
then wait for a view to be delivered from its group process containing
the state that it should adopt.

There is no guarantee that the request is delivered to the leader, or
rather the leader could be dead and we have not detected this yet. The
requesting application process is however not told about this so we
can not do anything but wait and hope for the best. We will use a time
out and if we have not been invited in a second we might as well abort
the attempt.

\section{the first implementation}

Our first version, called {\tt gms1}, will only handle starting of a
single node and adding of more nodes. Failures will not be handled so
some of the states that we need to keep track of is not described. We
will then extend this implementation to handle failures.

The group process will when started be a slave but might in the future
become a leader. The first process that is started will however become a
leader directly.

\subsection{the leader process}

The leader process can be in two states, the regular state where it is
multicasting messages or a state where it has asked the application
layer if a new node can enter the group. The leader keeps the
following state:

\begin{itemize}
\item Id: a unique name of the node, only used for debugging
\item Master: the process identifier of the application layer
\item Peers: an ordered list of the process identifiers of all slaves in the group
\end{itemize}

The list of peers is ordered based on when they were admitted to the
group. We will use this order in the election procedure.

In its regular state the leader should be able to handle the following
messages:

\begin{itemize}
\item {\tt\{mcast, Msg\}}: a message either from its own master of
  from a peer node. A message {\tt\{msg, Msg\}} is multicasted to all
  peers and a message {\tt\{deliver, Msg\}} is sent to the application
  layer.
\item {\tt \{join, Peer\}}: a message, from a peer or the master, that
  is a request from a node to join the group. A {\tt request} message
  is sent to the application layer and the leader moves into the
  joining state.
\end{itemize}

The regular state of a leader is implemented by the following
procedure. We use a function {\tt bcast/3} that will send a message to
each of the processes in a list.

\begin{verbatim}
leader(Id, Master, Peers) ->    
    receive
        {mcast, Msg} ->
            bcast(Id, {msg, Msg}, Peers),
            Master ! {deliver, Msg},
            leader(Id, Master, Peers);
        {join, Peer} ->
            Master ! request,
            joining(Id, Master, Peer, Peers);
        stop ->
            ok
    end.
\end{verbatim}

In the joining state the leader can only accept one message:

\begin{itemize}
\item {\tt \{ok, State\}}: a message from the master deciding that the
  peer node should join the group and that it should start with a
  specified state. The leader returns to its regular state.
\end{itemize}

We ignore the possibility that a node is refused to enter the group
but this is of course a simple extension.

The joining of a new node is not a asynchronous process. The leader
has to stop delivering messages until it has got the reply from the
application layer. If this was not the case we could have a situation
where the application layer accepts the new peer but the state that it
is given is already old.


\begin{verbatim}
joining(Id, Master, Peer, Peers) ->
    receive 
        {ok, State} ->
            Peers2 = lists:append(Peers, [Peer]),           
            bcast(Id, {view, State, self(), Peers2}, Peers2),
            leader(Id, Master, Peers2);
        stop ->
            ok
    end.
\end{verbatim}

Notice that we add the new node at the end of the list of peers. This
is important, we want the new node to be the last one to see the view
message that we send out. More on this later when we look at failing nodes.

\subsection{a slave}

A slave has an even simpler job, it will not make any complicated
decisions. It is simply forwarding messages from its master to the
leader and vice verse. The state of a slave is as follows:

\begin{itemize}
\item Id: a unique name of the node, only used for debugging
\item Master: the process identifier of the application layer
\item Leader: the process identifier of the leader
\item Peers: an ordered list of the process identifiers of all slaves in the group
\end{itemize}

The messages from the master are the following:

\begin{itemize}
\item {\tt \{mcast, Msg\}}: a request from its master to multicast a
  message, the message is forwarded to the leader.
\item {\tt \{join, Peer\}}: a request from its master to allow a new
  node to join the group, the message is forwarded to the leader.
\item {\tt \{msg, Msg\}}: a multicasted message from the leader. A
  message {\tt \{deliver, Msg\}} is sent to the master.
\item {\tt \{view, State, Leader, View\}}: a multicasted view from the
  leader. For the slave only the new set of peers is of interest.
\end{itemize}

This is the implementation of the slave:


\begin{verbatim}
slave(Id, Master, Leader, Peers) ->    
    receive
        {mcast, Msg} ->
            Leader ! {mcast, Msg},
            slave(Id, Master, Leader, Peers);
        {join, Peer} ->
            Leader ! {join, Peer},
            slave(Id, Master, Leader, Peers);
        {msg, Msg} ->
            Master ! {deliver, Msg},
            slave(Id, Master, Leader, Peers);
        {view, _, _, View} ->
            slave(Id, Master, Leader, N+1, View);           
        stop ->
            ok
    end.
\end{verbatim}

Since we will not yet deal with failure there is no transition between
being a slave and becoming a leader. We will add this later but first
let us have this thing up and running.


\subsection{initialization}

Initializing a process that is the first node in a group is
simple. The only thing we need to do is to give it an empty list of
peers. Since it is the only node in the group it will of course be the
leader of the group.

\begin{verbatim}
start(Id) ->
    Self = self(),
    spawn_link(fun()-> init(Id, Self) end).

init(Id, Master) ->
    leader(Id, Master, []).
\end{verbatim}

Starting a node that should join an existing group is only slightly
more problematic. We need to send a {\tt\{join, self()\}} message to a
node in the group and wait for an invitation. The invitation is
delivered as a view message containing everything we need to
know. The initial state is of course as a slave.

\begin{verbatim}
start(Id, Grp) ->
    Self = self(),
    spawn_link(fun()-> init(Id, Grp, Self) end).    

init(Id, Grp, Master) ->
    Self = self(), 
    Grp ! {join, Self},
    receive
        {view, State, Leader, Peers} ->
            Master ! {ok, State},
            slave(Id, Master, Leader, Peers)
    end.
\end{verbatim}

Note that the view that is delivered contains the state that should be
passed on to the application layer. 

\subsection{the application process}

To do some experiment we create worker that uses a gui to describe its
state. A worker and gui is given in the end of this paper. Compile the
system and do some experiments to see that you can create a group and
add some peers. 


\section{handling failure}

We will build up our fault tolerance gradually. First we will make sure
that we detect crashes, then to make sure that a new leader is elected
an then make sure that the layer preserves the properties of the atomic
multicast. Keep {\tt gms1} as a reference and call the adapted module
{\tt gms2}.

\subsection{failure detectors}

We will use the Erlang built in support to detect and report that
processes have crashed. A process can monitor another node an if that
nodes dies a message will be received. For now we will assume that the
monitors a re perfect i.e. they will eventually report the crash of a
node and they will never report the death of anode that has not died.

We will also assume that the message that inform a process about a
death of a process is the last message that it will see from the
node. The message will thus be received in FIFO order as any regular
message.

The question we first need to answer is, who should monitor who? In
our architecture we need not report new views when a slave dies and
there is nothing to prevent a dead slave to be part of a view so we
will keep things simple; the only node that will be monitored is the
leader.A slave that detects that a leader has died will move to an
election state. 

This is implemented by first adding a call to {\tt erlang:monitor/2}
in the initialization of the slave:

\begin{verbatim}
            erlang:monitor(process, Leader)
\end{verbatim}

and a new clause in the state of the slave:

\begin{verbatim}
        {'DOWN', _Ref, process, Leader, _Reason} ->
            election(Id, Master, N, Last, Peers);
\end{verbatim}


In the election state the process will select the first node in its
lists of peers and elect this as the leader. It could of course be
that the process finds itself being the first node and it will thus
become the leader of the group.

\begin{verbatim}
election(Id, Master, [Leader|Rest]) ->
    if 
        Leader == self() ->
            leader(Id, Master, Rest);
        true ->
            erlang:monitor(process, Leader),
            slave(Id, Master, Leader, Rest)
    end.
\end{verbatim}

Since the leader can crash it could be that a node that wants to join
the group will never receive a reply. The message could be forwarded
to a dead leader and the joining node is never informed of the fact
that its request was lost. We simply add a timeout when waiting for an
invitation to join the group.

\begin{verbatim}
    after 1000 ->
            Master ! {error, "no reply from leader"}
\end{verbatim}

That is it we can now both add new nodes to the system and survive
even if nodes crash. That was not that hard was it? Do some
experiments to see that it works and then ship the product.

\subsection{missing messages}

Is seams to be too easy and unfortunately it is. To show that it is
not working we can change the {\tt bcast/3} procedure and introduce a
random crash. We define a constant {\tt arghh} that defines the risk
of crashing. A value of {\tt 100} means that a process will crash in
average once in a hundred attempts. The definition of {\tt bcast/3}
now looks like this:

\begin{verbatim}
bcast(Id, Msg, Nodes) ->
    lists:foreach(fun(Node) -> Node ! Msg, crash(Id) end, Nodes).

crash(Id) ->
    case random:uniform(?arghh) of
        ?arghh ->
            io:format("leader ~w: crash~n", [Id]),
            exit(no_luck);
        _ ->
            ok
    end.
\end{verbatim}

We also add seeding of the random number generator when starting a
process so that we will not have all processes crashing at the same
time. The initialization is for example done as follows, the slave
will be initialized in a similar manner. 

\begin{verbatim}
start(Id) ->
    Rnd = random:uniform(1000),
    Self = self(),
    spawn_link(fun()-> init(Id, Rnd, Self) end).

init(Id, Rnd, Master) ->
    random:seed(Rnd, Rnd, Rnd),
    leader(Id, Master, 0, []).
\end{verbatim}

Run some experiments and see if you can have the state of the workers
become out of synch. What is happening?

\subsection{reliable multicast}

To remedy the problem we could replace the basic multicaster with a
reliable multicaster. A process that would forward all messages before
delivering them to the higher layer. Using a vanilla reliable
multicaster would however be very costly, we could try a smarter
solution.

Assume that we keep a copy of the last message that we have seen from
the leader. If we detect the death of the leader it could be that it
died during the basic multicast procedure and that some nodes have not
seen the message. We will now make an assumption that we will discuss later:

\begin{itemize}
\item Messages are reliably delivered and thus,
\item if the leader sends a message to A and then B and B receives the
  message, then also A will receive the message.
\end{itemize}

The leader is sending messages to the peers in the order that they
occur in the list of peers. If anyone receives a message then the
first peer in the list receives the message. This means that only the
next leader needs to resend the message.

This will of course introduce the possibilities of doublets of
messages being received. In order to detect this we will number all
messages and only deliver new messages to the application layer.

Lets go through the changes that we need to make and create a new
module {\tt gms3} that implements these changes.

\begin{itemize}
\item {\tt slave(Id, Master, Leader, N, Last, Peers)}: the slave
  procedure is extended with two arguments: {\tt N} and {\tt
    Last}. {\tt N} is the expected sequence number of the next message
  and {\tt Last} is a copy of the last message (a regular message or a
  view) received from the leader.
\item {\tt election(Id, Master, N, Last, Peers)}: the election
  procedure is extended with the same two arguments.
\item {\tt leader(Id, Master, N, Peers)}: the leader procedure is
  extended with the the argument {\tt N}, the sequence number of the
  next message (regular message or view) to be sent.
\end{itemize}

The messages are also changed and will now contain the sequence number.

\begin{itemize}
\item {\tt \{msg, N, Msg\}}: a regular message with a sequence number.
\item {\tt \{view, N, State, Leader, View\}}: a view message with a sequence number.
\end{itemize}

We must also add clauses to the slave to accept and ignore duplicate
messages. If we do not remove these from the message queue they will
add up and after a year generate a very hard to handle trouble report.

When discarding messages we discard messages we only want to discard
messages that we have seen i.e. messages with a sequence number less
than $N$. We can do this by using the {\tt when} construction. For example:

\begin{verbatim}
{msg, I, _} when I < N ->
    slave(Id, Master, Leader, N, Last, Peers);
\end{verbatim}

You might wonder how a message possibly could arrive early but there
is a a small window where this could actually happen.

The crucial part is then in the election procedure where the elected
leader will forward the last received message to all peers in the
group. Hopefully this will be enough to keep slaves synchronized.

\begin{verbatim}
            bcast(Id, Last, Rest),
\end{verbatim}

This completes the transition and {\tt gms3} should be ready for
release. 

\subsection{some experiments}

Run some experiments and create a large group spanning several
computers. Can we keep a group rolling by adding more nodes as
existing nodes die?

Assuming all test went well we're ready to ship the product. There is
however one thing we need to mention and that is that our
implementation does not work. Well, it sort of works depending on what
the Erlang environment guarantees and how strong our requirements are.

\section{what could possibly go wrong}

The first thing we have to realize is what guarantees the Erlang
system actually gives on message sending. The specifications only
guarantee that messages are delivered in FIFO order, not that they
actually do arrive. We have built our system relying on reliable
delivery of messages, something that is not guaranteed.

How would we have to change the implementation to handle the possibly
lost messages? How would this impact performance?

The second reason why things will not work is that we rely on that the
Erlang failure detector is perfect i.e. that it will never suspect any
correct node for having crashed. Is this really the case? Can we adapt
the system so that it will behave correctly if it does make progress
even though it might not always make progress?

The third reason why things do not work is that we could have a
situation where one node delivers a message that will not be delivered
by any correct node. This could happen even if we had reliable send
operations and perfect failure detectors.  How could this happen and
how likely is it that it does?  What would a solution look like?


\pagebreak
\nnsection{gms1.erl}
\begin{verbatim}
-module(gms1).

-export([start/1, start/2]).

start(Id) ->
    Self = self(),
    spawn_link(fun()-> init(Id, Self) end).

init(Id, Master) ->
    leader(Id, Master, []).

start(Id, Grp) ->
    Self = self(),
    spawn_link(fun()-> init(Id, Grp, Self) end).    

init(Id, Grp, Master) ->
    Self = self(), 
    Grp ! {join, Self},
    receive
        {view, State, Leader, Peers} ->
            Master ! {ok, State},
            slave(Id, Master, Leader, Peers)
    end.

slave(Id, Master, Leader, Peers) ->    
    receive
        {mcast, Msg} ->
            Leader ! {mcast, Msg},
            slave(Id, Master, Leader, Peers);
        {join, Peer} ->
            Leader ! {join, Peer},
            slave(Id, Master, Leader, Peers);
        {msg, Msg} ->
            Master ! {deliver, Msg},
            slave(Id, Master, Leader, Peers);
        {view, _, _, View} ->
            slave(Id, Master, Leader,  View);       
        stop ->
            ok;
        Error ->
            io:format("gms ~w: slave, strange message ~w~n", [Id, Error])
    end.

leader(Id, Master, Peers) ->    
    receive
        {mcast, Msg} ->
            bcast(Id, {msg, Msg}, Peers),
            Master ! {deliver, Msg},
            leader(Id, Master, Peers);
        {join, Peer} ->
            Master ! request,
            joining(Id, Master, Peer, Peers);
        stop ->
            ok;
        Error ->
            io:format("gms ~w: leader, strange message ~w~n", [Id, Error])
    end.

joining(Id, Master, Peer, Peers) ->
    receive 
        {ok, State} ->
            Peers2 = lists:append(Peers, [Peer]),           
            bcast(Id, {view, State, self(), Peers2}, Peers2),
            leader(Id, Master, Peers2)
    end.

bcast(_, Msg, Nodes) ->
    lists:foreach(fun(Node) -> Node ! Msg end, Nodes).
\end{verbatim}
\pagebreak
\nnsection{worker.erl}
\begin{verbatim}
-module(worker).

-export([start/4, start/5]).

-define(change, 20).
-define(color, {0,0,0}).

start(Id, Module, Rnd, Sleep) ->
    spawn(fun() -> init(Id, Module, Rnd, Sleep) end).

init(Id, Module, Rnd, Sleep) ->
    Cast = apply(Module, start, [Id]),
    Color = ?color,
    init_cont(Id, Rnd, Cast, Color, Sleep).

start(Id, Module, Rnd, Peer, Sleep) ->
    spawn(fun() -> init(Id, Module, Rnd, Peer, Sleep) end).

init(Id, Module, Rnd, Peer, Sleep) ->
    Cast = apply(Module, start, [Id, Peer]),
    receive
        {ok, Color} ->
            init_cont(Id, Rnd, Cast, Color, Sleep);
        {error, Error} ->
            io:format("error: ~s~n", [Error])
    end.

init_cont(Id, Rnd, Cast, Color, Sleep) ->
    random:seed(Rnd, Rnd, Rnd),
    Gui = gui:start(Id, self()),
    Gui ! {color, Color}, 
    worker(Id, Cast, Color, Gui, Sleep),
    Cast ! stop,
    Gui ! stop.

worker(Id, Cast, Color, Gui, Sleep) ->
    Wait = if Sleep == 0 -> 0; true -> random:uniform(Sleep) end,
    receive
        {deliver, {_From, N}} ->
            Color2 = change_color(N, Color),
            Gui ! {color, Color2},
            worker(Id, Cast, Color2, Gui, Sleep);
        {join, Peer} ->
            Cast ! {join, Peer},
            worker(Id, Cast, Color, Gui, Sleep);            
        request ->
            Cast ! {ok, Color},
            worker(Id, Cast, Color, Gui, Sleep);
        stop ->
            ok;
        Error ->
            io:format("strange message: ~w~n", [Error]),
            worker(Id, Cast, Color, Gui, Sleep)
    after Wait ->
            Cast !  {mcast, {Id, random:uniform(?change)}},
            worker(Id, Cast, Color, Gui, Sleep)     
    end.

change_color(N, {R,G,B}) ->
    {G, B, ((R+N) rem 256)}.
\end{verbatim}
\pagebreak
\nnsection{gui.erl}
\begin{verbatim}
-module(gui).

-define(width, 200).
-define(height, 200).
-define(bg, black).

-export([start/2]).

start(Id, Master) ->
    spawn_link(fun() -> init(Id, Master) end).

init(Id, Master) ->
    Win = gs:window(gs:start(),[{map,true},{title, Id}, {bg, ?bg}, 
      {width,?width},{height,?height}]),
    loop(Win, Master).

loop(Win, Master)->
    receive
        {color, Color} ->
            color(Win, Color),
            loop(Win, Master);
        {gs,_,destroy,[],[]} ->
            Master ! stop,
            ok;
        stop ->
            ok;
        Error ->
            io:format("gui: strange message ~w ~n", [Error]),
            loop(Win, Master)   
    end.

color(Win, Color) ->
    gs:config(Win, [{bg, Color}]).
\end{verbatim}
\end{document}

